\documentclass{book}
\usepackage{amsmath}
\usepackage{amsfonts}
\usepackage{amssymb}
\usepackage[mathscr]{eucal}
\usepackage{amsthm}
\usepackage{makeidx}
\usepackage{ifthen}
%\usepackage[dvips]{graphicx}    %change to dvips for production
\usepackage{graphicx}
\makeindex

%\setlength{\headsep}{0.4truein}             % Sizes for a 6 inch by 9 inch version
%\setlength{\topmargin}{-0.3 true in}
%\setlength{\topskip}{0 true in}
%\setlength{\textwidth}{4.5 true in}
%\setlength{\oddsidemargin}{-0.1 true in}
%\setlength{\evensidemargin}{-0.4 true in}
%\setlength{\textheight}{6.8 true in}

%\special{papersize=6in,9in}      % This is processed by dvips/dvipdf

%\newcommand{\breakSixByNine}{\newpage}  % for inserting a forced line break in the 6-by-9 version
                                     % Version 2.1.  (Also see definition of \eps below)
                                     % Used only once, in turing.tex
                                     
\newcommand{\breakSixByNine}{}

% FOR 6-BY-9 version, an extra scaling factor of 0.8 is added to the \eps command.

%\newcommand*{\eps}[1]{\scalebox{0.8}{\includegraphics{figures/#1.eps}}}

\newcommand*{\eps}[1]{\includegraphics{figures/#1.eps}}

\newcommand*{\scaledeps}[2]{\resizebox{#1}{!}{\eps{#2}}}

\newtheorem{theorem}{Theorem}[chapter]       % theorems,lemmas corrolaries
\newtheorem{lemma}[theorem]{Lemma}           % numbered together, by chapter
\newtheorem{corrolary}[theorem]{Corollary}

\theoremstyle{definition}
\newtheorem{definition}{Definition}[chapter]
\newtheorem{example}{Example}[chapter]



%\nw is "new word"
%\nw{text} sets text in slanted font and puts it in the index
%\nw[none]{text} sets text in slanted font, but does not put it in the index
%\nw[index-entry]{text} puts text in slanted font and puts index-entry in the index
\newcommand*{\nw}[2][]{\textsl{\textbf{#2}}\ifthenelse{\equal{#1}{}}{\index{#2}}{\ifthenelse{\equal{#1}{none}}{}{\index{#1}}}}


\newcommand*{\largesize}[1]{{\Large#1}}
\newcommand*{\startchapter}[1]{\textsc{\largesize #1}}  % for first few words


\newcommand{\AND}{\wedge}
\newcommand{\OR}{\vee}
\newcommand{\NOT}{\lnot}
\newcommand{\IMP}{\rightarrow}
\newcommand{\IFF}{\leftrightarrow}
\newcommand{\XOR}{\oplus}
\newcommand{\T}{\mathbb{T}}
\newcommand{\F}{\mathbb{F}}
\newcommand{\LOGIMP}{\Longrightarrow}

%For Chapter 2
\newcommand{\N}{{\mathbb N}}
\newcommand{\Z}{{\mathbb Z}}
\newcommand{\Q}{{\mathbb Q}}
\newcommand{\R}{{\mathbb R}}
\newcommand{\Zpos}{{{\mathbb Z}^+}}
\newcommand{\SUB}{\subseteq}
\newcommand{\SUP}{\supseteq}
\newcommand{\PSUB}{\varsubsetneq}
\newcommand{\PSUP}{\varsupsetneq}
\newcommand{\SETDIFF}{\smallsetminus}   % the set difference operator.
\newcommand{\POW}{{\mathscr P}}   % For the power set of a set
\newcommand{\st}{\,|\,}   % for the | in sets { x | P(x) }


\newcounter{problemcounter}
\newcounter{partcounter}[problemcounter]

\newcommand{\Item}[1]{\par\hangafter=0
                         \hangindent=15pt
                         \noindent\llap{#1}\ignorespaces}
\newcommand{\IItem}[1]{\par\hangafter=0
                         \hangindent=40pt
                         \noindent\llap{#1}\ignorespaces}

\newcommand{\problem}{\smallskip\refstepcounter{problemcounter}\Item{\bfseries\arabic{problemcounter}.\ }}
\newcommand{\ppart}{\stepcounter{partcounter}\IItem{\bfseries\alph{partcounter})\ }}
\newcommand{\pparts}[1]{\vskip\parskip
   \halign{\hskip40pt\stepcounter{partcounter}\llap{\bfseries\alph{partcounter})\ }$##$\qquad\hfil&&
            \hskip30pt\stepcounter{partcounter}\llap{\bfseries\alph{partcounter})\ }$##$\qquad\hfil\cr
   #1\crcr}}
%\tparts is same as \pparts, but doesn't use math mode
\newcommand{\tparts}[1]{\vskip\parskip
   \halign{\hskip40pt\stepcounter{partcounter}\llap{\bfseries\alph{partcounter})\ }##\qquad\hfil&&
            \hskip30pt\stepcounter{partcounter}\llap{\bfseries\alph{partcounter})\ }##\qquad\hfil\cr
   #1\crcr}}
\newenvironment{exercises}
   {\setcounter{problemcounter}{0}
    \bigbreak\medbreak
    \leftline{\bfseries\large Exercises}
    \medskip
    \small}
   {}
    
    
\renewcommand{\strut}{\vrule width 0pt depth 1ex height 2.5ex}
\newcommand{\bigstrut}{\vrule width 0pt depth 1ex height 3ex}

%arguments are: label,caption,figure data
\newcommand{\fig}[3]{\begin{figure}[t]
  \begin{center}
    #3
  \parbox{0.9\textwidth}{\textit{\caption{\label{#1}#2}}}
  \end{center}
  \end{figure}
}

%Added for Chapter 6

\newcommand{\EMPTYSTRING}{\varepsilon}
\newcommand{\PRODUCES}{\longrightarrow}
\newcommand{\YIELDS}{\Longrightarrow}
\newcommand{\YIELDSTAR}{{\Longrightarrow}^*}
\newcommand{\NT}[1]{\hbox{$\langle$\textit{#1}$\rangle$}}
\newcommand{\BNFPRODUCES}{\hbox{\texttt{::=}}}
\newcommand{\BNFALT}{\hbox{$|$}}

%BTH
\newcommand{\engAND}{\mathbin{\textsc{and}}}
\newcommand{\engOR}{\mathbin{\textsc{or}}}
\newcommand{\engXOR}{\mathbin{\textsc{xor}}}
\newcommand{\engNOT}{\mathop{\textsc{not}}}
\newcommand{\0}{\ensuremath{\mathtt{0}}}
\newcommand{\1}{\ensuremath{\mathtt{1}}}

% Use commands within \Verb and \Verbatim
\usepackage{fancyvrb}
\fvset{commandchars=\\\{\},commentchar=none}
\fvset{fontsize=\normalsize} % The font size of all verbatim text can be changed here

% Typeset #1 centered on the space that #2 would have occupied
\newcommand{\CenterSpace}[2]{\savebox{0}{#2}\savebox{2}{}\ht2=\ht0 \dp2=\dp0 \wd2=0.5\wd0
    \usebox{2}\makebox[0pt][c]{#1}\usebox{2}}

\newcommand{\Caret}{{\CenterSpace{\raisebox{-0.5ex}{\Large\textasciicircum}}{x}}}
%\newcommand{\bthTilde}{{\ \llap{\Large\textasciitilde}}{}}
\newcommand{\bthTilde}{\raisebox{-1ex}{\Large\textasciitilde}}


\newenvironment{tailquote}{\par\hfill\begin{minipage}[t]{0.7\textwidth}\raggedright\itshape}{\end{minipage}}

%\setkeys{Gin}{width=\linewidth,totalheight=\textheight,keepaspectratio} % Improves figure scaling

\usepackage{color}
\usepackage{hyperref}
\hypersetup{
    colorlinks=true, % make the links colored
    linkcolor=blue, % color TOC links in blue
    urlcolor=red, % color URLs in red
    linktoc=all % 'all' will create links for everything in the TOC
}

\usepackage[type={CC}, modifier={by-sa}, version={4.0}]{doclicense}

\usepackage{microtype} % Improves character and word spacing

\usepackage{ifsym} % \textifsym{0123456789} gives 7-segment display characters

\usepackage{tikz}
\usetikzlibrary{calc,arrows,automata}
\usepackage{tikz-timing}

\newcommand{\tikzmark}[1]{\tikz[overlay,remember picture] \node (#1) {};}
\newcommand{\DrawBox}[3][]{%
    \tikz[overlay,remember picture]{%
        \coordinate (TopLeft)     at ($(#2)+(-0.2em,0.9em)$);
        \coordinate (BottomRight) at ($(#3)+(0.2em,-0.3em)$);
        %
        \draw [black,thick,rounded corners,#1] (TopLeft) rectangle (BottomRight);
    }
}
\newcommand{\DrawBoxS}[3][]{%
    \tikz[overlay,remember picture]{%
        \coordinate (TopLeft)     at ($(#2)+(-0.2em,0.9em)$);
        \coordinate (BottomRight) at ($(#3)+(0.2em,-0.3em)$);
        \coordinate (BottomLeft)  at (TopLeft |- BottomRight);
        \coordinate (TopRight)    at (TopLeft -| BottomRight);
        %
        \draw [black,thick,rounded corners,#1] (BottomLeft) |- ($(TopLeft)!0.5!(TopRight)$) -| (BottomRight);
    }
}
\newcommand{\DrawBoxN}[3][]{%
    \tikz[overlay,remember picture]{%
        \coordinate (TopLeft)     at ($(#2)+(-0.2em,0.9em)$);
        \coordinate (BottomRight) at ($(#3)+(0.2em,-0.3em)$);
        \coordinate (BottomLeft)  at (TopLeft |- BottomRight);
        \coordinate (TopRight)    at (TopLeft -| BottomRight);
        %
        \draw [black,thick,rounded corners,#1] (TopLeft) |- ($(BottomLeft)!0.5!(BottomRight)$) -| (TopRight);
    }
}
\newcommand{\DrawBoxW}[3][]{%
    \tikz[overlay,remember picture]{%
        \coordinate (TopLeft)     at ($(#2)+(-0.2em,0.9em)$);
        \coordinate (BottomRight) at ($(#3)+(0.2em,-0.3em)$);
        \coordinate (BottomLeft)  at (TopLeft |- BottomRight);
        \coordinate (TopRight)    at (TopLeft -| BottomRight);
        %
        \draw [black,thick,rounded corners,#1] (TopLeft) -| ($(TopRight)!0.5!(BottomRight)$) |- (BottomLeft);
    }
}
\newcommand{\DrawBoxE}[3][]{%
    \tikz[overlay,remember picture]{%
        \coordinate (TopLeft)     at ($(#2)+(-0.2em,0.9em)$);
        \coordinate (BottomRight) at ($(#3)+(0.2em,-0.3em)$);
        \coordinate (BottomLeft)  at (TopLeft |- BottomRight);
        \coordinate (TopRight)    at (TopLeft -| BottomRight);
        %
        \draw [black,thick,rounded corners,#1] (TopRight) -| ($(TopLeft)!0.5!(BottomLeft)$) |- (BottomRight);
    }
}


\begin{document}

\frontmatter

\begin{titlepage}

\vglue 1.5 in 
\centerline{\Huge \textit{Foundations of Computation}}
\vskip 0.5 in
\centerline{Second Edition (Version 2.4, Summer 2019)}
\vskip 2 in
\centerline{\Large Carol Critchlow and David Eck}
\medskip
\centerline{Department of Mathematics and Computer Science}
\smallskip
\centerline{Hobart and William Smith Colleges}
\smallskip
\centerline{Geneva, New York\quad 14456}
\vskip 0.5 in
\centerline{\Large Additional Material by Brian T. Howard}
\medskip
\centerline{Department of Computer Science}
\smallskip
\centerline{DePauw University}
\smallskip
\centerline{Greencastle, Indiana\quad 46135}



\newpage
\vglue 3.5 true in


\noindent \copyright\ 2011, Carol Critchlow and David Eck.\\
Some material \copyright\ 2019, Brian T. Howard.
\bigskip

\noindent \startchapter{Foundations of Computation} is a textbook for a one semester introductory course
in theoretical computer science.  It includes topics from discrete mathematics,
automata theory, formal language theory, and the theory of computation, along with
practical applications to computer science. It has no prerequisites other than a
general familiarity with computer programming.  Version 2.3, Summer 2010, was a minor 
update from Version 2.2, which was published in Fall 2006.  Version 2.3.1 is an even
smaller update of Version 2.3.  Version 2.3.2 is identical to Version 2.3.1 except for
a change in the license under which the book is released. Version 2.4, Summer 2019,
incorporates additional material and minor edits by Brian T. Howard.

\bigskip


%\small{%\advance\rightskip by 0pt plus 30 pt minus 5 pt
%\noindent This book can be redistributed in unmodified form,
%or in modified form with proper attribution and under the same license as the original, 
%for non-commercial uses only, as specified by the
%\textit{Creative Commons Attribution-Noncommercial-ShareAlike 4.0 License}.
%(See http://creativecommons.org/licenses/by-nc-sa/4.0/)\\[8pt]

\doclicenseThis\index{license}

\noindent{\small Version 2.3.2 of this textbook is available for on-line use and for free download
in PDF form at \url{http://math.hws.edu/FoundationsOfComputation/}
and it can be purchased in print form for the cost of reproduction
plus shipping at \url{http://www.lulu.com}.}






\end{titlepage}

\tableofcontents

%\include{preface}
%\include{introduction}


\mainmatter

\include{logic}
\include{sets}
% !TEX root = main.tex

\newcommand{\ab}{\mbox{$\{a,b\}$}}
\newcommand{\aetc}[2]{\mbox{{${#1}_1{#1}_2\ldots {#1}_{#2}$}}}
\newcommand{\varep}{\varepsilon}
\newcommand{\fsafig}[1]{\medskip\centerline{\eps{fsa#1}}\medskip}

\newcommand{\REOR}{\hbox{$\,|\,$}}

\chapter[Regular Expressions and FSA's]{Regular Expressions and Finite-State Automata}

\startchapter{With the set of mathematical tools } 
from the first two chapters, we
are now ready to study \nw{languages} and \nw{formal language theory}.
Our intent is to examine the question of how, and which, languages
can be mechanically generated and recognized; and, ultimately, to see
what this tells us about what computers can and can't do.

\bigskip
\section{Languages}
In formal language theory, an \nw{alphabet} is a finite, non-empty 
set.  The elements of the set are called \nw{symbols}.  A finite 
sequence of symbols $a_1a_2\ldots a_n$
from an alphabet is called a \nw{string} over that alphabet.  

\smallskip

\begin{example}
$\Sigma = \{0,1\}$ is an alphabet, and {\em 011}, 
{\em 1010}, and {\em 1} are all strings over $\Sigma$.
\end{example}

\smallskip

Note that strings really are \emph{sequences} of symbols, which 
implies that 
order matters.  Thus {\em 011}, {\em 101}, and {\em 110} are all 
different strings, though they are made up of the same symbols.
The strings $x=\aetc{a}{n}$ and $y=\aetc{b}{m}$ are \nw{equal} only
if $m=n$ (i.e.\ the strings contain the same number of symbols) and 
$a_i=b_i$ for all
$1 \leq i \leq n$.

Just as there are operations defined on numbers, truth values, sets,
and other mathematical entities, there are operations defined on
strings.  Some important operations are:
\begin{enumerate}
\item {\em length}: the \nw{length} of a string $x$ is the number of symbols
in it.  The notation for the length of $x$ is $|x|$.  Note that
this is consistent with other uses of $|\ |$, all of which 
involve some notion of size: $|number|$ measures how big a number
is (in terms of its distance from 0);  $|set|$ measures the size
of a set (in terms of the number of elements).

We will occasionally refer to a {\em length-n string}.  This is a
slightly awkward, but concise, shorthand for ``a string whose length
is $n$".

\item {\em concatenation}: the \nw{concatenation} of two strings $x=a_1
a_2\ldots a_m$ and $y=b_1b_2\ldots b_n$ is the sequence of symbols
$a_1\ldots a_mb_1\ldots b_n$.  Sometimes $\cdot$ is used to denote
concatenation, but it is far more usual to see the concatenation of 
$x$ and $y$ denoted by $xy$ than by $x\cdot y$.  You can easily
convince yourself that concatenation is associative (i.e.\ $(xy)z =
x(yz)$ for all strings $x,y$ and $z$.)  Concatenation is not
commutative (i.e.\ it is not always true that $xy = yx$:
for example, if $x=a$ and $y=b$ then $xy=ab$ while $yx=ba$ and, as
discussed above, these strings are not equal.)

\item {\em reversal}: the \nw{reverse} of a string $x=a_1a_2\ldots a_n$ is
the string $x^R = a_na_{n-1}\ldots a_2a_1$.
\end{enumerate}

\begin{example}Let $\Sigma = \ab$, $x=a$, $y=abaa$, and $z=bab$.
Then $|x| = 1$, $|y| = 4$, and $|z|=3$.  Also, $xx = aa$, $xy =
aabaa$, $xz = abab$, and $zx = baba$.  Finally, $x^R = a$,
$y^R = aaba$, and $z^R=bab$.
\end{example}

\smallskip

By the way, the previous example illustrates a naming convention standard
throughout language theory texts: if a letter is
intended to represent a single symbol in an alphabet, the convention
is to use a letter from the beginning of the English alphabet ({\em a,
b, c, d }); if a letter is intended to represent a string, the 
convention is to use a letter from the end of the English alphabet
({\em u, v, } etc).

\bigskip

In set theory, we have a special symbol to designate the set that 
contains no elements.  Similarly, language theory has a special 
symbol $\varepsilon$ which is used to represent the \nw{empty string}, the
string with no 
symbols in it.  (Some texts use the symbol $\lambda$ instead.)
It is worth noting that $|\varep| = 0$, that $\varep^R = \varep$,
and that $\varep \cdot x = x \cdot \varep = x$ for all strings $x$.
(This last fact may appear a bit confusing.  Remember that $\varep$
is not a symbol in a string with length 1, but rather the name given
to the string made up of 0 symbols.  Pasting those 0 symbols onto the
front or back of a string $x$ still produces $x$.) 

\bigskip

The set of all strings over an alphabet $\Sigma$ is denoted $\Sigma^*$.
(In language theory, the symbol $^*$ is typically used to denote ``zero
or more'', so $\Sigma^*$ is the set of strings made up of zero or 
more symbols from $\Sigma$.)  Note that while an alphabet 
$\Sigma$ is by 
definition a \emph{finite} set of symbols, and strings are by
definition \emph{finite} sequences of those symbols, the set $\Sigma^*$
is \emph{always infinite}.  Why is this?  Suppose $\Sigma$ contains $n$
elements.  Then there is one string over $\Sigma$ with 0 symbols,
$n$ strings with 1 symbol, $n^2$ strings with 2 symbols (since there
are $n$ choices for the first symbol and $n$ choices for the second),
$n^3$ strings with 3 symbols, etc.

\smallskip

\begin{example} If $\Sigma = \{1\}$, then $\Sigma^* = \{\varep,
1, 11, 111, \ldots\}$.  If $\Sigma = \ab$, then $\Sigma^* = \{
\varep, a, b, aa, ab, ba, bb, aaa, aab, \ldots\}$.
\end{example}

\smallskip

Note that $\Sigma^*$ is \emph{countably} infinite: if we list the strings as in
the preceding example (length-0 strings, length-1 strings in ``alphabetical"
order, length-2 strings similarly ordered, etc) then any string over $\Sigma$
will eventually appear.  (In fact, if $|\Sigma| = n \geq 2$ and $x \in \Sigma^*$ has
length $k$, then $x$ will appear on the list within the first $\frac{n^{k+1} -
1}{n-1}$ entries.)

\bigskip

We now come to the definition of a \nw{language} in the formal language
theoretical sense.


\begin{definition} A language over an alphabet $\Sigma$ is a subset
of $\Sigma^*$.  Thus, a language over $\Sigma$ is an element of
${\cal P}(\Sigma^*)$, the power set of $\Sigma^*$.
\end{definition}

\smallskip
In other words, any set of strings (over alphabet $\Sigma$) constitutes a
language (over alphabet $\Sigma$).

\smallskip

\begin{example} Let $\Sigma = \{0,1\}$.  Then the following are all
languages over $\Sigma$:

$L_1 = \{011, 1010, 111\}$

$L_2 = \{0, 10, 110, 1110, 11110, \ldots\}$

$L_3 = \{x \in \Sigma^* \ | \ n_0(x) = n_1(x) \}$, where the notation 
\nw{$n_0(x)$}
stands for the 

\ \ \ number of 0's in the string $x$, and similarly for $n_1(x)$.

$L_4 = \{x \ | \ \mbox{\ $x$ represents a multiple of 5 in binary}\}$
\end{example}

\smallskip

Note that languages can be either finite or infinite.
Because $\Sigma^*$ is infinite, it clearly has an
infinite number of subsets, and so there are an infinite number of languages
over $\Sigma$.  But are there countably or uncountably many such languages?

\smallskip

\begin{theorem}
For any alphabet $\Sigma$, the number of languages over $\Sigma$ is
uncountable.
\end{theorem}

\smallskip
This fact is an immediate consequence of the result, proved in a previous
chapter, that the power set of a countably infinite set is uncountable.  Since
the elements of ${\cal P}(\Sigma)$ are exactly the languages over $\Sigma$,
there are uncountably many such languages.

\medskip

Languages are sets and therefore, as for any sets, it makes sense to talk about
the union, intersection, and complement of languages.  (When taking the complement
of a language over an alphabet $\Sigma$, we always consider the univeral set
to be $\Sigma^*$, the set of all strings over~$\Sigma$.)
Because languages are
sets of strings, there are additional operations that can be defined on
languages, operations that would be meaningless on more general sets.  For
example, the idea of concatenation can be extended from strings to languages.

For two sets of strings $S$ and $T$, we define the \nw{concatenation} of $S$ and
$T$ (denoted $S\cdot T$ 
or just $ST$) to
be the set $ST = \{ st \ | \ s \in S \AND t \in T \}$.  For example, if $S =
\{ab, aab\}$ and $T=\{\varep, 110, 1010\}$, then 
$ST = \{ab,  ab110,  ab1010,  aab,  aab110,  aab1010\}$.  
Note in particular that $ab \in ST$, because $ab \in S$, $\varep \in T$, and
$ab \cdot \varep = ab$.
Because 
concatenation of sets is defined in terms of the
concatenation of the
strings that the sets contain, concatenation of sets is associative
and not commutative.  (This can easily be verified.)  

When a set $S$
is concatenated with itself, the notation $SS$ is usually scrapped
in favour of $S^2$; if $S^2$ is concatenated with $S$, we write
$S^3$ for the resulting set, etc.  So $S^2$ is the set of all strings formed by
concatenating two (possibly different, possibly identical) strings from $S$,
$S^3$ is the set of strings formed by concatenating three strings from $S$,
etc.  Extending this notation, we take $S^1$ to be the set of strings formed
from one string in $S$ (i.e.\ $S^1$ is $S$ itself), and $S^0$ to be the set of
strings formed from zero strings in $S$ (i.e.\ $S^0 = \{\varep\}$).  If we take
the union $S^0 \cup S^1 \cup S^2 \cup \ldots$, then the resulting set is the set of
all strings formed by concatenating zero or more strings from $S$, and is
denoted $S^*$.  The set $S^*$ is called the \nw{Kleene closure} of $S$, and
the $^*$ operator is called the \nw{Kleene star} operator.

\smallskip

\begin{example}
Let $S = \{01, ba\}$.  Then

$S^0 = \{\varep\}$

$S^1 = \{01, ba\}$

$S^2 = \{0101, 01ba, ba01, baba\}$

$S^3 = \{010101, 0101ba, 01ba01, 01baba, ba0101, ba01ba, baba01, bababa\}$

etc, so

$S^* =\{\varep,01,ba,0101,01ba,ba01,baba,010101,0101ba,\ldots\}.$
\end{example}
 
\smallskip

Note that this is the second time we have seen the notation $something^*$.  We
have previously seen that for an alphabet $\Sigma$, $\Sigma^*$ is defined to be 
the set of all
strings over $\Sigma$.  If you think of $\Sigma$ as being a set of length-1
strings, and take its Kleene closure, the result is once again the set of all
strings over $\Sigma$, and so the two notions of $^*$ coincide.

\smallskip

\begin{example}
Let $\Sigma = \ab$.  Then

$\Sigma^0 = \{\varep\}$

$\Sigma^1 = \ab$

$\Sigma^2 = \{aa, ab, ba, bb\}$

$\Sigma^3 = \{aaa, aab, aba, abb, baa, bab, bba, bbb\}$

etc, so

$\Sigma^* =\{\varep,a,b,aa,ab,ba,bb,aaa,aab,aba,abb,baa,bab,\ldots\}.$
\end{example}

\begin{exercises}
\problem Let $S = \{\varep, ab, abab\}$ and $T = \{aa, aba, abba, abbba,
\ldots\}$.  Find the following:
\pparts{ S^2 & S^3 & S^* & ST & TS }
\problem The \nw{reverse} of a language $L$ is defined to be 
$L^R = \{ x^R \ | \ x \in L\}$.  Find $S^R$ and $T^R$ for the $S$ and $T$ in the
preceding problem.
\problem Give an example of a language $L$ such that $L=L^*$.

\end{exercises}

\section{Regular Expressions}

Though we have used the term {\em string} throughout to refer to a sequence of
symbols from an alphabet, an alternative term that is frequently used is {\em
word}.  The analogy seems fairly obvious: strings are made up of ``letters"
from an alphabet, just as words are in human languages like English.
In English, however, there are no particular rules specifying which sequences 
of letters can be used to form legal English words---even unlikely
combinations like {\em ghth} and {\em ckstr} have their place.  
While some formal languages may simply
be random collections of arbitrary strings, more interesting languages
are those where the strings in the language all share some 
common structure:  $L_1 = \{ x\in \ab^* \ | n_a(x) =
n_b(x)\}$; $L_2 = \{\mbox{legal Java identifiers}\}$; $L_3 = \{\mbox{legal C++
programs}\}$.  In all of these languages, there are structural 
rules which determine which sequences of symbols are in the language and which
aren't.
So
despite the terminology of ``alphabet" and ``word" in formal
language theory, the concepts don't necessarily match ``alphabet"
and ``word" for human languages.  A better parallel is to think of
the {\em alphabet} in a formal language as corresponding to the {\em words} in a
human language; the {\em words} in a formal language correspond to
the {\em sentences} in a human language, as there are rules ({\em grammar 
rules}) which determine how they can legally be constructed.

One way of describing the grammatical structure of the strings in a language is
to use a mathematical formalism called a \nw{regular expression}.  A regular
expression is a pattern that ``matches" strings that have a particular form.  For
example, consider the language (over alphabet $\Sigma = \ab$) $L= \{x \ | \ x
\mbox{\ starts and ends with\ } a\}$.  What is the symbol-by-symbol 
structure of
strings in this language?  Well, they start with an $a$, followed by zero or more
$a$'s or $b$'s or both, followed by an $a$.  The regular expression 
$a \cdot (a \REOR b)^* \cdot a$ is a pattern that captures this structure and matches any string in
$L$ ($\cdot$ and $^*$ have their usual meanings, and $\REOR $ designates {\em or}.\footnote{Various
symbols have been used to represent the ``or'' operation in regular expressions.  Both
$+$ and $\cup$ have been used for this purpose.  In this book, we use the symbol $|$ because
it is commonly used in computer implementations of regular expressions.}) 
Conversely, consider the regular expression ($a\cdot(a\REOR b)^*) \REOR  ((a\REOR b)^*\cdot a)$.
This is a pattern that matches any string that either has the form ``$a$ followed
by zero or more $a$'s or $b$'s or both" (i.e.\ any string that starts with an $a$)
{\em or} has the form ``zero or more $a$'s or $b$'s or both followed by an $a$"
(i.e.\ any string that ends with an $a$).  Thus the regular expression 
{\em generates} the language of all strings that start or end (or both) in an
$a$: this is the set of strings that match the regular expression. 

Here are the formal definitions of a regular expression and the language
generated by a regular expression:

\begin{definition}
Let $\Sigma$ be an alphabet.  Then the 
following patterns are \nw{regular expressions} over $\Sigma$:
\begin{enumerate}
\item $\Phi$ and $\varep$ are regular expressions;
\item $a$ is a regular expression, for each $a \in \Sigma$;
\item if $r_1$ and $r_2$ are regular expressions, then so are
$r_1\REOR r_2$, $r_1\cdot r_2$, $r_1^*$ and $(r_1)$ (and of course, $r_2^*$
and $(r_2)$).
As in concatenation of strings, the $\cdot$ is often left out of
the second expression.  (Note: the order of precedence of operators, from lowest to highest,
is $\REOR $, $\cdot$, $*$.)
\end{enumerate}
No other patterns are regular expressions.
\end{definition}

\begin{definition}\label{lgbre}
The \nw{language generated by a regular expression $r$}, 
denoted $L(r)$,
is defined as follows:
\begin{enumerate}
\item $L(\Phi) = \emptyset$, i.e.\ no strings match $\Phi$;
\item $L(\varep) = \{\varep\}$, i.e.\ $\varep$ matches only the 
empty string;
\item $L(a) = \{a\}$, i.e.\ $a$ matches only the string $a$;
\item $L(r_1\REOR r_2) = L(r_1) \cup L(r_2)$, i.e.\ $r_1\REOR r_2$ matches
strings that match $r_1$ or $r_2$ or both;
\item $L(r_1r_2) = L(r_1)L(r_2)$, i.e.\ $r_1r_2$ matches strings of the form 
``something that matches $r_1$ followed by something that 
matches $r_2$";
\item $L(r_1^*) = (L(r_1))^*$, i.e.\ $r_1^*$ matches sequences of 0
or more strings each of which matches $r_1$.
\item $L((r_1)) = L(r_1)$, i.e.\ $(r_1)$ matches exactly those strings
matched by $r_1$.
\end{enumerate}
\end{definition}

\begin{example}
Let $\Sigma = \ab$, and consider the regular expression $r=a^*b^*$.  What is
$L(r)$?  Well, $L(a) = \{a\}$ so $L(a^*) = (L(a))^* = \{a\}^*$, and $\{a\}^*$ is
the set of all strings of zero or more $a$'s, so $L(a^*) = \{\varep, a, aa, aaa,
\ldots\}$.  Similarly, $L(b^*) = \{\varep, b, bb, bbb, \ldots\}$.  
Since  $L(a^*b^*) = L(a^*)L(b^*) = \{xy \ | \ x\in L(a^*)\AND y\in L(b^*)\}$, we
have $L(a^*b^*) = \{\varep, a, b, aa, ab, bb, aaa, aab, abb, bbb, \ldots\}$,
which is the set of all strings of the form ``zero or more $a$'s followed by zero
or more $b$'s".
\end{example}

\begin{example}
Let $\Sigma = \ab$, and consider the regular expression $r=(a\REOR aa\REOR aaa)(bb)^*$.
Since $L(a) = \{a\}$, $L(aa) = L(a)L(a) = \{aa\}$.  Similarly, $L(aaa) = \{aaa\}$
and $L(bb) = \{bb\}$.  Now $L(a\REOR aa\REOR aaa) = L(a) \cup L(aa) \cup L(aaa) = \{a, aa,
aaa\}$, and $L((bb)^*) = (L((bb)))^* = (L(bb))^*$  (the last equality is from
clause 7 of Definition~\ref{lgbre}), and $(L(bb))^* = \{bb\}^* = \{\varep, bb,
bbbb, \ldots\}$.  So $L(r)$ is the set of strings formed by
concatenating $a$ or $aa$ or $aaa$ with zero or more pairs of $b$'s.
\end{example}

\begin{definition}
A language is {\em regular} if it is generated by
a regular expression.
\end{definition}

Clearly the union of two regular languages is regular; likewise, 
the concatenation of regular languages is regular; and the Kleene
closure of a regular language is regular. It is less clear whether the
intersection of regular languages is always regular; nor is it clear whether the
complement of a regular language is guaranteed to be regular.  These are
questions that will be taken up in Section~\ref{S-fsa-3}.

Regular languages, then, are languages whose strings' structure can be described
in a very formal, mathematical way.  The fact that a language can be
``mechanically" described or generated means that we are likely to be
able to get a computer to recognize strings in that language.
We will pursue the question of mechanical language recognition in
Section~\ref{S-fsa-1}, and subsequently will see that our first attempt to model mechanical
language recognition does in fact produce a family of ``machines" that recognize
exactly the regular languages.  But first, in the next section, we will look at some
practical applications of regular expressions.

\begin{exercises}
\problem Give English-language descriptions of the languages generated by the
following regular expressions.
\pparts{ (a\REOR b)^* & a^*\REOR b^* & b^*(ab^*ab^*)^* & b^*(abb^*)}
\problem Give regular expressions over $\Sigma=\ab$ that generate the 
following languages.
\ppart $L_1 = \{ x \ | \ x \mbox{ contains 3 consecutive $a$'s}\}$
\ppart $L_2 = \{ x \ | \ x \mbox{ has even length}\}$
\ppart $L_3 = \{ x \ | \ n_b(x) = 2 \bmod{3}\}$
\ppart $L_4 = \{ x \ | \ x \mbox{ contains the substring } aaba\}$
\ppart $L_5 = \{ x \ | \ n_b(x) < 2 \}$
\ppart $L_6 = \{ x \ | \ x \mbox{ doesn't end in } aa\}$
\problem Prove that all finite languages are regular.

\end{exercises}


\section{Application: Using Regular Expressions}

\newcommand{\bk}{\char`\\}
\newcommand{\vb}{\char`\|}
\newcommand{\sol}{\char`\^}

A common operation when editing text is to search for a
given string of characters, sometimes with the purpose of
replacing it with another string.  Many ``search and replace''\index{search
and replace} facilities have the option of using regular expressions
instead of simple strings of characters.  A regular expression describes
a language, that is, a \textit{set} of strings.  We can think of a regular
expression as a \nw{pattern} that matches certain strings, namely all
the strings in the language described by the regular expression.
When a regular expression is used in a search operation, the
goal is to find a string that matches the expression.  This type
of \nw{pattern matching} is very useful.\index{regular expressions!and
pattern matching}

The ability to do pattern matching with regular expressions is provided
in many text editors, including \textit{jedit} and \textit{kwrite}.
Programming languages often come with libraries for working with
regular expressions.  Java (as of version 1.4) provides regular
expression handling though a package named \textit{java.util.regexp}.
C++ typically provides a header file named \textit{regexp.h} for
the same purpose.  In all these applications, many new notations are added to the syntax to make it
more convenient to use.  The syntax can vary from one implementation
to another, but most implementations include the capabilities
discussed in this section.

\medskip

In applications of regular expressions, the alphabet usually includes
all the characters on the keyboard.  This leads to a problem, because
regular expressions actually use two types of symbols:  symbols that
are members of the alphabet and special symbols such a ``\texttt{*}'' and ``\texttt{)}'' that
are used to construct expressions.  These special symbols, which
are not part of the language being described but are used in the
description, are called \nw{meta-characters}.  The problem is,
when the alphabet includes all the available characters, what do we
do about meta-characters?  If the language that we are describing 
uses the ``\texttt{*}'' character, for example, how can we represent the
Kleene star operation?

The solution is to use a so-called ``escape character,'' which is
usually the backslash,~\texttt{\bk}.  We agree, for example, that the notation
\texttt{\bk*} refers to the symbol \texttt{*} that is a member of
the alphabet, while \texttt{*} by itself is the meta-character
that represents the Kleene star operation.  Similarly,
\texttt{(} and \texttt{)} are the meta-characters that are used
for grouping, while the corresponding characters in the language
are written as \texttt{\bk(} and \texttt{\bk)}.  For example,
a regular expression that matches the string \texttt{a*b} repeated
any number of times would be written: \texttt{(a\bk*b)*}.
The backslash is also used to represent certain non-printing
characters.  For example, a tab is represented as \texttt{\bk t}
and a new line character is \texttt{\bk n}.

%Outside this section of this book, *****
%we use the symbol + as a meta-character to represent
%a choice between alternatives in a regular expression.  In applications,
%however, the same operation is almost universally expressed using
%the vertical bar symbol~\texttt{\vb}, which computer scientists tend to
%associate with the word ``or.''  In this section, we follow the
%same convention and use \texttt{a\vb b} rather than \texttt{a+b} for 
%the regular expression that matches either \texttt{a} or
%\texttt{b}.  (This means, of course, that if we want to use
%\texttt{\vb} as a normal character rather than a meta-character, we must
%write it as~\texttt{\bk\vb}.  The same remark applies to all the new 
%meta-characters that are introduced below.)

We introduce two new common operations on regular expressions and two
new meta-characters to represent them.
The first operation is represented by the meta-character~\texttt{+}:
If \texttt{r} is a regular expression, then \texttt{r+} represents the
occurrence of \texttt{r} one or more times.  The second operation
is represented by~\texttt{?}: The notation \texttt{r?} represents an occurrence of \texttt{r} 
zero or one times.  That is to say, \texttt{r?} represents an optional 
occurrence of \texttt{r}.  Note that these operations are introduced
for convenience only and do not represent any real increase
in the power.  In fact, \texttt{r+} is exactly equivalent to
\texttt{rr*}, and \texttt{r?} is equivalent to \texttt{(r|$\varep$)} 
(except that in applications there is generally no equivalent to $\varep$).

To make it easier to deal with the large number of characters in the
alphabet, \nw{character classes} are introduced.  A character class
consists of a list of characters enclosed between brackets, \texttt{[} and
\texttt{]}.  (The brackets are meta-characters.)  A character class
matches a single character, which can be any of the characters in
the list.  For example, \texttt{[0123456789]} matches any one of
the digits 0 through 9.  The same thing could be expressed
as \texttt{(0\vb1\vb2\vb3\vb4\vb5\vb6\vb7\vb8\vb9)}, so once again
we have added only convenience, not new representational power.
For even more convenience, a hyphen can be included in a character
class to indicate a range of characters.  This means that
\texttt{[0123456789]} could also be written as \texttt{[0-9]}
and that the regular expression \texttt{[a-z]} will match any
single lowercase letter.  A character class can include multiple
ranges, so that \texttt{[a-zA-Z]} will match any letter, lower- or
uppercase.  The period~(\texttt{.}) is a meta-character that will
match any single character, except (in most implementations)
for an end-of-line.
These notations can, of course, be used in more complex
regular expressions.  For example, \texttt{[A-Z][a-zA-Z]*}
will match any capitalized word, and \texttt{\bk(.*\bk)} matches
any string of characters enclosed in parentheses.

In most implementations, the meta-character \texttt{\sol} can be used in
a regular expression to match the beginning of a line of text, so that
the expression \texttt{\sol [a-zA-Z]+} will only match a word that
occurs at the start of a line.  Similarly, \texttt{\$} is used
as a meta-character to match the end of a line.  Some implementations
also have a way of matching beginnings and ends of words.
Typically, \texttt{\bk b} will match such ``word boundaries.''
Using this notation, 
the pattern \texttt{\bk band\bk b} will match the string ``and''
when it occurs as a word, but will not match the \hbox{a-n-d}
in the word ``random.''  We are going a bit beyond
basic regular expressions here: Previously, we only thought of
a regular expression as something that either will match
or will not match a given string in its entirety.   When
we use a regular expression for a search operation, however,
we want to find a \textit{substring} of a given string that
matches the expression.  The notations \texttt{\sol},
\texttt{\$} and \texttt{\bk b} put a restrictions
on \textit{where} the matching substring can be located in the string.

\medskip

When regular expressions are used in search-and-replace operations,
a regular expression is used for the search pattern.  A search is
made in a (typically long) string for a substring that matches the pattern,
and then the substring is replaced by a specified replacement
pattern.  The replacement pattern is not used for matching
and is not a regular expression.  However, it can be more than
just a simple string.  It's possible to include parts of the
substring that is being replaced in the replacement string.
The notations \texttt{\bk0}, \texttt{\bk1}, \dots, \texttt{\bk9}
are used for this purpose.  The first of these, \texttt{\bk0},
stands for the entire substring that is being replaced.
The others are only available when parentheses are used in
the search pattern.  The notation \texttt{\bk1} stands for
``the part of the substring that matched the part of the
search pattern beginning with the first \texttt{(} in the
pattern and ending with the matching \texttt{)}.''  Similarly,
\texttt{\bk2} represents whatever matched the part of the
search pattern between the second pair of parentheses, and so on.

Suppose, for example, that you would like to search for
a name in the form \textit{last-name,~first-name} and
replace it with the same name in the form \textit{first-name last-name}.
For example, ``Reeves, Keanu'' should be converted to ``Keanu Reeves''.
Assuming that names contain only letters,
this could be done using the search pattern \texttt{([A-Za-z]+),~([A-Za-z]+)}
and the replacement pattern \texttt{\bk2 \bk1}.  When the match is
made, the first \texttt{([A-Za-z]+)} will match ``Reeves,'' 
so that in the replacement pattern, \texttt{\bk1} represents the
substring ``Reeves''. Similarly, \texttt{\bk2} will represent
``Keanu''.  Note that the parentheses
are included in the search pattern \textit{only} to specify what parts
of the string are represented by \texttt{\bk1} and \texttt{\bk2}.
In practice, you might use \texttt{\sol([A-Za-z]+),~([A-Za-z])\$}
as the search pattern to constrain it so that it will only 
match a complete line of text.  By using a ``global'' search-and-replace,
you could convert an entire file of names from one format to the other
in a single operation.

\medskip

Regular expressions are a powerful and useful technique that
should be part of any computer scientist's toolbox.  This section
has given you a taste of what they can do, but you should check
out the specific capabilities of the regular expression implementation
in the tools and programming languages that you use.




\begin{exercises}

\problem The backslash is itself a meta-character.  Suppose that
you want to match a string that contains a backslash
character.  How do you suppose you would represent the backslash in
the regular expression?

\problem Using the notation introduced in this section,
write a regular expression that could be used to match
each of the following:
\ppart Any sequence of letters (upper- or lowercase) that
includes the letter Z (in uppercase).
\ppart Any eleven-digit telephone number written in the form
\texttt{(xxx)xxx-xxxx}.
\ppart Any eleven-digit telephone number \textit{either}
in the form \texttt{(xxx)xxx-xxxx} or \texttt{xxx-xxx-xxxx}.
\ppart A non-negative real number with an optional decimal
part.  The expression should match numbers such as
17, 183.9999, 182., 0, 0.001, and 21333.2.
\ppart A complete line of  text that contains only letters.
\ppart A C++ style one-line comment consisting of \texttt{//} and all the
following characters up to the end-of-line.

\problem Give a search pattern and a replace pattern that could
be used to perform the following conversions:
\ppart Convert a string that is enclosed in a pair of double quotes to
the same string with the double quotes replaced by single quotes.
\ppart Convert seven-digit telephone numbers in the format
\texttt{xxx-xxx-xxxx} to the format \texttt{(xxx)xxx-xxxx}.
\ppart Convert C++ one-line comments, consisting of characters
between \texttt{//} and end-of-line, to C style comments enclosed
between \texttt{/*} and \texttt{*/}$\,$.
\ppart Convert any number of consecutive spaces and tabs to
a single space.

\problem In some implementations of ``regular expressions,'' the
notations \texttt{\bk 1}, \texttt{\bk 2}, and so on can occur
in a search pattern.  For example, consider the search pattern
\texttt{\sol([a-zA-Z]).*\bk1\$}.  Here, \texttt{\bk1} represents
a recurrence of the same substring that matched \texttt{[a-zA-Z]},
the part of the pattern between the first pair of parentheses.
The entire pattern, therefore, will match a line of text that
begins and ends with the same letter.  Using this notation,
write a pattern that matches all strings in the language
$L=\{a^nba^n\,\st\,n\ge0\}$.  (Later in this chapter, we will
see that $L$ is \textit{not} a regular language, so allowing the
use of \texttt{\bk1} in a ``regular expression'' means that it's
not really a regular expression at all!  This notation can add
a real increase in expressive power to the patterns that contain it.)

\end{exercises}


%\newcommand{\gap}{\vspace{2.5ex}}
\newcommand{\dstar}{\delta^*}
\newcommand{\pstar}{\partial^*}
\newcommand{\winss}{w \in \Sigma^*}

\section{Finite-State Automata}\label{S-fsa-1}

We have seen how regular expressions can be used to generate languages
mechanically. How might languages be recognized mechanically? 
The question is of interest because if we can mechanically recognize languages
like $L= \{$all legal C++ programs that will not go into infinite loops on any
input$\}$, then it would be possible to write \"uber-compilers that can do
semantic error-checking like testing for infinite loops, in addition to the
syntactic error-checking they currently do.

What formalism might we use to model what it means to recognize a language
``mechanically''?  We look for inspiration to a language-recognizer with which
we are all familiar, and which we've already in fact mentioned: a compiler.
Consider how a C++ compiler might handle recognizing a legal {\em if}
statement.  Having seen the word {\em if}, the compiler will be in a {\em state}
or {\em phase of its execution} where it expects to see a `('; in this state,
any other character will put the compiler in a ``failure" state.  If the compiler
does in fact see a `(' next, it will then be in an ``expecting a boolean
condition" state; if it sees a sequence of symbols that make up a legal boolean
condition, it will then be in an ``expecting a `)'" state; and then ``expecting
a `$\{$' or a legal statement"; and so on.  Thus one can think of the compiler as
being in a series of states; on seeing a new input symbol, it moves on to a new
state; and this sequence of transitions eventually leads to either a ``failure"
state (if the {\em if} statement is not syntactically correct) or a ``success"
state (if the {\em if} statement is legal).  We isolate these three
concepts---states, input-inspired transitions from state to state, and
``accepting" vs ``non-accepting" states---as the key features of a mechanical
language-recognizer, and capture them in a model called a {\em finite-state
automaton}.  (Whether this is a successful distillation of the essence of
mechanical language recognition remains to be seen; the question will be taken
up later in this chapter.)

A \nw{finite-state automaton (FSA)}, then, is a machine which takes, as input, 
a finite
string of symbols from some alphabet $\Sigma$.
There is a finite set of \nw{states} in which the machine can find itself.  The
state it is in before consuming any input is called the \nw{start state}.
Some of the states are \nw{accepting}
or \nw{final}.  If the machine ends in such a state after completely consuming
an input string, the string is said to be \nw{accepted} by the machine.
The actual functioning of the machine is described by something called a 
\nw{transition function}, which specifies 
what happens if the machine is in a particular state and looking at a
particular input symbol.  (``What happens" means ``in which state does the
machine end up".)    

\begin{example} Below is a table that describes the transition function of a 
finite-state automaton with states $p$, $q$, and $r$, on inputs $0$ and $1$.


\begin{center}
\begin{tabular}{|c||c|c|c|}
        \hline
        $\ $& $p$& $q$& $r$\\
        \hline
        \strut 0& $p$& $q$& $r$\\
        1& $q$& $r$& $r$\\
        \hline
     \end{tabular}
\end{center}

The table indicates, for example, 
that if the FSA were in state $p$ and consumed a $1$, it would
move to state $q$.
\end{example}
    
FSAs actually come in two flavours depending on what
properties you require of the transition function.  We will look first at a class
of FSAs called deterministic finite-state automata (DFAs).  In these
machines, the current state of the machine and the current input symbol together
determine exactly which state the machine ends up in: for every $<$current state,
current input symbol$>$ pair, there is exactly one possible next state for the
machine.

\smallskip

\begin{definition}
Formally,
a \nw{deterministic finite-state automaton} $M$ is specified by 5 components:
$M=(Q, \Sigma, q_0, \delta, F)$ where
\begin{itemize} 
\item $Q$ is a finite set of states; 
\item $\Sigma$ is an alphabet called the {\em input alphabet}; 
\item $q_0 \in Q$ is a state which is designated as the {\em start state}; 
\item $F$ is a subset of $Q$; the states in $F$ are states designated as 
{\em final} or {\em accepting}  states; 
\item $\delta$ is a transition function that takes 
$<$state, input symbol$>$ pairs and maps each one to a state: $\delta : Q \times
\Sigma \rightarrow Q$.  To say
$\delta(q,a) = q'$ means that
if the machine is in state $q$ and the input symbol $a$ is consumed, then the
machine will move into state $q'$.  The function $\delta$ must be a total
function, meaning that $\delta(q,a)$ must be defined for every state $q$ and
every input symbol $a$.  (Recall also that, according to the definition of a
function, there can be only one output for any particular input.  This means
that for any given $q$ and $a$, $\delta(q,a)$ can have only one value.  This is
what makes the finite-state automaton deterministic: given the current state and
input symbol, there is only one possible move the machine can make.)
\end{itemize}
\end{definition}

\begin{example}
The transition function described by the table in the preceding example is that of
a DFA.  
If we take $p$ to be the start state and $r$ to be a final state, then the
formal description  of the resulting machine 
is $M= (\{p,q,r\}, \{0,1\}, p, \delta, \{r\})$, where $\delta$
is given by

\medskip

$\hspace{0.5in}\delta(p,0)=p$ \hspace{1.5in} $\delta(p,1)=q$

$\hspace{0.5in}\delta(q,0)=q$ \hspace{1.5in} $\delta(q,1)=r$

$\hspace{0.5in}\delta(r,0)=r$ \hspace{1.5in} $\delta(r,1)=r$
\end{example}
\smallskip

The transition function $\delta$ describes only individual steps of the machine
as individual input symbols are consumed.  However, we will often want to refer
to
``the
state the automaton will be in if it starts in state $q$ and consumes input
string $w$", where $w$ is a string of input symbols rather than a single symbol.
Following the usual practice of using $^*$ to designate
``0 or more", we define \nw{$\dstar(q,w)$} as a convenient shorthand for 
``the state that the automaton will be in
if it starts in state $q$ and consumes the input string $w$". For any string,
it is easy to see, based on $\delta$, what steps the machine will
make as those symbols are consumed, and what $\dstar(q,w)$ will be for any $q$
and $w$. Note that if no input is consumed, a DFA makes no move, and so
$\dstar(q, \varepsilon) = q$ for any state $q$.\footnote{$\delta^*$ can be defined
formally by saying that $\delta^*(q,\varepsilon)=q$ for every state $q$,
and $\delta^*(q,ax)=\delta^*(\delta(q,a),x)$ for any state $q$, $a\in\Sigma$
and $x\in\Sigma^*$.  Note that this is a recursive definition.}

\smallskip

\begin{example}
Let $M$ be the automaton in the preceding example.  Then, for example:

$\dstar(p, 001)=q$, since $\delta(p,0)=p$, $\delta(p,0)=p$, and $\delta(p,1)=q$; 

$\dstar(p, 01000)= q$;

$\dstar(p, 1111) = r$;

$\dstar(q, 0010) = r$.
\end{example}

\smallskip

We have divided the states of a DFA into accepting and non-accepting states, with
the idea that some strings will be recognized as ``legal" by the automaton, and
some not.  Formally:

\begin{definition}
Let $M=(Q, \Sigma, q_0, \delta, F)$.  A string $w \in \Sigma^*$ is \nw{accepted}
by $M$ iff $\dstar(q_0, w) \in F$. \ \ (Don't get confused by the notation.  
Remember, it's just a shorter and neater way of saying
``$w \in \Sigma^*$ is accepted by $M$ if and only if the state that $M$ will end
up in
if it starts in $q_0$ and consumes $w$ is one of the states in $F$.")

The \nw{language accepted by $M$}, denoted $L(M)$, is the set of all strings 
$w \in \Sigma^*$ that are accepted by $M$: 
$L(M) = \{ w \in\Sigma^* \ | \ \delta^*(q_0, w) \in F\}$.

\end{definition}

\smallskip
Note that we sometimes use a slightly different phrasing and say that a language
$L$ is accepted by some machine $M$.  We don't mean by this that $L$ {\em and
maybe some other strings} are accepted by $M$; we mean $L = L(M)$, i.e.\ $L$ is
{\em exactly} the set of strings accepted by $M$.

It may not be easy, looking at a formal specification of a DFA, to determine what
language that automaton accepts.  Fortunately, the mathematical description of
the automaton $M=(Q, \Sigma, q_0, \delta, F)$ can be neatly and helpfully
captured in a picture called a \nw{transition diagram}.  
Consider again the DFA of the two preceding examples.  It
can be represented pictorially as:

\fsafig{1}

\noindent The arrow on the left indicates that $p$ is the start state; double
circles indicate that a state is accepting.  Looking at this picture, it should
be fairly easy to see that the language accepted by the DFA $M$ is 
$L(M) = \{ x \in \{0,1\}^* \ | \ n_1(x) \geq 2\}$.

\begin{example}
Find the language accepted by the DFA shown below (and describe it using a
regular expression!)

\fsafig{2}

The start state of $M$ is accepting, which means $\varep \in L(M)$.  If $M$ is
in state $q_0$, a
sequence of two $a$'s or three $b$'s will move $M$ back to $q_0$ and hence
be accepted.  So $L(M) = L((aa\REOR bbb)^*)$.
\end{example}

The state $q_4$ in the preceding example is often called a {\em garbage} or {\em
trap} state: it is a non-accepting state which, once reached by the machine,
cannot be escaped.  It is fairly common to omit such states from transition
diagrams.  For example, one is likely to see the diagram:

\fsafig{3}

Note that this cannot be a complete DFA, because a DFA is required to have a
transition defined for every state-input pair.  The diagram is ``short for" the
full diagram:

\fsafig{4}


As well as recognizing what language is accepted by a given DFA, we often want to
do the reverse and come up with a DFA that accepts a given language.
Building DFAs for specified languages is an art, not a science.
There is no algorithm that you can apply to produce a DFA from an English-language
description of the set of strings the DFA should accept.  On the other hand, it
is not generally successful, either, to simply write down a half-dozen strings
that are in the language and design a DFA to accept those strings---invariably
there are strings that are in the language that aren't accepted, and other
strings that aren't in the language that are accepted.  So how do you go about
building DFAs that accept all and only the strings they're supposed to accept?
The best advice I can give is to
think about relevant characteristics that determine whether a string is in the
language or not, and to think about what the possible values or ``states" of 
those characteristics
are; then build a machine that has a state corresponding to each possible
combination of values of relevant characteristics, and determine how the
consumption of inputs affects those values.  I'll illustrate what I mean with a
couple of examples.
\begin{example}
Find a DFA with input alphabet $\Sigma = \ab$ that accepts the language
$L= \{\winss \ | \ n_a(w) \mbox{ and } n_b(w) \mbox{ are both even } \}$.

The characteristics that determine whether or not a string $w$ is in $L$ are the
parity of $n_a(w)$ and $n_b(w)$.  There are four possible combinations of
``values" for these characteristics: both numbers could be even, both could be
odd, the first could be odd and the second even, or the first could be even and
the second odd.  So we build a machine with four states $q_1, q_2, q_3, q_4$
corresponding to the four cases.  We want to set up $\delta$ so that the machine
will be in state $q_1$ exactly when it has consumed a string with an even number
of $a$'s and an even number of $b$'s, in state $q_2$ exactly when it has 
consumed a string with an
odd number of $a$'s and an odd number of $b$'s, and so on.  

To do this, we first make 
the state $q_1$ into our start state,
because the DFA will be in the start state after consuming the empty string
$\varep$, and $\varep$ has an even number (zero) of both $a$'s and $b$'s.  Now we
add transitions by reasoning about how the parity of $a$'s and $b$'s is changed
by additional input.  For instance, if the machine is in $q_1$ (meaning an even
number of $a$'s and an even number of $b$'s have been seen) and a further $a$ is
consumed, then we want the machine to move to state $q_3$, since the machine has
now consumed an odd number of $a$'s and still an even number of $b$'s.  So we add
the transition $\delta(q_1, a) = q_3$ to the machine.  Similarly, if the machine
is in $q_2$ (meaning an odd
number of $a$'s and an odd number of $b$'s have been seen) and a further $b$ is
consumed, then we want the machine to move to state $q_3$ again, since the 
machine has
still consumed an odd number of $a$'s, and now an even number of $b$'s.
So we add
the transition $\delta(q_2, b) = q_3$ to the machine.  Similar reasoning produces
a total of eight transitions, one for each state-input pair.  Finally, we have to
decide which states should be final states.  The only state that corresponds to
the desired criteria for the language $L$ is $q_1$, so we make $q_1$ a final
state.  The complete machine is shown below.


\fsafig{5}


\end{example}

\begin{example}
Find a DFA with input alphabet $\Sigma = \ab$ that accepts the language
$L$ = $\{\winss \ | \  n_a(w)  \mbox{ is divisible by 3 } \}$.

The relevant characteristic here is of course whether or not the number of $a$'s
in a string is divisible by 3, perhaps suggesting a two-state machine.  But in
fact, there is more than one way for a number to not be divisible by 3: dividing
the number by 3 could produce a remainder of either 1 or 2 (a remainder of 0
corresponds to the number in fact being divisible by 3).  So we build a machine
with three states $q_0$, $q_1$, $q_2$, and add transitions so that the machine
will be in state $q_0$ exactly when the number of $a$'s it has consumed is evenly
divisible by 3, in state $q_1$ exactly when the number of $a$'s it has consumed
is equivalent to $ 1 \bmod{3}$, and similarly for $q_2$.  State $q_0$ will be the
start state, as $\varep$ has 0 $a$'s and 0 is divisible by 3.  The completed
machine is shown below.  Notice that because the consumption of a $b$ does not
affect the only relevant characteristic, $b$'s do not cause changes of 
state.

\fsafig{6}

\end{example}


\begin{example}
Find a DFA with input alphabet $\Sigma = \ab$ that accepts the language
$L$ = $\{\winss \ | w \mbox{ contains three consecutive a's } \}$.

Again, it is not quite so simple as making a two-state machine where the states
correspond to ``have seen $aaa$" and ``have not seen $aaa$".
Think dynamically: as you move through the
input string, how do you arrive at the goal of having seen three consecutive
$a$'s?  You might have seen two consecutive $a$'s and still need a third, or
you might just have seen one $a$ and be looking for two more to come
immediately, or you might just have seen a $b$ and be right back at the
beginning as far as seeing 3 consecutive $a$'s goes.  So once again there will be
three states, with the ``last symbol was not an $a$'' state being the start
state.  The complete automaton is shown below.

\medskip
\fsafig{7}
\end{example}

\begin{exercises}
\problem Give DFAs that accept the following languages over $\Sigma =\ab$.
\ppart $L_1= \{ x \ | \ x \mbox{ contains the substring } aba\}$
\ppart $L_2= L(a^*b^*)$
\ppart $L_3= \{ x \ | \ n_a(x)+n_b(x) \mbox{ is even }\}$
\ppart $L_4= \{ x \ | \ n_a(x) \mbox{ is a multiple of 5 }\}$
\ppart $L_5= \{ x \ | \ x \mbox{ does not contain the substring } abb\}$
\ppart $L_6= \{ x \ | \ x \mbox{ has no $a$'s in the even positions} \}$
\ppart $L_7 = L(aa^* \REOR  aba^*b^*)$
\problem What languages do the following DFAs accept?

\fsafig{1ex}

\fsafig{2ex}


\problem Let $\Sigma=\{0,1\}$. Give a DFA that accepts the language 
$$ L = \{ x \in \Sigma^* \ | \ x \mbox{ is the binary representation of an integer
divisible by 3}\}.$$ 

\end{exercises}



\section{Nondeterministic Finite-State Automata}\label{S-fsa-2}

As mentioned briefly above, there is an alternative school of though as to what
properties should be required of a finite-state automaton's transition function.
Recall our motivating example of a C++ compiler and a legal {\em if} statement. 
In our description, we had the compiler in an ``expecting a `)'$\,$" state; on
seeing a `)', the compiler moved into an ``expecting a `$\{$' or a legal
statement" state.  An alternative way to view this would be to say that the
compiler, on seeing a `)', could move into one of two different states: it could
move to an ``expecting a `$\{$'" state {\bf or} move to an ``expecting a legal
statement" state. Thus, from a single state, on input `)', the compiler has
multiple moves.  This alternative interpretation is not
allowed by the DFA model.  A second point on which one 
might question the DFA model is the fact that input must be consumed for the
machine to change state.
Think of the syntax for C++ function declarations.  The return type of a
function need not be specified (the default is taken to be {\em int}).  The
start state of the compiler when parsing a function declaration might be 
``expecting a return type"; then with no
input consumed, the compiler can move to the state ``expecting a legal function 
name".  To model this, it might seem reasonable to allow transitions that do 
not require
the consumption of input (such transitions are called \nw{$\varep$-transitions}).  
Again, this is not supported by the DFA abstraction.
There is, therefore, a second class of finite-state automata that people
study, the class of nondeterministic finite-state automata.  

\smallskip

A \nw{nondeterministic finite-state automaton (NFA)} is the same as a 
deterministic
finite-state automaton except that the transition function is no longer a
function that maps a state-input pair to a state; rather, it maps a state-input
pair or a state-$\varep$ pair to a {\bf set} of states.  No longer do we have 
$\delta(q,a) = q'$, meaning that the machine
must change to state $q'$ if it is in state $q$ and consumes an $a$.  Rather,
we have $\partial(q,a) = \{q_1, q_2, \ldots, q_n\}$, meaning that if the
machine is in state $q$ and consumes an $a$, it might move directly to any one
of the states $q_1, \ldots, q_n$.  Note that the set of next states
$\partial(q,a)$ is defined for every state $q$ and every input symbol $a$,
but for some $q$'s and $a$'s it could be empty, or contain just one state (there
don't {\bf have} to be multiple next states).  The function $\partial$ must
also specify whether it is possible for the machine to make any moves 
without input being consumed, i.e.\ $\partial(q, \varepsilon)$ must be
specified for every state $q$.  Again, it is quite possible that 
$\partial(q, \varepsilon)$ may be empty for some states $q$: there need not be
$\varep$-transitions out of $q$.

\smallskip

\begin{definition}
Formally,
a nondeterministic finite-state automaton $M$ is specified by 5 components:
$M=(Q, \Sigma, q_0, \partial, F)$ where
\begin{itemize} 
\item  $Q$, $\Sigma$, $q_0 $ and $F$ are as in the definition of DFAs;
\item $\partial$ is a transition function that takes 
$<$state, input symbol$>$ pairs and maps each one to a set of states.  To say
$\partial(q,a) = \{q_1, q_2, \ldots , q_n\}$ means that
if the machine is in state $q$ and the input symbol $a$ is consumed, then the
machine may move directly into any one of states $q_1, q_2, \ldots , q_n$.  
The function $\partial$ must also be defined for every $<$state,$\varep$$>$ pair.
To say
$\partial(q,\varep) = \{q_1, q_2, \ldots , q_n\}$ means that there are direct
$\varep$-transitions from state $q$ to each of  $q_1, q_2, \ldots , q_n$.


The formal description of the function $\partial$ is $\partial : Q \times
(\Sigma \cup \{\varep\}) \rightarrow \POW(Q)$.
\end{itemize}
\end{definition}


The function $\partial$ describes how the machine functions on zero or one 
input symbol. 
As with DFAs, we will often want to refer to the behavior of the machine on a
string of inputs, and so we use the notation $\pstar(q,w)$ as shorthand
for ``the set of states in which
the machine might be if it starts in state $q$ and consumes input string $w$".  
As with DFAs, $\pstar(q,w)$ is
determined by the specification of $\partial$.  Note that for every state $q$,
$\pstar(q, \varep)$ contains at least $q$, and may contain additional states if
there are (sequences of) $\varep$-transitions out of $q$. 

We do have to think a bit carefully about what it means for an NFA to accept a
string $w$.  Suppose $\pstar(q_0,w)$ contains both accepting and non-accepting
states, i.e.\ the machine could end in an accepting state after consuming $w$,
but it might also end in a non-accepting state.  Should we consider the machine
to accept $w$, or should we require every state in $\pstar(q_0,w)$ to be
accepting before we admit $w$ to the ranks of the accepted?  Think of the C++
compiler again: provided that an {\em if} statement fits one of the legal
syntax specifications, the compiler will accept it.  So we take as the
definition of acceptance by an NFA: A string $w$ is accepted by an NFA provided
that at least one of the states in $\pstar(q_0, w)$ is an accepting state. 
That is, if there is some sequence of steps of the machine that consumes $w$
and leaves the machine in an accepting state, then the machine accepts $w$.
Formally:

\smallskip

\begin{definition}
Let $M= (Q, \Sigma, q_0, \partial, F)$ be a nondeterministic finite-state
automaton.  The string $\winss$ is \nw{accepted} 
by $M$ iff $\pstar(q_0,w)$ contains at least one state $q_F \in F$.

The \nw{language accepted by $M$}, denoted $L(M)$, is the set of all strings 
$\winss$ that are
accepted by $M$: $L(M) = \{ w \in \Sigma^* \ | \ \pstar(q_0, w) \cap F \not= 
\emptyset\}$.
\end{definition}

\smallskip


\begin{example}\label{asome} The NFA shown below accepts all strings of $a$'s 
and $b$'s in which the second-to-last symbol is $a$.

\fsafig{8}

\end{example} 

It should be fairly clear that every language that is accepted by a DFA is also
accepted by an NFA.  Pictorially, a DFA looks exactly like an NFA (an NFA that
doesn't happen to have any $\varep$-transitions or multiple same-label
transitions from any state), though there is slightly more going on behind
the scenes.  Formally, given the DFA $M=(Q, \Sigma, q_0, \delta, F)$, you can
build an NFA $M'=(Q, \Sigma, q_0, \partial, F)$ where 4 of the 5 components
are the same and where every transition $\delta(q,a) = q'$ has been replaced by
$\partial(q,a) = \{q'\}$. 

But is the reverse true?  Can any NFA-recognized language be recognized by a DFA?
Look, for example, at the language in Example~\ref{asome}.  Can you come up with
a DFA that accepts this language?  Try it.  It's pretty difficult to do.  But
does that mean that there really is {\bf no} DFA that accepts the language, or
only that we haven't been clever enough to find one?

It turns out that the limitation is in fact in our cleverness, and not in the
power of DFAs.

\begin{theorem}
Every language that is accepted by an NFA is accepted by a DFA.
\end{theorem}
\begin{proof} Suppose we are given an NFA $N = (P, \Sigma, p_0, \partial, F_p)$, and we want to
build a DFA $D=(Q, \Sigma, q_0, \delta, F_q)$ that accepts the same language.
The idea is to make the states in $D$ correspond to {\em subsets}
of $N$'s states, and
then to set up $D$'s transition function $\delta$ so that for any string $w$, 
$\dstar(q_0, w)$ corresponds to $\pstar(p_0,w)$; i.e.\ the {\bf single} state that
$w$ gets you to in $D$ corresponds to the {\bf set} of states that $w$ could get
you to in $N$.  
If any of those states is accepting in $N$, $w$ would
be accepted by $N$, and so the corresponding state in $D$ would be made accepting
as well.

So how do we make this work?  The first thing to do is to deal with a start state
$q_0$ for $D$.  If we're going to make this state correspond to a subset of $N$'s
states, what subset should it be?  Well, remember (1) that in any DFA,
$\dstar(q_0, \varep) = q_0$; and (2) we want to make $\dstar(q_0, w)$ correspond
to $\pstar(p_0,w)$ for every $w$.  Putting these two limitations together tells
us that we should make $q_0$ correspond to $\pstar(p_0, \varep)$.  So $q_0$
corresponds to the subset of all of $N$'s states that can be reached with no
input.

Now we progressively set up $D$'s transition function $\delta$ by repeatedly
doing the following:

-- find a state $q$ that has been added to $D$ but whose out-transitions have not
yet been added.  (Note that $q_0$ initially fits this description.)  Remember
that the state $q$ corresponds to some subset $\{p_1, \ldots , p_n\}$ of $N$'s
states.

-- for each input symbol $a$, look at all $N$'s states that can be reached from
any one of $p_1, \ldots , p_n$ by consuming $a$ (perhaps making some
$\varep$-transitions as well).  That is, look at $\pstar(p_1,a) \cup \ldots \cup
\pstar(p_n,a)$.  If there is not already a DFA state $q'$ that corresponds to
this subset of $N$'s states, then add one, and add the transition 
$\delta(q, a)= q'$ to $D$'s transitions.

The above process must halt eventually, as there are only a finite
number of states $n$ in the NFA, and therefore there can be at most $2^n$ states in the
DFA, as that is the number of subsets of the NFA's states.  The final states of
the new DFA are those where at least one of the associated NFA states is an
accepting state of the NFA.  

Can we now argue that $L(D) = L(N)$?  We can, if we can argue that
$\dstar(q_0,w)$ corresponds to $\pstar(p_0,w)$ for all $\winss$: if this
latter property holds, then $w \in L(D)$ iff $\dstar(q_0,w)$ is accepting, which
we made be so iff $\pstar(p_0,w)$ contains an accepting state of $N$, which
happens iff $N$ accepts $w$ i.e.\ iff $w \in L(N)$.

So can we argue that $\dstar(q_0,w)$ does in fact correspond to $\pstar(p_0,w)$
for all $w$?  We can, using induction on the length of $w$.

First, a preliminary observation.  Suppose $w=xa$, i.e.\ $w$ is the string $x$
followed by the single symbol $a$.  How are $\pstar(p_0,x)$ and $\pstar(p_0,w)$
related?  Well, recall that $\pstar(p_0,x)$ is the set of all states that $N$ can
reach when it starts in $p_0$ and  consumes $x$: 
$\pstar(p_0,x) = \{p_1, \ldots, p_n\}$ for some states
$p_1, \ldots, p_n$.  Now, $w$ is just $x$ with an additional $a$, so where might
$N$ end up if it starts in $p_0$ and  consumes $w$?  We know that $x$ gets $N$ to
$p_1$ or $\ldots$ or $p_n$, so $xa$ gets $N$ to any state that can be reached
from $p_1$ with an $a$ (and maybe some $\varep$-transitions), and to any state
that can be reached from $p_2$ with an $a$ (and maybe some $\varep$-transitions),
etc.  Thus, our relationship between $\pstar(p_0,x)$ and $\pstar(p_0,w)$ is that
if $\pstar(p_0,x) = \{p_1, \ldots, p_n\}$, then $\pstar(p_0,w) = \pstar(p_1,a)
\cup \ldots \cup \pstar(p_n,a)$.  With this observation in hand, let's proceed to
our proof by induction.

We want to prove that $\dstar(q_0,w)$ corresponds to $\pstar(p_0,w)$ for all
$\winss$.  We use induction on the length of $w$.
\begin{enumerate}
\item Base case: Suppose $w$ has length 0.  The only string $w$ with length 0 is
$\varep$, so we want to show that 
$\dstar(q_0,\varep)$ corresponds to $\pstar(p_0,\varep)$.  Well, 
$\dstar(q_0, \varep) = q_0$, since in a DFA, $\dstar(q, \varep) = q$ for any
state~$q$.  We explicitly made $q_0$ correspond to 
$\pstar(p_0,\varep)$, and so the property holds for $w$ with length 0.
\item Inductive case: Assume that the desired property holds for some number $n$,
i.e.\ that  $\dstar(q_0,x)$ corresponds to $\pstar(p_0,x)$ for all $x$ with
length $n$.  Look at an arbitrary string $w$ with length $n+1$.  
We want to show that $\dstar(q_0,w)$ corresponds to $\pstar(p_0,w)$.
Well, the string $w$
must look like $xa$ for some string $x$ (whose length is $n$) and some symbol
$a$.  By our inductive hypothesis, we know
$\dstar(q_0,x)$ corresponds to $\pstar(p_0,x)$.  We know $\pstar(p_0,x)$ is a
set of $N$'s states, say 
$\pstar(p_0,x) = \{p_1, \ldots, p_n\}$.

At this point, our subsequent reasoning might be a bit clearer if we give
explicit names
to $\dstar(q_0,w)$ (the state $D$
reaches on input $w$) and $\dstar(q_0,x)$ (the state $D$
reaches on input $x$).  Call $\dstar(q_0, w)$  \ $q_w$, and call
$\dstar(q_0,x)$ \ $q_x$.  We know, because $w=xa$, there must be an 
$a$-transition from $q_x$ to $q_w$.  Look at how we added transitions to
$\delta$: the fact that there is an $a$-transition from $q_x$ to $q_w$ means that
$q_w$ corresponds to the set $\pstar(p_1,a)
\cup \ldots \cup \pstar(p_n,a)$ of $N$'s states.  By our preliminary observation,
$\pstar(p_1,a)
\cup \ldots \cup \pstar(p_n,a)$ is just $\pstar(p_0,w)$.  So $q_w$ (or
$\dstar(q_0,w)$) corresponds to $\pstar(p_0,w)$, which is what we wanted to
prove.  Since $w$ was an arbitrary string of length $n+1$, we have shown that 
the property holds for $n+1$.
\end{enumerate}

Altogether, we have shown by induction that $\dstar(q_0,w)$ corresponds to
$\pstar(p_0,w)$ for all
$\winss$.  As indicated at the very beginning of this proof, that is enough to
prove that $L(D)= L(N)$.  So for any NFA $N$, we can find a DFA $D$ that accepts
the same language.
\end{proof}

\bigskip

\begin{example}\label{nfatodfaex}
Consider the NFA shown below.

\fsafig{9}

We start by looking at $\pstar(p_0, \varep)$, and then add transitions and
states as described above.
\begin{itemize}
\item
$\pstar(p_0, \varep) = \{p_0\}$ so $q_0 = \{p_0\}$.

\item
$\delta(q_0,a)$ will be $\pstar(p_0,a)$, which is $\{p_0\}$, 
so $\delta(q_0,a) = q_0$.

\item
$\delta(q_0,b)$ will be $\pstar(p_0,b)$, which is $\{p_0, p_1\}$,
so we need to add a new state
$q_1 = \{p_0, p_1\}$ to the DFA; and add $\delta(q_0,b) = q_1$ to the DFA's
transition function.

\item
$\delta(q_1,a)$ will be $\pstar(p_0,a)$ unioned with $\pstar(p_1,a)$ since
$q_1 = \{p_0, p_1\}$.  Since $\pstar(p_0,a) \cup \pstar(p_1,a) = \{p_0\} \cup
\{p_2\} = \{p_0,p_2\}$, we need to add a new state $q_2 = \{p_0, p_2\}$ to the
DFA, and a transition $\delta(q_1,a) = q_2$.

\item 
$\delta(q_1,b)$ will be $\pstar(p_0,b)$ unioned with $\pstar(p_1,b)$, which
gives $\{p_0, p_1\} \cup \{p_2\}$, which again gives us a new state $q_3$ to add to
the DFA, together with the transition $\delta(q_1,b) = q_3$.
\end{itemize}

At this point, our partially-constructed DFA looks as shown below:

\fsafig{10}

The construction continues as long as there are new states being added, and new
transitions from those states that have to be computed.
The final DFA is shown below.

\fsafig{11}

\end{example}


\begin{exercises}
\problem What language does the NFA in Example~\ref{nfatodfaex} accept?
\problem Give a DFA that accepts the language accepted by the 
following NFA.

\fsafig{3ex}

\problem Give a DFA that accepts the language accepted by the following NFA.
(Be sure to note that, for example, it is possible to reach both $q_1$ and
$q_3$ from $q_0$ on consumption of an $a$, because of the 
$\varep$-transition.)

\fsafig{4ex}


\end{exercises}



\section{Finite-State Automata and Regular Languages}\label{S-fsa-3}

We know now that our two models for mechanical language recognition actually
recognize the same class of languages.  The question still remains: do they
recognize the same class of languages as the class generated mechanically by regular
expressions?  The answer turns out to be ``yes".  There are two parts to proving
this: first that every language generated can be recognized, and second that
every language recognized can be generated.

\begin{theorem}\label{retonfa}
Every language generated by a regular expression can be recognized by an NFA.
\end{theorem}

\begin{proof} The proof of this theorem is a nice example of a proof by induction on
the structure of regular expressions.  The definition of regular expression is
inductive: $\Phi$, $\varep$, and $a$ are the simplest regular expressions,
and then more complicated regular expressions can be built from these.  We will
show that there are NFAs that accept the languages generated by the simplest
regular expressions, and then show how those machines can be put together to
form machines that accept languages generated by more complicated regular
expressions.

Consider the regular expression $\Phi$.  $L(\Phi) = \{\}$.  Here is a machine
that accepts $\{\}$: 

\fsafig{12}

Consider the regular expression $\varep$.  $L(\varep) = \{\varepsilon\}$.  
Here is a machine that accepts $\{\varepsilon\}$:

\fsafig{13}

Consider the regular expression $a$.  $L(a) = \{a\}$.  Here is a
machine that accepts $\{a\}$:

\fsafig{14}

Now suppose that you have NFAs that accept the languages generated by the
regular expressions $r_1$ and $r_2$.  Building a machine that accepts $L(r_1 \REOR 
r_2)$ is fairly straightforward: take an NFA $M_1$ that accepts $L(r_1)$ and an
NFA $M_2$ that accepts $L(r_2)$.  Introduce a new state $q_{new}$, connect
it to the start states of $M_1$ and $M_2$ via $\varepsilon$-transitions, and
designate it as the start state of the new machine.  No other transitions are
added.  The final states of $M_1$ together with the final states of $M_2$ are
designated as the final states of the new machine.  It should be fairly clear
that this new machine accepts exactly those strings accepted by $M_1$ together
with those strings accepted by $M_2$: any string $w$ that was accepted by $M_1$
will be accepted by the new NFA by starting with an $\varep$-transition to the
old start state of $M_1$ and then following the accepting path through $M_1$;
similarly, any string accepted by $M_2$ will be accepted by the new machine;
these are the only strings that will be accepted by the new machine, as on any
input $w$ all the new machine can do is make an $\varep$-move to $M_1$'s (or
$M_2$'s) start state, and from there $w$ will only be accepted by the new
machine if it is accepted by $M_1$ (or $M_2$).  Thus, the new machine accepts
$L(M_1) \cup L(M_2)$, which is $L(r_1) \cup L(r_2)$, which is exactly the
definition of $L(r_1 \REOR  r_2)$.

\fsafig{15}

(A pause before we continue: note that for the simplest regular expressions,
the machines that we created to accept the languages generated by the regular
expressions were in fact DFAs.  In our last case above, however, we needed
$\varep$-transitions to build the new machine, and so if we were trying to
prove that every regular language could be accepted by a DFA, our proof would
be in trouble.  THIS DOES NOT MEAN that the statement ``every regular language
can be accepted by a DFA" is false, just that we can't prove it using this kind
of argument, and would have to find an alternative proof.)

Suppose you have machines $M_1$ and $M_2$ that accept $L(r_1)$ and $L(r_2)$
respectively.  To build a machine that accepts $L(r_1)L(r_2)$ proceed as
follows.  Make the start state $q_{01}$ of $M_1$ be the start state of the new
machine.  Make the final states of $M_2$ be the final states of the new machine.
Add $\varep$-transitions from the final states of $M_1$ to the start state
$q_{02}$ of
$M_2$.

\fsafig{16}

It should be fairly clear that this new machine accepts exactly those strings of
the form $xy$ where $x\in L(r_1)$ and $y \in L(r_2)$: first of all, any string
of this form will be accepted because $x\in L(r_1)$ implies there is a path that
consumes $x$ from
$q_{01}$ to a final state of $M_1$; a $\varep$-transition moves to $q_{02}$; 
then $y \in L(r_2)$ implies there is a path that consumes $y$ from $q_{02}$ to a
final state of $M_2$; and the final states of $M_2$ are the final states of the
new machine, so $xy$ will be accepted.  Conversely, suppose $z$ is accepted by
the new machine.  Since the only final states of the new machine are in the old
$M_2$, and the only way to get into $M_2$ is to take a $\varep$-transition from
a final state of $M_1$, this means that $z=xy$ where $x$ takes the machine from
its start state to a final state of $M_1$, a $\varep$-transition occurs, and
then $y$ takes the machine from $q_{02}$ to a final state of $M_2$.  Clearly,
$x\in L(r_1)$ and $y \in L(r_2)$. 

We leave the construction of an NFA that accepts $L(r^*)$ from an NFA that 
accepts $L(r)$ as an exercise.

\end{proof}

\smallskip

\begin{theorem}\label{T-DFAeqReg}
Every language that is accepted by a DFA or an NFA is generated by a regular 
expression.
\end{theorem}

Proving this result is actually fairly involved and not very illuminating. 
Before presenting a proof, we will give an illustrative example of how one
might actually go about extracting a regular expression from an NFA or a DFA.
You can go on to read the proof if you are interested.

\begin{example}
Consider
the DFA shown below:

\fsafig{17}

Note that there is a loop from state $q_2$ back to state $q_2$: any number of
$a$'s will keep the machine in state $q_2$, and so we label the transition with
the regular expression $a^*$.  We do the same thing to the transition labeled
$b$ from $q_0$.  (Note that the result is no longer a DFA, but that doesn't
concern us, we're just interested in developing a regular expression.)

\fsafig{18}

Next we note that there is in fact a loop from $q_1$ to $q_1$ via $q_0$.  A
regular expression that matches the strings that would move around the loop is
$ab^*a$.  So we add a transition labeled $ab^*a$ from $q_1$ to
$q_1$, and remove the now-irrelevant $a$-transition from $q_1$ to $q_0$.  (It is
irrelevant because it is not part of any other loop from $q_1$ to 
$q_1$.)

\fsafig{19}
  
Next we note that there is also a loop from $q_1$ to $q_1$ via $q_2$.  A
regular expression that matches the strings that would move around the loop is
$ba^*b$.  Since the transitions in the loop are the only transitions to or from
$q_2$, we simply remove $q_2$ and replace it with a transition from $q_1$ to
$q_1$.

\fsafig{20}

It is now clear from the diagram that strings of the form $b^*a$ get you to
state $q_1$, and any number of repetitions of strings that match $ab^*a$ or
$ba^*b$ will keep you there.  So the machine accepts $L(b^*a(ab^*a\REOR ba^*b)^*)$. 
\end{example}

%It is a fact that every DFA or NFA is equivalent to an NFA whose start state is
%not accepting, and whose final states number exactly one.  Any such machine can
%be massaged, pictorially speaking, into a "machine" that has exactly two
%states---a non-accepting start state and an accepting second state---and whose
%transition arcs are labeled by regular expressions.  The idea is perhaps best
%illustrated by means of an example:   ***** check these claims



\begin{proof}[Proof of Theorem~\ref{T-DFAeqReg}]
We prove that the language accepted by a DFA is regular.  The proof for NFAs
follows from the equivalence between DFAs and NFAs.

Suppose that $M$ is a DFA, where $M=(Q,\Sigma,q_0,\delta,F)$.  Let $n$ be the
number of states in $M$, and write $Q=\{q_0,q_1,\dots,q_{n-1}\}$.  We want
to consider computations in which $M$ starts in some state $q_i$, reads a string
$w$, and ends in state $q_k$.  In such a computation, $M$ might go through a
series of intermediates states between $q_i$ and $q_k$:
$$q_i\longrightarrow p_1\longrightarrow p_2 \cdots\longrightarrow p_r\longrightarrow q_k$$
We are interested in computations in which all of the intermediate states---$p_1,p_2,\dots,p_r$---are
in the set $\{q_0,q_1,\dots,q_{j-1}\}$, for some number~$j$.
We define $R_{i,j,k}$ to be the set of all strings $w$ in $\Sigma^*$ that are consumed
by such a computation.  That is, $w\in R_{i,j,k}$ if and only if when $M$ starts in state
$q_i$ and reads $w$, it ends in state $q_k$, and all the intermediate states between
$q_i$ and $q_k$ are in the set $\{q_0,q_1,\dots,q_{j-1}\}$.
$R_{i,j,k}$ is a language over $\Sigma$.  We show that $R_{i,j,k}$ for
$0\le i < n$, $0\le j \le n$, $0\le k < n$.

Consider the language $R_{i,0,k}$.  For $w\in R_{i,0,k}$, the set of allowable intermediate
states is empty.  Since there can be no intermediate states,
it follows that there can be at most one step in the computation that
starts in state $q_i$, reads $w$, and ends in state $q_k$.  So, $|w|$ can be at most one.
This means that $R_{i,0,k}$ is finite, and hence is regular.  (In fact,
$R_{i,0,k}=\{a\in\Sigma\st \delta(q_i,a)=q_k\}$, for $i\ne k$, and
$R_{i,0,i}=\{\varep\}\cup\{a\in\Sigma\st \delta(q_i,a)=q_i\}$.  Note that in many
cases, $R_{i,0,k}$ will be the empty set.)

We now proceed by induction on $j$ to show that $R_{i,j,k}$ is regular for all $i$ and $k$.
We have proved the base case, $j=0$.  Suppose that $0\le j< n$ we already know that $R_{i,j,k}$
is regular for all $i$ and all $k$.  We need to show that $R_{i,j+1,k}$ is regular for all $i$ and $k$.
In fact, 
$$R_{i,j+1,k}=R_{i,j,k}\cup \left( R_{i,j,j}R_{j,j,j}^*R_{j,j,k}\right)$$
which is regular because $R_{i,j,k}$ is regular for all $i$ and $k$, and because the union, concatenation,
and Kleene star of regular languages are regular.

To see that the above equation holds, consider a string $w\in\Sigma^*$.
Now, $w\in R_{i,j+1,k}$ if and only if when $M$ starts in state $q_i$ and reads $w$,
it ends in state $q_k$, with all intermediate states in the computation in the set
$\{q_0,q_1,\dots,q_j\}$.  Consider such a computation.  There are two
cases: Either $q_j$ occurs as an intermediate state in the computation, or it does not.
If it does \textbf{not} occur, then all the intermediate states are in the set
$\{q_0,q_1,\dots,q_{j-1}\}$, which means that in fact $w\in R_{i,j,k}$.
If $q_j$ \textbf{does} occur as an intermediate state in the computation, then we can break the
computation into phases, by dividing it at each point where $q_j$ occurs
as an intermediate state.  This breaks $w$ into a concatenation $w=xy_1y_2\cdots y_rz$.
The string $x$ is consumed in the first phase of the computation, during which $M$
goes from state $q_i$ to the first occurrence of $q_j$; since the intermediate states
in this computation are in the set $\{q_0,q_1,\dots,q_{j-1}\}$, $x\in R_{i,j,j}$.
The string $z$ is consumed by the last phase of the computation, in which $M$
goes from the final occurrence of $q_j$ to $q_k$, so that $z\in R_{j,j,k}$.
And each string $y_t$ is consumed in a phase of the computation in which $M$ goes
from one occurrence of $q_j$ to the next occurrence of $q_j$, so that $y_r\in R_{j,j,j}$.
This means that $w=xy_1y_2\cdots y_rz\in R_{i,j,j}R_{j,j,j}^*R_{j,j,k}$.

We now know, in particular, that $R_{0,n,k}$ is a regular language for all $k$.
But $R_{0,n,k}$ consists of all strings $w\in\Sigma^*$ such that when $M$ starts
in state $q_0$ and reads $w$, it ends in state $q_k$ (with \textbf{no} restriction
on the intermediate states in the computation, since every state of $M$ is in
the set $\{q_0,q_1,\dots,q_{n-1}\}$).
To finish the proof that $L(M)$ is regular, it is only necessary to note that
$$L(M)=\bigcup_{q_k\in F} R_{0,n,k}$$
which is regular since it is a union of regular languages.
This equation is true since
a string $w$ is in $L(M)$ if and only if when $M$ starts in state $q_0$ and reads $w$,
in ends in some accepting state $q_k\in F$. This is the same as saying
$w\in R_{0,n,k}$ for some $k$ with $q_k\in F$.
\end{proof}


\bigskip

We have already seen that if two languages $L_1$ and $L_2$ are
regular, then so are $L_1 \cup L_2$, $L_1L_2$, and $L_1^*$ 
(and of course $L_2^*$).  We have not yet seen, however, how the 
common
set operations intersection and complementation affect regularity.
Is the complement of a regular language regular?  How about the
intersection of two regular languages?

Both of these questions can be answered by thinking of regular
languages in terms of their acceptance by DFAs.  Let's consider
first the question of complementation.  Suppose we have an arbitrary
regular language $L$.  We know there is a DFA $M$ that accepts $L$.
Pause a moment and try to think of a modification that you could make
to $M$ that would produce a new machine $M'$ that accepts $\overline{L}$....
Okay, the obvious thing to try is to make $M'$ be a copy of $M$ 
with all final states of $M$ becoming non-final states of $M'$ and
vice versa.  This is in fact exactly right: $M'$ does in fact accept
$\overline{L}$.  To verify this, consider an arbitrary string $w$.  The
transition functions for the two machines $M$ and $M'$ are identical, so $\dstar
(q_0, w)$ is the same state in both $M$ and $M'$; if that state is
accepting in $M$ then it is non-accepting in $M'$, so if $w$ is
accepted by $M$ it is not accepted by $M'$; if the state is
non-accepting in $M$ then it is accepting in $M'$, so if $w$ is
not accepted by $M$ then it is accepted by $M'$. Thus $M'$ accepts
exactly those strings that $M$ does not, and hence accepts $\overline{L}$.  

It is worth pausing for a moment and looking at the above argument
a bit longer.  Would the argument have worked if we had looked at an
arbitrary language $L$ and an arbitrary $N$FA $M$ that accepted $L$?
That is, if we had built a new machine $M'$ in which the final and
non-final states had been switched, would the new NFA $M'$ accept
the complement of the language accepted by $M$?  The answer is
``not necessarily".  Remember that acceptance in an NFA is determined
based on whether or not at least one of the states reached by a
string is accepting.  So any string $w$ with the property that
$\pstar(q_0, w)$ contains both accepting and non-accepting states of $M$
would be accepted both by $M$ and by $M'$.

Now let's turn to the question of intersection.  Given two regular
languages $L_1$ and $L_2$, is $L_1 \cap L_2$ regular?  Again, it is
useful to think in terms of DFAs: given machines $M_1$ and $M_2$
that accept $L_1$ and $L_2$, can you use them to build a new
machine that accepts $L_1 \cap L_2$? The answer is yes, and the
idea behind the construction bears some resemblance to that behind
the NFA-to-DFA construction.  
We want a new machine where transitions reflect the transitions
of both $M_1$ and $M_2$ simultaneously, and we want to accept a
string $w$ only if that those sequences of transitions lead to 
final states in both $M_1$ and $M_2$. So we associate the
states of our new machine $M$ with pairs of states from $M_1$
and $M_2$.  For each state $(q_1,q_2)$ in the new machine and input symbol $a$,
define $\delta((q_1,q_2),a)$ to be the state 
$(\delta_1(q_1,a), \delta_2(q_2,a))$.
The start state $q_0$ of $M$ is
$(q_{01}, q_{02})$, where $q_{0i}$ is the start state
of $M_i$.  The final states of $M$ are the the states of the form $(q_{f1},
q_{f2})$ where $q_{f1}$ is an accepting state of $M_1$ and $q_{f2}$ is an
accepting state of $M_2$.  You should convince yourself that $M$ accepts a
string $x$ iff $x$ is accepted by both $M_1$ and $M_2$.

The results of the previous section and the preceding discussion are summarized
by the following theorem:

\begin{theorem}\label{closure} 
The intersection  of two
regular languages is a regular language.  

The union of two
regular languages is a regular language.  

The concatenation of two
regular languages is a regular language.  

The complement of a regular language is a regular language.

The Kleene closure of a regular language is a regular language.
\end{theorem}
 

\begin{exercises}
\problem Give a DFA that accepts the intersection of the languages accepted by
the machines shown below.  (Suggestion: use the construction discussed in the
chapter just before Theorem~\ref{closure}.)

\fsafig{5ex}

\problem Complete the proof of Theorem~\ref{retonfa} by showing how to modify a
machine that accepts $L(r)$ into a machine that accepts $L(r^*)$.
\problem Using the construction described in Theorem~\ref{retonfa}, build an NFA
that accepts $L((ab\REOR a)^*(bb))$.
\problem Prove that the reverse of a regular language is regular.
\problem Show that for any DFA or NFA, there is an NFA with exactly one final
state that accepts the same language.
\problem Suppose we change the model of NFAs to allow NFAs to have multiple
start states.  Show that for any ``NFA" with multiple start states, there is an
NFA with exactly one start state that accepts the same language.
\problem Suppose that $M_1=(Q_1,\Sigma,q_1,\delta_1,F_1)$ and 
$M_2=(Q_2,\Sigma,q_2,\delta_2,F_2)$ are DFAs over the alphabet $\Sigma$.  It is possible
to construct a DFA that accepts the langauge $L(M_1)\cap L(M_2)$ in a single step.
Define the DFA $$ M = (Q_1\times Q_2, \Sigma, (q_1,q_2), \delta, F_1\times F_2)$$
where $\delta$ is the function from $(Q_1\times Q_2)\times\Sigma$ to $Q_1\times Q_2$
that is defined by: $\delta((p1,p2),\sigma))=(\delta_1(p_1,\sigma),\delta_2(p_2,\sigma))$.
Convince yourself that this definition makes sense.  (For example, note that
states in $M$ are pairs $(p_1,p_2)$ of states, where $p_1\in Q_1$ and $p_2\in Q_2$,
and note that the start state $(q_1,q_2)$ in $M$ is in fact a state in $M$.)
Prove that $L(M)=L(M_1)\cap L(M_2)$, and explain why this shows that the intersection of
any two regular languages is regular.  This proof---if you can get past the
notation---is more direct than the one outlined above.

\end{exercises}


\section{Non-regular Languages}

The fact that our models for mechanical language-recognition accept exactly the
same languages as those generated by our mechanical language-generation system
would seem to be a very positive indication that in ``regular" 
we have in fact managed to
isolate whatever characteristic it is that makes a language ``mechanical". 
Unfortunately, there are languages that we intuitively think of as being
mechanically-recognizable (and which we could write C++ programs to recognize)
that are not in fact regular.

How does one prove that a language is not regular?  We could try proving that
there is no DFA or NFA that accepts it, or no regular expression that generates
it, but this kind of argument is generally rather difficult to make.  It is hard
to rule out all possible automata and all possible regular expressions.  Instead,
we will look at a property that all
regular languages have; proving that a given language does not have this
property then becomes a way of proving that that language is not regular.

Consider the language 
$L = \{ w \in \{a,b\}^* \ | \ n_a(w) =2 \bmod{3},  \ n_b(w)
= 2 \bmod{3} \}$.  Below is a DFA that accepts this language, with states numbered
1 through 9.

\fsafig{21}

Consider the sequence of states that the machine passes through while
processing the string $abbbabb$. Note that there is a repeated state (state
2).  We say that $abbbabb$ ``goes through the state 2 twice", meaning
that in the course of the string being processed, the
machine is in state 2 twice (at least). 
Call the section of the string that takes you around the loop $y$, 
the preceding section $x$,
and the rest $z$.  Then $xz$ is accepted, $xyyz$ is accepted, $xyyyz$ is
accepted, etc. Note that the string $aabb$ cannot
be divided this way, because it does not go through the same state twice. 
Which
strings {\bf can} be divided this way?  
Any string that goes through the same state
twice.  This may include some relatively short strings and must include any
string with length greater than or equal to 9, because there are only 9 states in
the machine, and so repetition must occur after 9 input symbols at the latest.

More generally, consider an arbitrary DFA $M$, and
let the number of states in $M$ be $n$.  Then any string $w$ that is accepted
by $M$ and has $n$ or more symbols must go through the same state twice, and
can therefore be broken up into three pieces $x,y,z$ (where $y$ contains at
least one symbol) so that $w=xyz$ and

$xz$ is accepted by $M$

$xyz$ is accepted by $M$ (after all, we started with $w$ in $L(M)$)

$xyyz$ is accepted by $M$ 

etc.

Note that you can actually say even more: within the first $n$ characters of
$w$ you must already get a repeated state, so you can always find an $x,y,z$ as
described above where, in addition, the $xy$ portion of $w$ (the portion of $w$
that takes you to and back to a repeated state) contains at most $n$ symbols.

So altogether, if $M$ is an $n$-state DFA that accepts $L$, and $w$ is a string
in $L$ whose length is at least $n$, then $w$ can be broken down into three
pieces $x$, $y$, and $z$, $w=xyz$, such that

(i) $x$ and $y$ together contain no more than $n$ symbols;

(ii) $y$ contains at least one symbol;

(iii) $xz$ is accepted by $M$

\ \ \ \ \ ($xyz$ is accepted by $M$)

\ \ \ \ \ $xyyz$ is accepted by $M$

\ \ \ \ \ etc.

\smallskip

The usually-stated form of this result is the Pumping Lemma:

\begin{theorem} If L is a regular language, then there is some number $n>0$ 
such that any
string $w$ in $L$ whose length is greater than or equal to $n$ can
be broken down into three
pieces $x$, $y$, and $z$, $w=xyz$, such that
\begin{itemize}
\item[(i)] $x$ and $y$ together contain no more than $n$ symbols;

\item[(ii)] $y$ contains at least one symbol;

\item[(iii)] $xz$ is accepted by $M$

($xyz$ is accepted by $M$)

$xyyz$ is accepted by $M$

etc.
\end{itemize}
\end{theorem}

Though the Pumping Lemma says something about regular languages, it is not used
to prove that languages are regular.  It says ``{\bf if} a language is regular,
then certain things happen", not ``if certain things happen, {\bf then} you can conclude
that the language is regular."  However, the Pumping Lemma is useful for
proving that languages are not regular, since the contrapositive of ``if a
language is regular
then certain things happen" is ``if certain things don't happen then you can conclude
that the language is not regular."  So what are the ``certain things"?  
Basically,
the P.L. says that if a language is regular, there is some ``threshold" 
length for
strings, and every string that goes over that threshold can be broken down in a
certain way.  Therefore, if we can show that ``there is some threshold length for
strings such that every string that goes over that threshold can be broken down in a 
certain way" is a false assertion about a language, we can conclude that the
language is not regular. How do you show that there is no threshold length? 
Saying a number is a threshold length for a language means that every string in
the language that is at least that long can be broken down in the ways
described.  So to show that a number is not a threshold value, we have to show
that there is some string in the language that is at least that long that cannot
be broken down in the appropriate way.

\begin{theorem}
$\{ a^nb^n \ | \ n \geq 0 \}$ is not regular.
\end{theorem}

\begin{proof} We do this by showing that there is no threshold value for the language.  Let
$N$ be an arbitrary candidate for threshold value.  We want to show that it is
not in fact a threshold value, so we want to find a string in the language
whose length is at least $N$ and which can't be broken down in the way
described by the Pumping Lemma.  What string should we try to prove
unbreakable?  We can't pick strings like $a^{100}b^{100}$ because we're
working with an arbitrary $N$ i.e.\ making no assumptions about $N$'s value; 
picking $a^{100}b^{100}$ is implicitly assuming that $N$ is no bigger than 200
--- for larger values of $N$, $a^{100}b^{100}$ would not be ``a string whose
length is at least $N$".  Whatever string we pick, we {\bf have} to be sure 
that its
length is at least $N$, no matter what number $N$ is.  So we pick, for instance,
$w = a^Nb^N$.  This string is in the language, and its length is at least $N$,
no matter what number $N$ is. 
If we can show that this string can't be broken down as described by the
Pumping Lemma, then we'll have shown that $N$ doesn't work as a threshold
value, and since $N$ was an arbitrary number, we will have shown that there is no
threshold value for $L$ and hence $L$ is not regular.  So let's show that $w =
a^Nb^N$ can't be broken down appropriately. 

We need to show that you can't
write $w = a^Nb^N$ as $w=xyz$ where $x$ and $y$ together contain at most $N$
symbols, $y$ isn't empty, and all the strings $xz$, $xyyz$, $xyyyz$, etc.\ are
still in $L$, i.e.\ of the form $a^nb^n$ for some number~$n$.  The best way to
do this is to show that any choice for $y$ (with $x$ being whatever precedes it
and $z$ being whatever follows) that satisfies the first two requirements fails
to satisfy the third.  So what are our possible choices for $y$?
Well, since $x$ and $y$ together can contain at most $N$
symbols, and $w$ starts with $N$ $a$'s, both $x$ and $y$ must be made up
entirely of $a$'s; since $y$ can't be empty, it must contain at least one $a$
and (from (i)) no more than $N$ $a$'s. So the possible
choices for $y$ are $y=a^k$ for some $1 \leq k \leq N$. We want to show now
that none of these choices will satisfy the third requirement by showing that
for any value of $k$, at least one of the strings $xz$, $xyyz$, $xyyyz$, etc
will not be in $L$.  No matter what value we try for $k$, we don't have to 
look far for our rogue string: the string
$xz$, which is $a^Nb^N$ with $k$ $a$'s deleted from it, looks like
$a^{N-k}b^N$, which is clearly not of the form $a^nb^n$.  So the only
$y$'s that satisfy (i) and (ii) don't satisfy (iii); so $w$ can't be broken
down as required; so $N$ is not a threshold value for $L$; and since $N$ was an
arbitrary number, there is no threshold value for $L$; so $L$ is not regular.
\end{proof}

\smallskip

The fact that languages like $\{a^nb^n\ | \ n \geq 0\}$ and $\{a^p \ |\ p \mbox{
is prime}\}$ are not regular is a severe blow to any idea that regular
expressions or finite-state automata capture the language-generation or
language-recognition capabilities of a computer: They are both languages that 
we could easily write programs to recognize.  It is not clear how the expressive
power of regular expressions could be increased, nor how one might modify the
FSA model to obtain a more powerful one.  However, 
in the next chapter you will be
introduced to the concept of a {\em grammar} as a tool for generating languages. 
The simplest grammars still only produce regular languages, but you will see
that more complicated grammars have the power to generate languages far beyond
the realm of the regular.

\begin{exercises}
\problem Use the Pumping Lemma to show that the following languages over $\ab$ 
are not regular.
\ppart $L_1 = \{ x \ | \ n_a(x) = n_b(x)\}$
\ppart $L_2 = \{ xx \ | \ x \in \ab^*\}$
\ppart $L_3 = \{ xx^R \ | \ x \in \ab^*\}$
\ppart $L_4 = \{ a^nb^m \ | \ n < m \}$

\end{exercises}




% !TEX root = main.tex

\chapter{Grammars}\label{C-grammars}


\startchapter{Both natural languages, }such as English and the
artificial languages used for programming have a structure
known as grammar or syntax.  In order to form legal sentences
or programs, the parts of the language must be fit together
according to certain rules.  For natural languages, the
rules are somewhat informal (although high-school English
teachers might have us believe differently).  For programming
languages, the rules are absolute, and programs that violate
the rules will be rejected by a compiler.

In this chapter, we will study formal grammars and languages
defined by them.  The languages we look at will, for the most part,
be ``toy'' languages, compared to natural languages or even
to programming languages, but the ideas and techniques are basic
to any study of language.  In fact, many of the ideas arose
almost simultaneously in the 1950s in the work of linguists who were
studying natural language and programmers who were looking for
ways to specify the syntax of programming languages.

The grammars in this chapter are \nw[generative grammar]{generative grammars}.
A generative grammar is a set of rules that can be used to generate
all the legal strings in a language.  We will also consider the closely
related idea of \nw{parsing}.  To parse a string means to determine
how that string can be generated according to the rules.

This chapter is a continuation of the preceding chapter.  
Like a regular expression, a grammar is a way to specify a possibly
infinite language with a finite amount of information.  In fact,
we will see that every regular language can be specified
by a certain simple type of grammar.  We will also see that some languages
that can be specified by grammars are not regular.

\section{Context-free Grammars}\label{S-grammars-1}

In its most general form, a grammar is a set of \nw[rewriting rule]{rewriting
rules}.  A rewriting rule specifies that a certain string of symbols can
be substituted for all or part of another string.  If $w$ and $u$ are
strings, then $w\PRODUCES u$ is a rewriting rule that specifies that
the string $w$ can be replaced by the string $u$.  The symbol ``$\PRODUCES$''
is read ``can be rewritten as.''  Rewriting rules are also called
\nw[production rule]{production rules} or \nw{productions}, and
``$\PRODUCES$'' can also be read as ``produces.''  For example,
if we consider strings over the alphabet $\{a,b,c\}$, then
the production rule $aba\PRODUCES cc$ can be applied to the
string $abbabac$ to give the string $abbccc$.  The substring $aba$
in the string $abbabac$ has been replaced with $cc$.

In a \nw{context-free grammar}, every rewriting rule has the
form $A\PRODUCES w$, where $A$ is single symbol and $w$ is a string
of zero or more symbols.  (The grammar is ``context-free'' in the
sense that $w$ can be substituted for $A$ wherever $A$ occurs in a string,
regardless of the surrounding context in which $A$ occurs.)
The symbols that occur on the left-hand
sides of production rules in a context-free grammar
are called \nw[non-terminal symbol]{non-terminal symbols}.
By convention, the non-terminal symbols are usually  uppercase letters.
The strings on the right-hand sides of the production rules can
include non-terminal symbols as well as other symbols, which are
called \nw[terminal symbol]{terminal symbols}.  By convention, the
terminal symbols are usually lowercase letters.  Here are some
typical production rules that might occur in context-free grammars:
\begin{align*}
   A&\PRODUCES aAbB\\
   S&\PRODUCES SS\\
   C&\PRODUCES Acc\\
   B&\PRODUCES b\\
   A&\PRODUCES\EMPTYSTRING
\end{align*}
In the last rule in this list, $\EMPTYSTRING$ represents the empty string,
as usual.  For example, this rule could be applied to the string
$aBaAcA$ to produce the string $aBacA$.  The first occurrence of
the symbol $A$ in $aBaAcA$ has been replaced by the empty string---which
is just another way of saying that the symbol has been dropped from the string.

In every context-free grammar, one of the non-terminal symbols is
designated as the \nw{start symbol} of the grammar.  The start symbol
is often, though not always, denoted by~$S$.  When the grammar
is used to generate strings in a language, the idea is to start
with a string consisting of nothing but the start symbol.  Then
a sequence of production rules is applied.  Each application of
a production rule to the string transforms the string to a new
string.  If and when this process produces a string that consists
purely of terminal symbols, the process ends.  That string of
terminal symbols is one of the strings in the language generated
by the grammar.  In fact, the language consists precisely of
all strings of terminal symbols that can be produced in this way.

As a simple example, consider a grammar that has three production
rules: $S\PRODUCES aS$, $S\PRODUCES bS$, and $S\PRODUCES b$.
In this example, $S$ is the only non-terminal symbol, and
the terminal symbols are $a$ and $b$.  Starting from the
string $S$, we can apply any of the three rules of the grammar
to produce either $aS$, $bS$, or $b$.  Since the string $b$ contains
no non-terminals, we see that $b$ is one of the strings in the language
generated by this grammar.  The strings $aS$ and $bS$ are not in
that language, since they contain the non-terminal symbol $S$,
but we can continue to apply production rules to these strings.
From $aS$, for example, we can obtain $aaS$, $abS$, or $ab$.
From $abS$, we go on to obtain $abaS$, $abbS$, or $abb$.
The strings $ab$ and $abb$ are in the language generated by
the grammar.  It's not hard to see that any string of $a$'s and
$b$'s that ends with a $b$ can be generated by this grammar,
and that these are the only strings that can be generated.
That is, the language generated by this grammar is the regular
language specified by the regular expression $(a\REOR b)^{*}b$.

It's time to give some formal definitions of the concepts which
we have been discussing.

\begin{definition}
A \nw{context-free grammar} is a 4-tuple $(V,\Sigma,P,S)$,
where:

\Item{1.\ }$V$ is a finite set of symbols.  The elements of $V$
are the non-terminal symbols of the grammar.

\Item{2.\ }$\Sigma$ is a finite set of symbols such that $V\cap\Sigma=\emptyset$.
The elements of $\Sigma$ are the terminal symbols of the grammar.

\Item{3.\ }$P$ is a set of production rules.  Each rule is of the
form $A\PRODUCES w$ where $A$ is one of the symbols in $V$ and
$w$ is a string in the language $(V\cup\Sigma)^*$.

\Item{4.\ }$S\in V$.  $S$ is the start symbol of the grammar.

\end{definition}

Even though this is the formal definition, grammars are often
specified informally simply by listing the set of production rules.
When this is done it is assumed, unless otherwise specified,
that the non-terminal symbols are just the symbols that occur
on the left-hand sides of production rules of the grammar.
The terminal symbols are all the other symbols that occur on
the right-hand sides of production rules.  The start symbol is the
symbol that occurs on the left-hand side of the first production
rule in the list.  Thus, the list of production rules
\begin{align*}
   T&\PRODUCES TT\\
   T&\PRODUCES A\\
   A&\PRODUCES aAa\\
   A&\PRODUCES bB\\
   B&\PRODUCES bB\\
   B&\PRODUCES \EMPTYSTRING 
\end{align*}
specifies a grammar $G=(V,\Sigma,P,T)$ where $V$ is $\{T,A,B\}$,
$\Sigma$ is $\{a,b\}$, and $T$ is the start symbol.  $P$,~of course, is a
set containing the six production rules in the list.


Let $G=(V,\Sigma,P,S)$ be a context-free grammar.
Suppose that $x$ and $y$ are strings in the language $(V\cup\Sigma)^*$.
The notation $x\YIELDS_G y$ is used to express the fact
that $y$ can be obtained from $x$ by applying one of the production
rules in $P$.  To be more exact, we say that $x\YIELDS_G y$
if and only if there is a production rule $A\PRODUCES w$ in the grammar
and two strings $u$ and $v$ in the language $(V\cup\Sigma)^*$
such that $x=uAv$ and $y=uwv$. The fact
that $x=uAv$ is just a way of saying that $A$ occurs somewhere in
$x$.  When the production rule $A\PRODUCES w$ is applied to
substitute $w$ for $A$ in $uAv$, the result is $uwv$, which is $y$.
Note that either $u$ or $v$ or both can be the empty string.

If a string $y$ can be obtained from a string $x$ by
applying a sequence of zero or more production rules, we
write $x\YIELDS_G^* y$.  In most cases, the ``$G$'' in
the notations $\YIELDS_G$ and $\YIELDS_G^*$ will be omitted,
assuming that the grammar in question is understood.
Note that $\YIELDS$ is a relation on the set $(V\cup\Sigma)^*$.
The relation $\YIELDSTAR$ is the reflexive, transitive closure of that relation.
(This explains the use of ``$*$'', which is usually used to
denote the transitive, but not necessarily reflexive, closure of a relation. 
In this case, $\YIELDSTAR$ is reflexive as well as transitive since
$x\;\YIELDSTAR x$ is true for any string~$x$.)
For example, using the grammar that is defined by the above
list of production rules, we have
\begin{align*}
 aTB&\YIELDS aTTB\\
    &\YIELDS aTAB\\
    &\YIELDS aTAbB\\
    &\YIELDS aTbBbB\\
    &\YIELDS aTbbB
\end{align*}
From this, it follows that $aTB\;\YIELDSTAR aTbbB$.  The relation $\YIELDS$
is read ``yields'' or ``produces'' while $\YIELDSTAR$ can be
read ``yields in zero or more steps'' or ``produces in zero or more
steps.''  The following theorem states some simple facts about
the relations $\YIELDS$ and $\YIELDSTAR$:

\begin{theorem}\label{T-yields}
Let $G$ be the context-free grammar $(V,\Sigma,P,S)$.
Then:
\Item{1. }If $x$ and $y$ are strings in $(V\cup\Sigma)^*$ such that $x\YIELDS y$, 
then $x\;\YIELDSTAR y$.
\Item{2. }If $x$, $y$, and $z$ are strings in $(V\cup\Sigma)^*$ such that $x\;\YIELDSTAR y$
and $y\;\YIELDSTAR z$, then $x\;\YIELDSTAR z$.
\Item{3. }If $x$ and $y$ are strings in $(V\cup\Sigma)^*$ such that $x\YIELDS y$, 
and if $s$ and $t$ are any strings in $(V\cup\Sigma)^*$, then $sxt\YIELDS syt$.
\Item{4. }If $x$ and $y$ are strings in $(V\cup\Sigma)^*$ such that $x\;\YIELDSTAR y$, 
and if $s$ and $t$ are any strings in $(V\cup\Sigma)^*$, then $sxt\;\YIELDSTAR syt$.
\end{theorem}
\begin{proof}
Parts 1 and 2 follow from the fact that $\YIELDSTAR$ is the transitive
closure of $\YIELDS$.  Part 4 follows easily from Part 3.  (I leave this
as an exercise.)  To prove Part 3, suppose that $x$, $y$, $s$, and $t$
are strings such that $x\YIELDS y$.  By definition, this means that
there exist strings $u$ and $v$ and a production rule $A\PRODUCES w$
such that $x=uAv$ and $y=uwv$.  But then we also have
$sxt=suAvt$ and $syt=suwvt$.  These two equations, along with
the existence of the production rule $A\PRODUCES w$ show, by definition,
that $sxt\YIELDS syt$.
\end{proof}

We can use $\YIELDSTAR$ to give a formal definition
of the language generated by a context-free grammar:

\begin{definition}
Suppose that  $G=(V,\Sigma,P,S)$ is a context-free grammar.
Then the language generated by $G$ is the language $L(G)$ over
the alphabet $\Sigma$ defined by
\[L(G)=\{w\in \Sigma^*\st S\YIELDS_G^* w\}\]
That is, $L(G)$ contains any string of terminal symbols that can be 
obtained by starting with the string consisting of the start symbol, $S$, 
and applying a sequence of production rules.  

A language $L$ is said to be a \nw{context-free language} if
there is a context-free grammar $G$ such that $L(G)$ is $L$.
Note that there might be many different context-free grammars
that generate the same context-free language.  Two context-free
grammars that generate the same language are said to be
\nw[equivalent grammars]{equivalent}.
\end{definition}

Suppose $G$ is a context-free grammar with start symbol $S$
and suppose $w\in L(G)$.  By definition, this means that
there is a sequence of one or more applications of production rules
which produces the string $w$ from $S$.  This sequence has the
form $S\YIELDS x_1\YIELDS x_2\YIELDS\cdots\YIELDS w$.  Such a sequence
is called a \nw{derivation} of $w$ (in the grammar $G$).  Note
that $w$ might have more than one derivation.  That is, it might
be possible to produce $w$ in several different ways.

Consider the language $L=\{a^nb^n\st n\in\N\}$.  We already know
that $L$ is not a regular language.  However, it is a context-free
language.  That is, there is a context-free grammar such that
$L$ is the language generated by $G$.  This gives us our first
theorem about grammars:

\begin{theorem}
Let $L$ be the language $L=\{a^nb^n\st n\in\N\}$.  Let $G$ be
the context-free grammar $(V,\Sigma,P,S)$ where $V=\{S\}$,
$\Sigma=\{a,b\}$ and $P$ consists of the productions
\begin{align*}
     S&\PRODUCES aSb\\
     S&\PRODUCES \EMPTYSTRING 
\end{align*}
Then $L=L(G)$, so that $L$ is a context-free language.  In particular,
there exist context-free languages which are not regular.
\end{theorem}
\begin{proof}
To show that $L=L(G)$, we must show both that $L\SUB L(G)$
and that $L(G)\SUB L$.  To show that $L\SUB L(G)$, let $w$
be an arbitrary element of $L$.  By definition of $L$,
$w=a^nb^n$ for some $n\in\N$.  We show that $w\in L(G)$ by
induction on $n$.  In the case where $n=0$, we have $w=\EMPTYSTRING$.
Now, $\EMPTYSTRING\in L(G)$ since $\EMPTYSTRING$ can be produced
from the start symbol $S$ by an application of the rule $S\PRODUCES\EMPTYSTRING$,
so our claim is true for $n=0$.
Now, suppose that $k\in\N$ and that we already know that $a^kb^k\in L(G)$.
We must show that $a^{k+1}b^{k+1}\in L(G)$.  Since 
$S\;\YIELDSTAR a^kb^k$, we also have, by Theorem \ref{T-yields}, that
$aSb\;\YIELDSTAR aa^kb^kb$.
That is, $aSb\;\YIELDSTAR a^{k+1}b^{k+1}$.  Combining this with the
production rule $S\PRODUCES aSb$, we see that $S\;\YIELDSTAR a^{k+1}b^{k+1}$.
This means that $a^{k+1}b^{k+1}\in L(G)$, as we wanted to show.
This completes the proof that $L\SUB L(G)$.

To show that $L(G)\SUB L$, suppose that $w\in L(G)$.  That is,
$S\;\YIELDSTAR w$.  We must show that $w=a^nb^n$ for some $n$.
Since $S\;\YIELDSTAR w$, there is a derivation
$S\YIELDS x_0\YIELDS x_1\YIELDS\cdots\YIELDS x_n$, where $w=x_n$.
We first prove by induction on $n$ that in any derivation
$S\YIELDS x_0\YIELDS x_1\YIELDS\cdots\YIELDS x_n$,
we must have either $x_n=a^nb^n$ or $x_n=a^{n+1}Sb^{n+1}$.
Consider the case $n=0$.  Suppose $S\YIELDS x_0$.
Then, we must have that $S\PRODUCES x_0$ is a rule in the grammar,
so $x_0$ must be either $\EMPTYSTRING$ or $aSb$. Since $\EMPTYSTRING=a^0b^0$
and $aSb=a^{0+1}Sb^{0+1}$, $x_0$ is of the required form. 
Next, consider the inductive case.  Suppose that $k>1$ and we already 
know that in any
derivation $S\YIELDS x_0\YIELDS x_1\YIELDS\cdots\YIELDS x_k$,
we must have $x_k=a^kb^k$ or $x=a^{k+1}Sb^{k+1}$.  Suppose that
$S\YIELDS x_0\YIELDS x_1\YIELDS\cdots\YIELDS x_k\YIELDS x_{k+1}$.
We know by induction that $x_k=a^kb^k$ or $x=a^{k+1}Sb^{k+1}$,
but since $x_k\YIELDS x_{k+1}$ and $a^kb^k$ contains no non-terminal
symbols, we must have $x_k=a^{k+1}Sb^{k+1}$.  Since $x_{k+1}$
is obtained by applying one of the production rules $S\PRODUCES\EMPTYSTRING$
or $S\PRODUCES aSb$ to~$x_k$, $x_{k+1}$ is either $a^{k+1}\EMPTYSTRING b^{k+1}$
or $a^{k+1}aSbb^{k+1}$.  That is, $x_{k+1}$ is either $a^{k+1}b^{k+1}$
or $a^{k+2}Sb^{k+2}$, as we wanted to show.  This completes the induction.
Turning back to $w$, we see that $w$ must be of the form $a^nb^n$ or
of the form $a^nSb^n$.  But since $w\in L(G)$, it can contain no
non-terminal symbols, so $w$ must be of the form $a^nb^n$, as we wanted to show.
This completes the proof that $L(G)\SUB L$.
\end{proof}
I have given a very formal and detailed proof of this theorem, to show how it
can be done and to show how induction plays a role in many proofs about
grammars.  However, a more informal proof of the theorem would probably
be acceptable and might even be more convincing.  To show that
$L\SUB L(G)$, we could just note that the derivation
$S\YIELDS aSb\YIELDS a^2Sb^2\YIELDS\cdots\YIELDS a^nSb^n\YIELDS a^nb^n$
demonstrates that $a^nb^n\in L$.  On the other hand, it is clear that every
derivation for this grammar must be of this form, so every string in $L(G)$
is of the form $a^nb^n$.

For another example, consider the language $\{a^nb^m\st n\ge m\ge0\}$.
Let's try to design a grammar that generates this language.
This is similar to the previous example, but now we want to include strings that
contain more $a$'s than $b$'s.  The production rule $S\PRODUCES aSb$
always produces the same number of $a$'s and $b$'s.  Can we modify
this idea to produce more $a$'s than $b$'s?

One approach would be to produce a string containing just as many
$a$'s as $b$'s, and then to add some extra $a$'s.  A rule that can
generate any number of $a$'s is $A\PRODUCES aA$. After
applying the rule $S\PRODUCES aSb$ for a while, we want to move
to a new state in which we apply the rule $A\PRODUCES aA$.  We can
get to the new state by applying a rule $S\PRODUCES A$
that changes the $S$ into an $A$.  We still need a way to 
finish the process, which means getting rid of all non-terminal
symbols in the string.  For this, we can use the rule $A\PRODUCES\EMPTYSTRING$.
Putting these rules together, we get the grammar
\begin{align*}
      S&\PRODUCES aSb\\
      S&\PRODUCES A\\
      A&\PRODUCES aA\\
      A&\PRODUCES \EMPTYSTRING
\end{align*}
This grammar does indeed generate the language $\{a^nb^m\st n\ge m\ge 0\}$.
With slight variations on this grammar, we can produce other related
languages.  For example, if we replace the rule $A\PRODUCES \EMPTYSTRING$
with $A\PRODUCES a$, we get the language $\{a^nb^m\st n> m\ge 0\}$.

There are other ways to generate the language $\{a^nb^m\st n\ge m\ge 0\}$.
For example, the extra non-terminal symbol, $A$, is not really necessary,
if we allow $S$ to sometimes produce a single $a$ without a $b$.  This
leads to the grammar
\begin{align*}
      S&\PRODUCES aSb\\
      S&\PRODUCES aS\\
      S&\PRODUCES \EMPTYSTRING
\end{align*}
(But note that the rule $S\PRODUCES Sa$ would not work in place
of $S\PRODUCES aS$, since it would allow the production of
strings in which an $a$ can follow a $b$, and there are no
such strings in the language $\{a^nb^m\st n\ge m\ge 0\}$.)
And here are two more grammars that generate this language:
\begin{align*}
      S&\PRODUCES AB             &\qquad S&\PRODUCES ASb\\
      A&\PRODUCES aA             &\qquad A&\PRODUCES aA\\
      B&\PRODUCES aBb            &\qquad S&\PRODUCES\EMPTYSTRING\\
      A&\PRODUCES \EMPTYSTRING   &\qquad A&\PRODUCES\EMPTYSTRING\\
      B&\PRODUCES \EMPTYSTRING   &\qquad  &
\end{align*}

\medbreak

Consider another variation on the language $\{a^nb^n\st n\in\N\}$,
in which the $a$'s and $b$'s can occur in any order, but the number
of $a$'s is still equal to the number of $b$'s.  This language
can be defined as
$L=\{w\in\{a,b\}^*\st n_a(w) = n_b(w)\}$.  This language includes
strings such as $abbaab$, $baab$, and $bbbaaa$.

Let's start with the grammar containing the rules $S\PRODUCES aSb$
and $S\PRODUCES\EMPTYSTRING$.  We can try adding the rule
$S\PRODUCES bSa$.  Every string that can be generated using these
three rules is in the language $L$.  However, not every string
in $L$ can be generated.  A derivation that starts with $S\YIELDS aSb$
can only produce strings that begin with $a$ and end with $b$.
A derivation that starts with $S\YIELDS bSa$ can only generate strings
that begin with $b$ and end with $a$.  There is no way to generate
the strings $baab$ or $abbbabaaba$, which are in the language $L$.
But we shall see that any string in $L$ that begins and
ends with the same letter can be written in the form $xy$ where
$x$ and $y$ are shorter strings in $L$.  To produce strings of
this form, we need one more rule, $S\PRODUCES SS$.  The complete set of
production rules for the language $L$ is
\begin{align*}
    S&\PRODUCES aSb\\
    S&\PRODUCES bSa\\
    S&\PRODUCES SS\\
    S&\PRODUCES \EMPTYSTRING
\end{align*}
It's easy to see that every string that can be generated using these
rules is in $L$, since each rule introduces the same number of
$a$'s as $b$'s.  But we also need to check that every string
$w$ in $L$ can be generated by these rules.  This can be done
by induction on the length of $w$, using the second form
of the principle of mathematical induction.  In the base case,
$|w|=0$ and $w=\EMPTYSTRING$.  In this case, $w\in L$ since
$S\YIELDS\EMPTYSTRING$ in one step.
Suppose $|w|=k$, where $k>0$, and suppose that
we already know that for any $x\in L$ with $|x|<k$, $S\;\YIELDSTAR x$.
To finish the induction we must show, based on this induction 
hypothesis, that $S\;\YIELDSTAR w$.

Suppose that the first and last characters of $w$ are
different.  Then $w$ is either of the form $axb$ or of the form $bxa$,
for some string $x$.  Let's assume that $w$ is of the form $axb$.
(The case where $w$ is of the form $bxa$ is handled in a similar way.)
Since $w$ has the same number of $a$'s and $b$'s
and since $x$ has one fewer $a$ than $w$ and one fewer $b$ than $w$,
$x$ must also have the same number of $a$'s as $b$'s.  That is $x\in L$.
But $|x|=|w|-2<k$, so by the induction hypothesis, $x\in L(G)$.
So we have $S\;\YIELDSTAR x$.  By Theorem~\ref{T-yields}, we
get then $aSb\;\YIELDSTAR axb$.  Combining this with the fact
that $S\YIELDS aSb$, we get that $S\;\YIELDSTAR axb$, that is,
$S\;\YIELDSTAR w$.  This proves that $w\in L(G)$.

Finally, suppose that the first and last characters of $w$ are
the same.  Let's say that $w$ begins and ends with $a$.  (The case
where $w$ begins and ends with $b$ is handled in a similar way.)
I claim that $w$ can be written in the form $xy$ where $x\in L(G)$
and $y\in L(G)$ and neither $x$ nor $y$ is the empty string.
This will finish the induction, since we will then have by
the induction hypothesis that $S\;\YIELDSTAR x$
and $S\;\YIELDSTAR y$, and we can derive $xy$ from $S$ by first
applying the rule $S\PRODUCES SS$ and then using the first
$S$ on the right-hand side to derive $x$ and the second to derive $y$.

It only remains to figure out how to divide $w$ into two strings
$x$ and $y$ which are both in $L(G)$.  The technique that is used
is one that is more generally useful.  Suppose that $w=c_1c_2\cdots c_k$,
where each $c_i$ is either $a$ or $b$.  Consider the sequence of
integers $r_1$, $r_2$, \dots, $r_k$ where for each $i = 1, 2, \dots, k$,
$r_i$ is the number of $a$'s in $c_1c_2\cdots c_i$ minus the
number of $b$'s in $c_1c_2\cdots c_i$.  Since $c_1=a$, $r_1=1$.
Since $w\in L$, $r_k=0$.  And since $c_k=a$, we must have $r_{k-1}=
r_k-1 = -1$.  Furthermore the difference between $r_{i+1}$
and $r_i$ is either $1$ or~$-1$, for $i=1,2,\dots,k-1$.

Since $r_1=1$ and $r_{k-1}=-1$ and the value of $r_i$ goes up or down
by 1 when $i$ increases by 1, $r_i$ must be zero for some $i$
between 1 and $k-1$.  That is, $r_i$ cannot get from 1 to~$-1$ unless
it passes through zero. Let $i$ be a number between 1 and $k-1$ such
that $r_i=0$.  Let $x=c_1c_2\cdots c_i$ and let $y=c_{i+1}c_{i+2}\cdots c_k$.
Note that $xy=w$.  The fact that $r_i=0$ means that 
the string $c_1c_2\cdots c_i$ has the same number of $a$'s and
$b$'s, so $x\in L(G)$.  It follows automatically that $y\in L(G)$
also.  Since $i$ is strictly between 1 and $k-1$, neither $x$ nor
$y$ is the empty string.  This is all that we needed to show
to finish the proof that $L=L(G)$.

The basic idea of this proof is that if $w$ contains the same 
number of $a$'s as $b$'s, then an $a$ at the beginning
of $w$ must have a ``matching'' $b$ somewhere in $w$.  This
$b$ matches the $a$ in the sense that the corresponding $r_i$ is
zero, and the $b$ marks the end of a string $x$ which contains
the same number of $a$'s as $b$'s.  For example, in the
string $aababbabba$, the $a$ at the beginning of the string
is matched by the third $b$, since $aababb$ is the shortest
prefix of $aababbabba$ that has an equal number of $a$'s
and $b$'s.

Closely related to this idea of matching $a$'s and $b$'s is
the idea of \nw{balanced parentheses}.  Consider a string
made up of parentheses, such as \texttt{(()(()))(())}.
The parentheses in this sample string are balanced because
each left parenthesis has a matching right parenthesis,
and the matching pairs are properly nested.  A careful definition
uses the sort of integer sequence introduced in the above
proof.  Let $w$ be a string of parentheses.  Write
$w=c_1c_2\cdots c_n$, where each $c_i$ is either \texttt{(}
or \texttt{)}.  Define a sequence of integers $r_1$, $r_2$, \dots,~$r_n$,
where $r_i$ is the number of left parentheses in $c_1c_2\cdots c_i$
minus the number of right parentheses.  We say that the parentheses
in $w$ are balanced if $r_n=0$ and $r_i\ge0$ for all $i=1,2,\dots,n$.
The fact that $r_n=0$ says that $w$ contains the same number of
left parentheses as right parentheses.  The fact the $r_i\ge0$
means that the nesting of pairs of parentheses is correct:
You can't have a right parenthesis unless it is balanced by a left
parenthesis in the preceding part of the string.  The language
that consists of all balanced strings of parentheses is context-free.  
It is generated by the grammar
\begin{align*}
   S&\PRODUCES (\,S\,)\\
   S&\PRODUCES SS\\
   S&\PRODUCES \EMPTYSTRING
\end{align*}
The proof is similar to the preceding proof about strings of
$a$'s and $b$'s.  (It might seem that I've made an awfully big
fuss about matching and balancing.  The reason is that this
is one of the few things that we can do with context-free languages
that we can't do with regular languages.)

\medbreak

Before leaving this section, we should look at a few more
general results.  Since we know that most operations on regular
languages produce languages that are also regular, we can
ask whether a similar result holds for context-free languages.
We will see later that the intersection of two context-free
languages is not necessarily context-free.  Also, the
complement of a context-free language is not necessarily
context-free.  However, some other operations on context-free
languages do produce context-free languages.

\begin{theorem}\label{T-CFG-closures}
Suppose that $L$ and $M$ are context-free languages.
Then the languages $L\cup M$, $LM$, and $L^*$ are also
context-free.
\end{theorem}
\begin{proof}
I will prove only the first claim of the theorem, that $L\cup M$ is
context-free.  In the exercises for this section, you are
asked to construct grammars for $LM$ and $L^*$ (without giving
formal proofs that your answers are correct).

Let $G=(V,\Sigma,P,S)$ and $H=(W,\Gamma,Q,T)$ be context-free grammars
such that $L=L(G)$ and $M=L(H)$.  We can assume that $W\cap V=\emptyset$,
since otherwise we could simply rename the non-terminal symbols in $W$.
The idea of the proof is that to generate a string in $L\cup M$,
we first decide whether we want a string in $L$ or a string in $M$.
Once that decision is made, to make a string in $L$, we use production
rules from $G$, while to make a string in $M$, we use rules from $H$.
We have to design a grammar, $K$, to represent this process.

Let $R$ be a symbol that is not in any of the alphabets $V$, $W$, $\Sigma$,
or $\Gamma$.  $R$ will be the start symbol of $K$.  The production rules
for $K$ consist of all the production rules from $G$ and $H$ together
with two new rules:
\begin{align*}
   R\PRODUCES S\\
   R\PRODUCES T
\end{align*}
Formally, $K$ is defined to be the grammar
\[ (V\cup W\cup\{R\},
       P\cup Q\cup \{R\PRODUCES S, R\PRODUCES T\},
       \Sigma\cup\Gamma, R) \]
Suppose that $w\in L$.  That is $w\in L(G)$, so there is
a derivation $S\YIELDS_G^*w$.
Since every rule from $G$ is also a rule in $K$, if follows that
$S\YIELDS_K^* w$.  Combining this with the fact that $R\YIELDS_K S$, we have
that $R\YIELDS_K^* w$, and $w\in L(K)$.  This shows that $L\SUB L(K)$.
In an exactly similar way, we can show that $M\SUB L(K)$.
Thus, $L\cup M\SUB L(K)$.

It remains to show that $L(K)\SUB L\cup M$.  Suppose $w\in L(K)$.
Then there is a derivation $R\YIELDS_K^*w$.  This derivation must
begin with an application of
one of the rules $R\PRODUCES S$ or $R\PRODUCES T$, since these are the
only rules in which $R$ appears.  If the first rule applied in the
derivation is $R\PRODUCES S$, then the remainder of the derivation
shows that $S\YIELDS_K^*w$.  Starting from $S$, the only rules
that can be applied are rules from $G$, so in fact we have
$S\YIELDS_G^*w$.  This shows that $w\in L$.  Similarly, if
the first rule applied in the derivation $R\YIELDS_K^*w$ is 
$R\PRODUCES T$, then $w\in M$.  In any case, $w\in L\cup M$.
This proves that $L(K)\SUB L\cup M$.
\end{proof}

Finally, we should clarify the relationship between context-free
languages and regular languages.  We have already seen that
there are context-free languages which are not regular.
On the other hand, it turns out that every regular language
is context-free.  That is, given any regular language, there
is a context-free grammar that generates that language.  This
means that any syntax that can be expressed by a regular expression,
by a DFA, or by an NFA could also be expressed by a context-free
grammar.  In fact, we only need a certain restricted type of
context-free grammar to duplicate the power of regular expressions.

\begin{definition}
A \nw{right-regular grammar}\index{regular grammar} is a context-free 
grammar in which the right-hand side of every production
rule has one of the following forms:  the empty string;
a string consisting of a single non-terminal symbol; or
a string consisting of a single terminal symbol followed
by a single non-terminal symbol.
\end{definition}

Examples of the types of production rule that are allowed in
a right-regular grammar are $A\PRODUCES\EMPTYSTRING$,
$B\PRODUCES C$, and $D\PRODUCES aE$.  The idea of the
proof is that given a right-regular grammar, we can build
a corresponding $NFA$ and \textit{vice-versa}.  The
states of the $NFA$ correspond to the non-terminal symbols
of the grammar.  The start symbol of the grammar corresponds
to the starting state of the NFA.
A production rule of the form $A\PRODUCES bC$
corresponds to a transition in the NFA from state $A$ to state
$C$ while reading the symbol $b$.  A production rule of
the form $A\PRODUCES B$ corresponds to an $\EMPTYSTRING$-transition
from state $A$ to state $B$ in the NFA.  And a production
rule of the form $A\PRODUCES\EMPTYSTRING$ exists in the grammar
if and only if $A$ is a final state in the NFA.  With this
correspondence, a derivation of a string $w$ in the grammar
corresponds to an execution path through the NFA as it 
accepts the string $w$.  I won't give a complete proof
here.  You are welcome to work through the details if you want.
But the important fact is:

\begin{theorem}\label{T-reggram}
A language $L$ is regular if and only if there is a right-regular
grammar $G$ such that $L=L(G)$.  In particular, every regular
language is context-free.
\end{theorem}


\begin{exercises}

\problem Show that Part 4 of Theorem \ref{T-yields} follows from Part 3.

\problem Give a careful proof that the language $\{a^nb^m\st n\ge m\ge 0\}$
is generated by the context-free grammar
\begin{align*}
      S&\PRODUCES aSb\\
      S&\PRODUCES A\\
      A&\PRODUCES aA\\
      A&\PRODUCES \EMPTYSTRING
\end{align*}

\problem Identify the language generated by each of the following
context-free grammars.
\smallskip
\tparts{
   \vtop{\halign{$#$\hfil\cr
      S\PRODUCES aaSb\cr
      S\PRODUCES \EMPTYSTRING\cr
   }}
\qquad&
   \vtop{\halign{$#$\hfil\cr
      S\PRODUCES aSb\cr
      S\PRODUCES aaSb\cr
      S\PRODUCES \EMPTYSTRING\cr
   }}
\cr\noalign{\medskip}
   \vtop{\halign{$#$\hfil\cr
      S\PRODUCES TS\cr
      S\PRODUCES \EMPTYSTRING\cr
      T\PRODUCES aTb\cr
      T\PRODUCES \EMPTYSTRING\cr
   }}
\qquad&
   \vtop{\halign{$#$\hfil\cr
      S\PRODUCES ABA\cr
      A\PRODUCES aA\cr
      A\PRODUCES a\cr
      B\PRODUCES bB\cr
      B\PRODUCES cB\cr
      B\PRODUCES\EMPTYSTRING\cr
   }}
}
\smallskip

\problem For each of the following languages
find a context-free grammar that generates the language:
\pparts{
   \{a^nb^m\st n\ge m > 0\} &   \{a^nb^m\st n, m\in\N \}\cr
   \{a^nb^m\st n\ge0\AND m=n+1\} & \{a^nb^mc^n\st n,m\in\N \} \cr
   \{ a^nb^mc^k \st n=m+k \} & \{a^nb^m\st n\not=m\} \cr
   \{ a^nb^mc^rd^t\st n+m=r+t \} & \{ a^nb^mc^k \st n\not=m+k \} \cr
}

\problem Find a context-free grammar that generates the language
$\{w\in\{a,b\}^*\st n_a(w) > n_b(w)\}$.

\problem Find a context-free grammar that generates the language
$\{w\in\{a,b,c\}^*\st n_a(w) = n_b(w)\}$.

\problem A \nw{palindrome} is a string that reads the same
backwards and forwards, such as ``mom'', ``radar'', or
``aabccbccbaa''.  That is, $w$ is a palindrome if $w=w^R$.
Let $L=\{w\in\{a,b,c\}^*\st$ $w$ is a palindrome~$\}$.
Show that $L$ is a context-free language by finding a context-free
grammar that generates $L$.

\problem Let $\Sigma=\{\,\texttt{(},\,\texttt{)},\,\texttt{[},\,\texttt{]}\,\}$.  That is, $\Sigma$
is the alphabet consisting of the four symbols \texttt{(}, \texttt{)}, \texttt{[}, and~\texttt{]}.
Let $L$ be the language over $\Sigma$ consisting of strings
in which both parentheses and brackets are balanced.
For example, the string \texttt{([][()()])([])} is in $L$
but \texttt{[(])} is not.  Find a context-free grammar that generates the
language $L$.

\problem Suppose that $G$ and $H$ are context-free grammars.
Let $L=L(G)$ and let $M=L(H)$.  Explain how to construct
a context-free grammar for the language $LM$.  You do not
need to give a formal proof that your grammar is correct.

\problem Suppose that $G$ is a context-free grammar.
Let $L=L(G)$.  Explain how to construct
a context-free grammar for the language $L^*$.  You do not
need to give a formal proof that your grammar is correct.

\problem Suppose that $L$ is a context-free language.
Prove that $L^R$ is a context-free language.  (Hint:
Given a context-free grammar $G$ for $L$, make a new grammar, $G^R$,
by reversing the right-hand side of each of the production
rules in $G$.  That is, $A\PRODUCES w$ is a production rule in
$G$ if and only if $A\PRODUCES w^R$ is a production rule in $G^R$.)

\problem Define a \nw{left-regular grammar}
to be a context-free grammar in which the right-hand side of
every production rule is of one of the following forms:
the empty string; a single non-terminal symbol; or a non-terminal
symbol followed by a terminal symbol.  Show that a language is
regular if and only if it can be generated by a left-regular
grammar.  (Hint: Use the preceding exercise and Theorem~\ref{T-reggram}.)




\end{exercises}


\section{Application: BNF}\label{S-grammars-2}

Context-free grammars are used to describe some aspects of
the syntax of programming languages.  However, the notation
that is used for grammars in the context of programming languages
is somewhat different from the notation introduced in the
preceding section.  The notation that is used is called
\nw{Backus-Naur Form} or BNF.  It is named after computer
scientists John Backus and Peter Naur, who developed the
notation.  Actually, several variations of BNF exist.
I will discuss one of them here.  BNF can be used to describe
the syntax of natural languages, as well as programming languages,
and some of the examples in this section will deal with the
syntax of English.

Like context-free grammars, BNF grammars make use of production rules, non-terminals,
and terminals.  The non-terminals are usually given meaningful,
multi-character names.  Here, I will follow a common practice
of enclosing non-terminals in angle brackets, so that they can
be easily distinguished.  For example, \NT{noun} and \NT{sentence}
could be non-terminals in a BNF grammar for English, while
\NT{program} and \NT{if-statement} might be used in a BNF grammar
for a programming language.  Note that a BNF non-terminal
usually represents a meaningful \nw{syntactic category},
that is, a certain type of building block in the syntax of
the language that is being described, such as an adverb,
a prepositional phrase, or a variable declaration statement.
The terminals of a BNF grammar are the things that actually
appear in the language that is being described.  In the case
of natural language, the terminals are individual words.

In BNF production rules, I will use the symbol ``\BNFPRODUCES''
in place of the ``$\PRODUCES$'' that is used in context-free grammars.
BNF production rules are more powerful than the production rules in
context-free grammars.  That is, one BNF rule might be equivalent to 
several context-free grammar rules.  As for context-free grammars,
the left-hand side of a BNF production rule is a single 
non-terminal symbol.  The right hand side can include terminals
and non-terminals, and can also use the following notations,
which should remind you of notations used in regular expressions:
\smallskip
\IItem{$\bullet\,\,$}A vertical bar, \BNFALT, indicates a choice of
   alternatives.  For example,
   
\smallskip   
\centerline{\NT{digit} \BNFPRODUCES\ 0 \BNFALT\ 1 \BNFALT\ 2
          \BNFALT\ 3 \BNFALT\ 4 \BNFALT\ 5 \BNFALT\ 6 \BNFALT\ 7
          \BNFALT\ 8 \BNFALT\ 9}
          
\smallskip

\IItem{}indicates that the non-terminal \NT{digit} can be replaced
by any one of the terminal symbols 0, 1, \dots,~9.
\smallskip
\IItem{$\bullet\,\,$}Items enclosed in brackets are optional.  For example,

\smallskip
\centerline{\NT{declaration} \BNFPRODUCES\ \NT{type} \NT{variable}
                 [ = \NT{expression} ] ;}

\smallskip
\IItem{}says that \NT{declaration} can be replaced either
by ``\NT{type} \NT{variable}~;'' or by ``\NT{type} \NT{variable}
= \NT{expression}~;''.
(The symbols ``='' and ``;'' are terminal symbols in this rule.)

\smallskip
\IItem{$\bullet\,\,$}Items enclosed between ``['' and ``]\dots''
can be repeated zero or more times.  (This has the same effect
as a ``$*$''in a regular expression.)  For example,

\centerline{\NT{integer} \BNFPRODUCES\ \NT{digit} [ \NT{digit} ]\dots}
\smallskip

\IItem{}says that an \NT{integer} consists of a \NT{digit} followed
optionally by any number of additional \NT{digit}'s.  That is,
the non-terminal \NT{integer} can be replaced by \NT{digit} or
by \NT{digit}\NT{digit} or by \NT{digit}\NT{digit}\NT{digit}, and
so on.

\smallskip
\IItem{$\bullet\,\,$}Parentheses can be used as usual, for grouping. 
\smallskip

All these notations can be expressed in a context-free grammar
by introducing additional production rules.  For example, the
BNF rule ``\NT{sign}~\BNFPRODUCES\ +~\BNFALT~$-$'' is equivalent
to the two rules, ``\NT{sign}~\BNFPRODUCES~+''
and ``\NT{sign}~\BNFPRODUCES~$-$''.  A rule that contains an
optional item can also be replaced by two rules.  For example,

\smallskip
\centerline{\NT{declaration} \BNFPRODUCES\ \NT{type} \NT{variable}
                 [ = \NT{expression} ] ;}

\smallskip
\noindent can be replaced by the two rules

\smallskip
\centerline{\vbox{\halign{#\hfil\cr
         \NT{declaration} \BNFPRODUCES\ \NT{type} \NT{variable} ;\cr
         \NT{declaration} \BNFPRODUCES\ \NT{type} \NT{variable}
                  = \NT{expression}  ;\cr}}}
                  
\smallskip
\noindent In context-free grammars, repetition can be expressed by
using a recursive rule such as ``$S\PRODUCES aS$'', in which the
same non-terminal symbol appears both on the left-hand side and on the right-hand
side of the rule.  BNF-style notation using ``['' and ``]\dots'' can
be eliminated by replacing it with a new non-terminal symbol and adding
a recursive rule to allow that symbol to repeat zero or more times.
For example, the production rule

\smallskip
\centerline{\NT{integer} \BNFPRODUCES\ \NT{digit} [ \NT{digit} ]\dots}

\smallskip
\noindent can be replaced by three rules using a new non-terminal symbol
\NT{digit-list} to represent a string of zero or more \NT{digit}'s:

\smallskip
\centerline{\vbox{\halign{#\hfil\cr
   \NT{integer} \BNFPRODUCES\ \NT{digit} \NT{digit-list}\cr
   \NT{digit-list} \BNFPRODUCES\ \NT{digit} \NT{digit-list}\cr
   \NT{digit-list} \BNFPRODUCES\ $\EMPTYSTRING$\cr}}}

\medbreak

As an example of a complete BNF grammar, let's look at a BNF grammar for a very small
subset of English.  The start symbol for the grammar
is \NT{sentence}, and the terminal symbols are English words.
All the sentences that can be produced from this grammar
are syntactically correct English sentences, although you wouldn't
encounter many of them in conversation.  Here is the grammar:

\smallskip

\ \NT{sentence} \BNFPRODUCES\ \NT{simple-sentence} [ and \NT{simple-sentence} ]\dots
\smallskip

\ \NT{simple-sentence} \BNFPRODUCES\ \NT{nout-part} \NT{verb-part}
\smallskip

\ \NT{noun-part} \BNFPRODUCES\ \NT{article} \NT{noun} [ who \NT{verb-part} ]\dots
\smallskip

\ \NT{verb-part} \BNFPRODUCES\ \NT{intransitive-verb} \BNFALT\ ( \NT{transitive-verb} \NT{noun-part} )
\smallskip

\ \NT{article} \BNFPRODUCES\ the \BNFALT\ a
\smallskip

\ \NT{noun} \BNFPRODUCES\ man \BNFALT\ woman \BNFALT\ dog  \BNFALT\ cat  \BNFALT\ computer
\smallskip

\ \NT{intransitive-verb} \BNFPRODUCES\ runs \BNFALT\ jumps \BNFALT\ hides
\smallskip

\ \NT{transitive-verb} \BNFPRODUCES\ knows \BNFALT\ loves \BNFALT\ chases  \BNFALT\ owns

\smallskip

\noindent This grammar can generate sentences such as ``A dog chases the cat and
the cat hides'' and ``The man loves a woman who runs.''
The second sentence, for example, is generated by the derivation
\begin{align*}
    \NT{sentence}\ &\YIELDS\ \NT{simple-sentence}\\
       &\YIELDS\ \NT{noun-part}\ \NT{verb-part}\\
       &\YIELDS\ \NT{article}\ \NT{noun}\ \NT{verb-part}\\
       &\YIELDS\ \mbox{the}\ \NT{noun}\ \NT{verb-part}\\
       &\YIELDS\ \mbox{the}\ \mbox{man}\ \NT{verb-part}\\
       &\YIELDS\ \mbox{the}\ \mbox{man}\ \NT{transitive-verb}\ \NT{noun-part}\\
       &\YIELDS\ \mbox{the}\ \mbox{man}\ \mbox{loves}\ \NT{noun-part}\\
       &\YIELDS\ \mbox{the}\ \mbox{man}\ \mbox{loves}\ \NT{article}
              \ \NT{noun}\ \mbox{who}\ \NT{verb-part}\\
       &\YIELDS\ \mbox{the}\ \mbox{man}\ \mbox{loves}\ \mbox{a}
              \ \NT{noun}\ \mbox{who}\ \NT{verb-part}\\
       &\YIELDS\ \mbox{the}\ \mbox{man}\ \mbox{loves}\ \mbox{a}
              \ \mbox{woman}\ \mbox{who}\ \NT{verb-part}\\
       &\YIELDS\ \mbox{the}\ \mbox{man}\ \mbox{loves}\ \mbox{a}
              \ \mbox{woman}\ \mbox{who}\ \NT{intransitive-verb}\\
       &\YIELDS\ \mbox{the}\ \mbox{man}\ \mbox{loves}\ \mbox{a}
              \ \mbox{woman}\ \mbox{who}\ \mbox{runs}
\end{align*}

\medskip

BNF is most often used to specify the syntax of programming languages.
Most programming languages are not, in fact, context-free languages, and
BNF is not capable of expressing all aspects of their syntax.
For example, BNF cannot express the fact that a variable must
be declared before it is used or the fact that the number of
actual parameters in a subroutine call statement must match the number
of formal parameters in the declaration of the subroutine.
So BNF is used to express the context-free aspects of the syntax
of a programming language, and other restrictions on the syntax---such
as the rule about declaring a variable before it is used---are expressed
using informal English descriptions.

When BNF is applied to programming languages, the terminal symbols
are generally ``tokens,'' which are the minimal meaningful units in
a program.  For example, the pair of symbols \verb?<=? constitute a
single token, as does a string such as \texttt{"Hello World"}.
Every number is represented by a single token.  (The actual value
of the number is stored as a so-called ``attribute'' of the token,
but the value plays no role in the context-free syntax of the
language.) I will use the
symbol \textbf{\textsl{number}} to represent a numerical token.
Similarly, every variable name, subroutine name, or other identifier
in the program is represented by the same token, which I will denote
as \textbf{\textsl{ident}}.  One final complication:  Some symbols
used in programs, such as ``]'' and ``('', are also used with
a special meaning in BNF grammars.  When such a symbol occurs as
a terminal symbol, I will enclose it in double quotes.  For
example, in the BNF production rule

\smallskip
\centerline{ \NT{array-reference} \BNFPRODUCES\ 
   \textbf{\textsl{ident}} ``['' \NT{expression} ``]'' }

\smallskip
\noindent the ``['' and ``]'' are terminal symbols in the language
that is being described, rather than the BNF notation for an
optional item.  With this notation, here is part of a BNF
grammar that describes statements in the Java programming
language:

\smallskip

\ \NT{statement} \BNFPRODUCES\ \NT{block-statement} \BNFALT\ \NT{if-statement}
                                    \BNFALT\ \NT{while-statement}
 
\ \hbox to 1.5 in{} \BNFALT\ \NT{assignment-statement} \BNFALT\ \NT{null-statement}
\smallskip

\ \NT{block-statement} \BNFPRODUCES\ $\{$ [ \NT{statement} ]\dots\ $\}$
\smallskip

\ \NT{if-statement} \BNFPRODUCES\ if ``('' \NT{condition} ``)'' \NT{statement} [ else \NT{statement}~]
\smallskip

\ \NT{while-statement} \BNFPRODUCES\ while ``('' \NT{condition} ``)'' \NT{statement}
\smallskip

\ \NT{assignment-statement} \BNFPRODUCES\ \NT{variable} = \NT{expression} ;
\smallskip

\ \NT{null-statement} \BNFPRODUCES\ $\EMPTYSTRING$

\smallskip

\noindent The non-terminals \NT{condition}, \NT{variable}, and 
\NT{expression} would, of course, have to be defined by other
production rules in the grammar.  Here is a set of rules that
define simple expressions, made up of numbers, identifiers,
parentheses and the arithmetic operators +, $-$, $*$ and~/:

\smallskip

\ \NT{expression} \BNFPRODUCES\ \NT{term} [ [ + \BNFALT\ $-$ ] \NT{term} ]\dots
\smallskip

\ \NT{term} \BNFPRODUCES\ \NT{factor} [ [ $*$ \BNFALT\ / ] \NT{factor} ]\dots
\smallskip

\ \NT{factor} \BNFPRODUCES\ \textbf{\textsl{ident}} \BNFALT\ \textbf{\textsl{number}} \BNFALT\ ``('' \NT{expression} ``)''
\smallskip

\noindent The first rule says that an \NT{expression} is a sequence of
one or more \NT{term}'s, separated by plus or minus signs.
The second rule defines a \NT{term} to be a sequence of one or more
\NT{factors}, separated by multiplication or division operators.
The last rule says that a \NT{factor} can be either an identifier
or a number or an \NT{expression} enclosed in parentheses.
This small BNF grammar can generate expressions 
such as ``$3*5$'' and ``$x*(x+1) - 3/(z+2*(3-x)) + 7$''.
The latter expression is made up of three terms: $x*(x+1)$,
$3/(z+2*(3-x))$, and~$7$.  The first of these terms is made
up of two factors, $x$ and $(x+1)$.  The factor $(x+1)$ consists
of the expression $x+1$ inside a pair of parentheses.

The nice thing about this grammar is that the precedence rules
for the operators are implicit in the grammar.  For example, according
to the grammar, the expression $3+5*7$ is seen as \NT{term} + \NT{term}
where the first term is $3$ and the second term is $5*7$.
The $5*7$ occurs as a group, which must be evaluated before the
result is added to $3$.  Parentheses can change the order of
evaluation.  For example, $(3+5)*7$ is generated by the grammar
as a single \NT{term} of the form $\NT{factor}*\NT{factor}$.
The first \NT{factor} is $(3+5)$.  When $(3+5)*7$ is evaluated,
the value of $(3+5)$ is computed first and then multiplied
by~$7$.  This is an example of how a grammar that describes
the syntax of a language can also reflect its meaning.

\medskip

Although this section has not introduced any really new ideas
or theoretical results, I hope it has demonstrated how 
context-free grammars can be applied in practice.  

\begin{exercises}

\problem One of the examples in this section was a grammar for
a subset of English.  Give five more examples of sentences that
can be generated from that grammar.  Your examples should, collectively,
use all the rules of the grammar.

\problem Rewrite the example BNF grammar for a subset of English as
a context-free grammar.

\problem Write a single BNF production rule that is equivalent to
the following context-free grammar:
\begin{align*}
    S &\PRODUCES aSa\\
    S &\PRODUCES bB\\
    B &\PRODUCES bB\\
    B &\PRODUCES \EMPTYSTRING
\end{align*}

\problem Write a BNF production rule that specifies the syntax of
real numbers, as they appear in programming languages such as Java and C.  
Real numbers can include a sign, a decimal point and an exponential part.
Some examples are:  17.3, .73, 23.1e67, $-$1.34E$-$12, +0.2, 100E+100

\problem Variable references in the Java programming language can be 
rather complicated.  Some examples include:
$x$, $list.next$, $A[7]$, $a.b.c$, $S[i+1].grid[r][c].red$, \dots.
Write a BNF production rule for Java variables.
You can use the token \textbf{\textsl{ident}} and the non-terminal
\NT{expression} in your rule.
\
\problem Use BNF to express the syntax of the try\dots catch statement in the
Java programming language.

\problem Give a BNF grammar for compound propositions made up
of propositional variables, parentheses, and the logical operators
$\AND$, $\OR$, and $\NOT$.  Use the non-terminal symbol \NT{pv} to represent
a propositional variable.  You do not have to give a definition of
\NT{pv}.

\end{exercises}


\section{Parsing and Parse Trees}\label{S-grammars-3}

Suppose that $G$ is a grammar for the language $L$.  That is, 
$L=L(G)$.  The grammar $G$ can be used to generate strings in
the language~$L$.  In practice, though, we often start with a string
which might or might not be in~$L$, and the problem is
to determine whether the string is in the language and, if so,
how it can be generated by~$G$.  The goal is to find a derivation
of the string, using the production rules of the grammar, or to
show that no such derivation exists.  This is known as \nw{parsing}
the string.  When the string is a computer program or a sentence
in a natural language, parsing the string is an essential step
in determining its meaning.

As an example that we will use throughout
this section, consider the language that consists of arithmetic
expressions containing parentheses, the binary operators $+$ and $*$,
and the variables $x$, $y$, and $z$.  Strings in this language
include $x$, $x+y*z$, and $((x+y)*y)+z*z$.  Here is a context-free
grammar that generates this language:
\begin{align*}
   E&\PRODUCES E+E\\
   E&\PRODUCES E*E\\
   E&\PRODUCES (E)\\
   E&\PRODUCES x\\
   E&\PRODUCES y\\
   E&\PRODUCES z
\end{align*}
Call the grammar described by these production rules $G_1$.
The grammar $G_1$ says that $x$, $y$, and $z$ are expressions, and that
you can make new expressions by adding two expressions, by multiplying
two expressions, and by enclosing an expression in parentheses.
(Later, we'll look at other grammars for the same language---ones that
turn out to have certain advantages over $G_1$.)

Consider the string $x+y*z$.  To show that this string is in the
language $L(G_1)$, we can exhibit a derivation of the string
from the start symbol $E$.  For example:
\begin{align*}
   E &\YIELDS E+E\\
     &\YIELDS E+E*E\\
     &\YIELDS E+y*E\\
     &\YIELDS x+y*E\\
     &\YIELDS x+y*z
\end{align*}
This derivation shows that the string $x+y*z$ is in fact in $L(G_1)$.
Now, this string has many other derivations.  At each step in the
derivation, there can be a lot of freedom about which rule in the
grammar to apply next.  Some of this freedom is clearly not very
meaningful.  When faced with the string $E+E*E$ in the above example,
the order in which we replace the $E\text{'}s$ with the variables $x$, $y$,
and $z$ doesn't much matter.  To cut out some of this meaningless
freedom, we could agree that in each step of a derivation, the
non-terminal symbol that is replaced is the leftmost non-terminal
symbol in the string.  A derivation in which this is true is
called a \nw{left derivation}.  The following left derivation
of the string $x+y*z$ uses the same production rules as the previous
derivation, but it applies them in a different order:
\begin{align*}
   E &\YIELDS E+E\\
     &\YIELDS x+E\\
     &\YIELDS x+E*E\\
     &\YIELDS x+y*E\\
     &\YIELDS x+y*z
\end{align*}
It shouldn't be too hard to convince yourself that any string that
has a derivation has a left derivation (which can be obtained
by changing the order in which production rules are applied).

We have seen that the same string might have several different derivations.
We might ask whether it can have several different left derivations.
The answer is that it depends on the grammar.  A context-free
grammar $G$ is said to be \nw[ambiguous grammar]{ambiguous}\index{grammar!ambiguous}
if there is a string $w\in L(G)$ such that $w$ has more than one
left derivation according to the grammar $G$.

Our example grammar $G_1$ is ambiguous.  In fact, in addition to the
left derivation given above, the string $x+y*z$ has the alternative
left derivation
\begin{align*}
   E &\YIELDS E*E\\
     &\YIELDS E+E*E\\
     &\YIELDS x+E*E\\
     &\YIELDS x+y*E\\
     &\YIELDS x+y*z
\end{align*}
In this left derivation of the string $x+y*z$, the first production
rule that is applied is $E\PRODUCES E*E$.  The first $E$ on the right-hand
side eventually yields ``$x+y$'' while the second yields ``$z$''.
In the previous left derivation, the first production rule that was
applied was $E\PRODUCES E+E$, with the first $E$ on the right yielding
``$x$'' and the second $E$ yielding ``$y*z$''.  If we think in terms
of arithmetic expressions, the two left derivations lead to
two different interpretations of the expression $x+y*z$.  In one
interpretation, the $x+y$ is a unit that is multiplied by $z$.
In the second interpretation, the $y*z$ is a unit that is added to $x$.
The second interpretation is the one that is correct according to
the usual rules of arithmetic.  However, the grammar allows either
interpretation.  The ambiguity of the grammar allows the string to
be parsed in two essentially different ways, and only one of the
parsings is consistent with the meaning of the string.  Of course,
the grammar for English is also ambiguous.  In a famous example,
it's impossible to tell whether a ``pretty girls' camp'' is
meant to describe a pretty camp for girls or a camp for pretty girls.

When dealing with artificial languages such as programming languages,
it's better to avoid ambiguity.
The grammar $G_1$ is perfectly correct in that it generates the correct
set of strings, but in a practical situation where we are interested
in the meaning of the strings, $G_1$ is not the right grammar for
the job.  There are other grammars that generate the same language
as $G_1$.  Some of them are unambiguous grammars that better reflect
the meaning of the strings in the language.  For example, the
language $L(G_1)$ is also generated by the BNF grammar
\begin{align*}
   E\ &\BNFPRODUCES\ T\ [\ +\ T\ ]\dots\\
   T\ &\BNFPRODUCES\ F\ [\ *\ F\ ]\dots\\
   F\ &\BNFPRODUCES\ \text{``(''}\ E\ \text{``)''}\ \BNFALT\ x\ \BNFALT\ y\ \BNFALT\ z
\end{align*} 
This grammar can be translated into a standard context-free grammar, which
I will call $G_2$:
\begin{align*}
   E &\PRODUCES TA\\
   A &\PRODUCES +TA\\
   A &\PRODUCES \EMPTYSTRING\\
   T &\PRODUCES FB\\
   B &\PRODUCES *FB\\
   B &\PRODUCES \EMPTYSTRING\\
   F &\PRODUCES (E)\\
   F &\PRODUCES x\\
   F &\PRODUCES y\\
   F &\PRODUCES z
\end{align*}
The language generated by
$G_2$ consists of all legal arithmetic expressions made up of 
parentheses, the operators $+$ and $-$, and the variables $x$, $y$,
and $z$.  That is, $L(G_2)=L(G_1)$.  However, $G_2$ is an unambiguous
grammar.  Consider, for example, the string $x+y*z$.  Using the
grammar $G_2$, the only left derivation for this string is:
\begin{align*}
   E &\YIELDS TA\\
     &\YIELDS FBA\\
     &\YIELDS xBA\\
     &\YIELDS xA\\
     &\YIELDS x+TA\\
     &\YIELDS x+FBA\\
     &\YIELDS x+yBA\\
     &\YIELDS x+y*FBA\\
     &\YIELDS x+y*zBA\\
     &\YIELDS x+y*zA\\
     &\YIELDS x+y*z
\end{align*}
There is no choice about the first step in this derivation, since the
only production rule with $E$ on the left-hand side is $E\PRODUCES TA$.
Similarly, the second step is forced by the fact that there is only
one rule for rewriting a $T$.  In the third step, we must replace
an $F$.  There are four ways to rewrite $F$, but only one way to produce
the $x$ that begins the string $x+y*z$, so we apply the rule $F\PRODUCES x$.
Now, we have to decide what to do with the $B$ in $xBA$.  There two rules
for rewriting $B$, $B\PRODUCES *FB$ and $B\PRODUCES\EMPTYSTRING$.  However,
the first of these rules introduces a non-terminal, $*$, which does not
match the string we are trying to parse.  So, the only choice is to
apply the production rule $B\PRODUCES\EMPTYSTRING$.  In the next step
of the derivation, we must apply the rule $A\PRODUCES +TA$ in order to 
account for the $+$ in the string $x+y*z$.  Similarly, each of the 
remaining steps in the left derivation is forced.

\medbreak

The fact that $G_2$ is an unambiguous grammar means that at each
step in a left derivation for a string $w$, there is only one production
rule that can be applied which will lead ultimately to a correct
derivation of~$w$.  However, $G_2$ actually satisfies a much stronger
property:  at each step in the left derivation of $w$, we can tell which
production rule has to be applied by looking ahead at the next
symbol in~$w$.  We say that $G_2$ is an \nw{LL(1) grammar}.
(This notation means that we can read a string from \textbf{L}eft to
right and construct a \textbf{L}eft derivation of the string by
looking ahead at most \textbf{1} character in the string.)
Given an LL(1) grammar for a language, it is fairly straightforward
to write a computer program that can parse strings in that language.
If the language is a programming language, then parsing is one of the
essential steps in translating a computer program into machine language.
LL(1) grammars and parsing programs that use them are often studied
in courses in programming languages and the theory of compilers.

Not every unambiguous context-free grammar is an LL(1) grammar.  Consider, for
example, the following grammar, which I will call $G_3$:
\begin{align*}
   E &\PRODUCES E + T\\
   E &\PRODUCES T\\
   T &\PRODUCES T*F\\
   T &\PRODUCES F\\
   F &\PRODUCES (E)\\
   F &\PRODUCES x\\
   F &\PRODUCES y\\
   F &\PRODUCES z
\end{align*}
This grammar generates the same language as $G_1$ and $G_2$,
and it is unambiguous.  However, it is not possible to construct
a left derivation for a string according to the grammar $G_3$ by
looking ahead one character in the string at each step.  
The first step in any left derivation must be either
$E\YIELDS E+T$ or $E\YIELDS T$.  But how can we decide which of
these is the correct first step?
Consider the strings $(x+y)*z$ and $(x+y)*z+z*x$, which are both
in the language $L(G_3)$.  For the string $(x+y)*z$, the
first step in a left derivation must be $E\YIELDS T$, while 
the first step in a left derivation of $(x+y)*z+z*x$ must be
$E\YIELDS E+T$.  However, the first seven characters of the strings
are identical, so clearly looking even seven characters ahead is not
enough to tell us which production rule to apply.  In fact,
similar examples show that looking ahead any given finite number of
characters is not enough.

However, there is an alternative parsing procedure that will work
for $G_3$.  This alternative method of parsing a string produces
a \nw{right derivation} of the string, that is, a derivation in
which at each step, the non-terminal symbol that is replaced is
the rightmost non-terminal symbol in the string.  Here, for example,
is a right derivation of the string $(x+y)*z$ according to the
grammar $G_3$:
\begin{align*}
  E &\YIELDS T\\
    &\YIELDS T*F\\
    &\YIELDS T*z\\
    &\YIELDS F*z\\
    &\YIELDS (E)*z\\
    &\YIELDS (E+T)*z\\
    &\YIELDS (E+F)*z\\
    &\YIELDS (E+y)*z\\
    &\YIELDS (T+y)*z\\
    &\YIELDS (F+y)*z\\
    &\YIELDS (x+y)*z
\end{align*}
The parsing method that produces this right derivation produces
it from ``bottom to top.''  That is, it begins with
the string $(x+y)*z$ and works backward to the start symbol $E$,
generating the steps of the right derivation in reverse order.
The method works because $G_3$ is what is called an
\nw{LR(1) grammar}.  That is, roughly, it is possible to read
a string from \textbf{L}eft to right and produce a \textbf{R}ight
derivation of the string, by looking ahead at most \textbf{1} symbol at
each step.  Although LL(1) grammars are easier for people to work
with, LR(1) grammars turn out to be very suitable for machine
processing, and they are used as the basis for the parsing
process in many compilers.

LR(1) parsing uses a \nw{shift/reduce} algorithm.  Imagine a
cursor or current position that moves through the string that
is being parsed.  We can visualize the cursor as a vertical
bar, so for the string $(x+y)*z$, we start with the
configuration $|(x+y)*z$.  A \textit{shift} operation simply
moves the cursor one symbol to the right.  For example,
a shift operation would convert $|(x+y)*z$ to $(|x+y)*z$,
and a second shift operation would convert that to
$(x|+y)*z$.  In a reduce
operation, one or more symbols immediately to the left of
the cursor are recognized as the right-hand side of one of
the production rules in the grammar.  These symbols are removed
and replaced by the left-hand side of the production rule.
For example, in the configuration $(x|+y)*z$, the $x$ to the left
of the cursor is the right-hand side of the production rule
$F\PRODUCES x$, so we can apply a reduce operation and replace
the $x$ with $F$, giving $(F|+y)*z$.  This first reduce operation
corresponds to the last step in the right derivation of the
string, $(F+y)*z\YIELDS (x+y)*z$.  Now the $F$ can be recognized
as the right-hand side of the production rule $T\PRODUCES F$,
so we can replace the $F$ with $T$, giving $(T|+y)*z$.
This corresponds to the next-to-last step in the right
derivation, $(T+y)*z\YIELDS (F+y)*z$.

At this point, we have the configuration $(T|+y)*z$.  The $T$
could be the right-hand side of the production rule $E\PRODUCES T$.
However, it could also conceivably come from the rule $T\PRODUCES T*F$.
How do we know whether to reduce the $T$ to $E$ at this point or to
wait for a $*F$ to come along so that we can reduce $T*F\,$?
We can decide by looking ahead at the next character after the
cursor.  Since this character is a $+$ rather than a $*$,
we should choose the reduce operation that replaces $T$ with $E$,
giving $(E|+y)*z$.  What makes $G_3$ an LR(1) grammar is the fact
that we can always decide what operation to apply by looking
ahead at most one symbol past the cursor.

After a few more shift and reduce operations, the configuration
becomes $(E)|*z$, which we can reduce to $T|*z$ by applying the
production rules $F\PRODUCES (E)$ and $T\PRODUCES F$.
Now, faced with $T|*z$, we must once again decide between
a shift operation and a reduce operation that applies the
rule $E\PRODUCES T$.  In this case, since the next character is
a $*$ rather than a $+$, we apply the shift operation, giving
$T*|z$.  From there we get, in succession, $T*z|$,
$T*F|$, $T|$, and finally $E|$.  At this point, we have reduced
the entire string $(x+y)*z$ to the start symbol of the grammar.
The very last step, the reduction of $T$ to $E$ corresponds to
the first step of the right derivation, $E\YIELDS T$.

In summary, LR(1) parsing transforms a string into the
start symbol of the grammar by a sequence of shift and
reduce operations.  Each reduce operation corresponds to a
step in a right derivation of the string, and these steps
are generated in reverse order.  Because the steps in the
derivation are generated from ``bottom to top,'' LR(1)
parsing is a type of \nw{bottom-up parsing}.  LL(1) parsing,
on the other hand, generates the steps in a left derivation
from ``top to bottom'' and so is a type of \nw{top-down parsing}.

\medbreak

Although the language generated by a context-free grammar
is defined in terms of derivations, there is another way of
presenting the generation of a string that is often more useful.
A \nw{parse tree} displays the generation of a string from
the start symbol of a grammar as a two dimensional diagram.
Here are two parse trees that show two derivations of the
string x+y*z according to the grammar $G_1$, which was given
at the beginning of this section:
\bigskip
\centerline{\scaledeps{2.5truein}{fig-5-1}}

\noindent A parse tree is made up of terminal and non-terminal symbols,
connected by lines.  The start symbol is at the top, or ``root,'' of
the tree.  Terminal symbols are at the lowest level, or ``leaves,'' of
the tree.  (For some reason, computer scientists traditionally
draw trees with leaves at the bottom and root at the top.)
A production rule $A\PRODUCES w$ is represented
in a parse tree by the symbol $A$ lying above all the symbols in $w$,
with a line joining $A$ to each of the symbols in $w$.  For
example, in the left parse tree above, the root,
$E$, is connected to the symbols $E$, $+$, and $E$, and this
corresponds to an application of the production rule
$E\PRODUCES E+E$.

It is customary to draw a parse tree with the string of non-terminals
in a row across the bottom, and with the rest of the tree built on
top of that base.  Thus, the two parse trees shown above might
be drawn as:

\bigskip
\centerline{\scaledeps{2.5truein}{fig-5-2}}

Given any derivation of a string, it is possible to construct
a parse tree that shows each of the steps in that derivation.
However, two different derivations can give rise to the same
parse tree, since the parse tree does not show the order in
which production rules are applied.  For example, the parse
tree on the left, above, does not show whether the production
rule $E\PRODUCES x$ is applied before or after the production
rule $E\PRODUCES y$.  However, if we restrict our attention to left
derivations, then we find that each parse tree corresponds to
a unique left derivation and \textit{vice versa}.  I will state this
fact as a theorem, without proof.  A similar result holds for
right derivations.

\begin{theorem}
Let $G$ be a context-free grammar.  There is a one-to-one correspondence
between parse trees and left derivations based on the grammar $G$.
\end{theorem}

Based on this theorem, we can say that a context-free grammar $G$
is ambiguous if and only if there is a string $w\in L(G)$ which has
two parse trees.



\begin{exercises}

\problem Show that each of the following grammars is ambiguous by finding
a string that has two left derivations according to the grammar:
\tparts{
   \vtop{\halign{$#$\hfil\cr
      S\PRODUCES SS\cr
      S\PRODUCES aSb\cr
      S\PRODUCES bSa\cr
      S\PRODUCES\EMPTYSTRING\cr
   }}\quad&
   \vtop{\halign{$#$\hfil\cr
      S\PRODUCES ASb\cr
      S\PRODUCES \EMPTYSTRING\cr
      A\PRODUCES aA\cr
      A\PRODUCES a\cr
   }}\cr
}

\problem Consider the string $z+(x+y)*x$.  Find a left derivation
of this string according to each of the grammars $G_1$, $G_2$, and
$G_3$, as given in this section.

\problem Draw a parse tree for the string $(x+y)*z*x$ according to
each of the grammars $G_1$, $G_2$, and $G_3$, as given in this section.

\problem Draw three different parse trees for the string
$ababbaab$ based on the grammar given in part a) of exercise 1.

\problem Suppose that the string $abbcabac$ has the following parse
tree, according to some grammar $G$:

\centerline{\scaledeps{2in}{fig-5-3}}
\medskip

\ppart List five production rules that must be rules in the grammar $G$,
given that this is a valid parse tree.
\ppart Give a left derivation for the string $abbcabac$ according to the
grammar $G$.
\ppart Give a right derivation for the string $abbcabac$ according to the
grammar $G$.

\problem Show the full sequence of shift and reduce operations
that are used in the LR(1) parsing of the string $x+(y)*z$ according
to the grammar $G_3$, and give the corresponding right derivation
of the string.

\problem This section showed how to use LL(1) and LR(1) parsing to
find a derivation of a string in the language $L(G)$ generated by
some grammar $G$.  How is it possible to use LL(1) or LR(1) parsing
to determine for an arbitrary string $w$ whether $w\in L(G)\,$?
Give an example.

\end{exercises}


\section{Pushdown Automata}\label{S-grammars-3b}

In the previous chapter, we saw that there is a neat correspondence
between regular expressions and finite automata.  That is, a language
is generated by a regular expression if and only if that language is
accepted by a finite automaton.  Finite automata come in two types,
deterministic and nondeterministic, but the two types of finite
automata are equivalent in terms of their ability to recognize
languages.  So, the class of regular languages\index{regular language} can be defined in
two ways: either as the set of languages that can be generated by
regular expressions or as the set of languages that can be recognized
by finite automata (either deterministic or nondeterministic).

In this chapter, we have introduced the class of context-free languages,
and we have considered how context-free grammars can be used to
generate context-free languages.  You might wonder whether there is any
type of automaton that can be used to recognize context-free languages.
In fact, there is:  The abstract machines known as 
\nw[pushdown automaton]{pushdown automata} can be used to define
context-free languages.  That is, a language is context-free if and only
if there is a pushdown automaton that accepts that language.

\medbreak

A pushdown automaton is essentially a finite automaton with an auxiliary 
data structure known as a \nw{stack}.  A stack consists of
a finite list of symbols.  Symbols can be added to and removed from the
list, but only at one end of the list.  The end of the list where items
can be added and removed is called the \nw[none]{top} of the stack.
The list is usually visualized as a vertical ``stack'' of symbols,
with items being added and removed at the top.  Adding a symbol at
the top of the stack is referred to as \nw[push operation on a stack]{pushing} a symbol onto 
the stack, and removing a symbol is referred to as \nw[pop operation on a stack]{popping}
an item from the stack.  During each step of its computation, 
a pushdown automaton is capable of doing several
push and pop operations on its stack (this in addition to possibly reading
a symbol from the input string that is being processed by the automaton).

Before giving a formal definition of pushdown automata, we will look
at how they can be represented by transition diagrams.  A diagram of
a pushdown automaton is similar to a diagram for an NFA, except that
each transition in the diagram can involve stack operations.  We will
use a label of the form $\sigma,x$/$y$ on a transition to mean
that the automaton consumes $\sigma$ from its input string, pops
$x$ from the stack, and pushes $y$ onto the stack.  $\sigma$ can be
either $\varepsilon$ or a single symbol.
$x$ and $y$ are strings, possibly empty. (When a string $x=a_1a_2\dots a_k$ is
pushed onto a stack, the symbols are pushed in the order $a_k,\dots,a_1$, so that
$a_1$ ends up on the top of the stack; for $y=b_1b_2\dots b_n$ to be popped
from the stack, $b_1$ must be the top symbol on the stack, followed by $b_2$, etc.) 
For example, consider the following transition diagram for a pushdown automaton:

\medskip
\centerline{\scaledeps{2truein}{fig-5-pa-1}}
\smallskip

This pushdown automaton has start state $q_0$ and one accepting
state, $q_1$.  It can read strings over the alphabet $\Sigma=\{a,b\}$.
The transition from $q_0$ to $q_0$, labeled with $a,\varepsilon$/1,
means that if the machine is in state $q_0$, then it can read an
$a$ from its input string, pop nothing from the stack, push 1 onto
the stack, and remain in state $q_0$.  Similarly, the transition
from $q_1$ to $q_1$ means that if the machine is in state $q_1$,
it can read a $b$ from its input string, pop a 1 from the stack,
and push nothing onto the stack.  Finally, the transition
from state $q_0$ to $q_1$, labeled with $\varepsilon,\varepsilon$/$\varepsilon$,
means that the machine can transition from state $q_0$ to state $q_1$
without reading, pushing, or popping anything.

Note that the automation can follow transition $b,1$/$\varepsilon$ only
if the next symbol in the input string is $b$ and if $1$ is on the 
top of the stack.  When it makes the transition, it consumes the $b$ from
input and pops the $1$ from the stack.  Since in this case, the automaton
pushes $\varepsilon$ (that is, no symbols at all) onto the stack, the net
change in the stack is simply to pop the 1.

We have to say what it means for this pushdown automaton to accept
a string.  For $w\in\{a,b\}^*$, we say that the pushdown automaton
accepts $w$ if and only if it is possible for the machine to start
in its start state, $q_0$, read all of $w$, and finish in the
accepting state, $q_1$, with an empty stack.  Note in particular that
it is not enough for the machine to finish in an accepting state---it
must also empty the stack.\footnote{We could relax this restriction
and require only that the machine finish in an accepting state after
reading the string $w$, without requiring that the stack be empty.
In fact, using this definition of accepting would not change the
class of languages that are accepted by pushdown automata.}

It's not difficult to see that with this definition, the language
accepted by our pushdown automaton is $\{a^nb^n\st n\in\N\}$.
In fact, given the string $w=a^kb^k$, the machine can process this
string by following the transition from $q_0$ to $q_0$ $k$ times.
This will consume all the $a$'s and will push $k$ 1's onto the stack.
The machine can then jump to state $q_1$ and follow the transition from
$q_1$ to $q_1$ $k$ times.  Each time it does so, it consumes one $b$ from
the input and pops one 1 from the stack.  At the end, the input has been
completely consumed and the stack is empty.  So, the string $w$ is
accepted by the automaton.  Conversely, this pushdown automaton \emph{only}
accepts strings of the form $a^kb^k$, since the only way that the
automaton can finish in the accepting state, $q_1$, is to follow
the transition from $q_0$ to $q_0$ some number of times, reading
$a$'s as it does so, then jump at some point to $q_1$, and then
follow the transition from $q_1$ to $q_1$ some number of times,
reading $b$'s as it does so.  This means that an accepted string
must be of the form $a^kb^\ell$ for some $k,\ell\in\N$.  However, in
reading this string, the automaton pushes $k$ 1's onto the stack
and pops $\ell$ 1's from the stack.  For the stack to end up empty,
$\ell$ must equal $k$, which means that in fact the string is of
the form $a^kb^k$, as claimed.  

\medbreak

Here are two more examples.  These pushdown automata use the capability
to push or pop more than one symbol at a time:

\medskip
\centerline{\scaledeps{4truein}{fig-5-pa-1b}}
\smallskip

\noindent The atomaton on the left accepts the language
$\{a^nb^m\st n\le m\le 2*n\}$.  Each time it reads an $a$,
it pushes either one or two 1's onto the stack, so that
after reading $n$ $a$'s, the number of 1's on the stack
is between $n$ and $2*n$.  If the machine then jumps to state
$q_1$, it must be able to read exactly enough $b$'s to empty 
the stack, so any string accepted by this machine must
be of the form $a^nb^m$ with $n\le m\le 2*n$.  Conversely,
any such string can be accepted by the machine. Similarly,
the automaton on the right above accepts the
language $\{a^nb^m\st n/2 \le m \le n\}$.
To accept $a^nb^m$, it must push $n$ 1'a onto the
stack and then pop one or two 1's for each b; this
can succeed only if the number of $b$'s is between
$n/2$ and $n$.



\medbreak

Note that an NFA can be considered to be a pushdown automaton that
does not make any use of its stack.  This means that any language
that can be accepted by an NFA (that is, any regular language) can
be accepted by a pushdown automaton.
Since the language $\{a^nb^n\st n\in\N\}$ is context-free but not regular,
and since it is accepted by the above pushdown automaton,
we see that pushdown automata are capable of recognizing context-free languages
that are not regular and so that pushdown automata are strictly more
powerful than finite automata.

\bigbreak

Although it is not particularly illuminating, we can give a formal
definition of pushdown automaton.  The definition does at least
make it clear that the set of symbols that can be used on the
stack is not necessarily the same as the set of symbols that can
be used as input.

\begin{definition}
A pushdown automaton\index{pushdown automaton} $M$ is specified by six components
$M=(Q,\Sigma,\Lambda,q_0,\partial,F)$ where
\begin{itemize}
\item $Q$ is a finite set of states.
\item $\Sigma$ is an alphabet.  $\Sigma$ is the \nw[none]{input alphabet} for $M$.
\item $\Lambda$ is an alphabet.  $\Lambda$ is the \nw[none]{stack alphabet} for $M$.
\item $q_0\in Q$ is the \nw[none]{start state} of $M$.
\item $F\SUB Q$ is the set of \nw[none]{final} or \nw[none]{accepting} states in $M$.
\item $\partial$ is the set of transitions in $M$.  $\partial$ can be taken
to be a finite subset of the set 
$(Q\times(\Sigma\cup\{\varepsilon\})\times\Lambda^*\big)\times\big(Q\times\Lambda^*\big)$.
An element $\big((q_1,\sigma,x),(q_2,y)\big)$ of $\partial$ represents a transition from
state $q_1$ to state $q_2$ in which $M$ reads $\sigma$ from its input string,
pops $x$ from the stack, and pushes $y$ onto the stack.
\end{itemize}
\end{definition}

We can then define the language $L(M)$ accepted by a pushdown
automaton $M=(Q,\Sigma,\Lambda,q_0,\partial,F)$ to be the set
$L(M)=\{w\in\Sigma^*\ |$ starting from state $q_0$, it is possible for $M$ to
read all of $w$ and finish in some state in $F$ with an empty stack$\}$.
With this definition, the class of languages accepted by pushdown automata
is the same as the class of languages generated by context-free grammars.

\begin{theorem}
Let $\Sigma$ be an alphabet, and let $L$ be a language over $L$.  Then
$L$ is context-free if and only if there is a pushdown automaton whose
input alphabet is $\Sigma$ such that $L=L(M)$.
\end{theorem}

We will not prove this theorem, but we do discuss how one direction can
be proved.  Suppose that $L$ is a context-free language over an alphabet
$\Sigma$.  Let $G=(V,\Sigma,P,S)$ be a context-free grammar for $L$.
Then we can construct a pushdown automaton $M$ that accepts $L$.  In fact,
we can take $M=(Q,\Sigma,\Lambda,q_0,\partial,F)$ where
$Q=\{q_0,q_1\}$, $\Lambda=\Sigma\cup V$, $F=\{q_1\}$, and
$\partial$ contains transitions of the forms

\begin{enumerate}
\item $\big((q_0,\varepsilon,\varepsilon),(q_1,S)\big)$;
\item $\big((q_1,\sigma,\sigma),(q_1,\varepsilon)\big)$, for $\sigma\in\Sigma$; and
\item $\big((q_1,\varepsilon,A),(q_1,x)\big)$, for each production $A\PRODUCES x$ in $G$.
\end{enumerate}

The transition $\big((q_0,\varepsilon,\varepsilon),(q_1,S)\big)$ lets $M$
move from the start state $q_0$ to the accepting state $q_1$ while reading
no input and pushing $S$ onto the stack.  This is the only possible first move
by $M$.  

A transition of the form $\big((q_1,\sigma,\sigma),(q_1,\varepsilon)\big)$, for $\sigma\in\Sigma$
allows $M$ to read $\sigma$ from its input string, provided there is a $\sigma$
on the top of the stack.  Note that if $\sigma$ is at the top of the stack, then this
transition is \emph{only} transition that applies.  Effectively, any terminal symbol
that appears at the top of the stack must be matched by the same symbol in the
input string, and the transition rule allows $M$ to consume the symbol from the
input string and remove it from the stack at the same time.

A transition of the third form, $\big((q_1,\varepsilon,A),(q_1,x)\big)$ can
be applied if and only if the non-terminal symbol $A$ is at the top of
the stack.  $M$ consumes no input when this rule is applied, but $A$ is
replaced on the top of the stack by the string on the right-hand
side of the production rule $A\PRODUCES x$.  Since the grammar $G$ can
contain several production rules that have $A$ as their left-hand side,
there can be several transition rules in $M$ that apply when $A$ is on the
top of the stack.  This is the only source of nondeterminism in $M$; note
that is also the source of nondeterminism in $G$.

The proof that $L(M)=L(G)$ follows from the fact that
a computation of $M$ that accepts a string $w\in\Sigma^*$ corresponds in
a natural way to a left derivation of $w$ from $G$'s start symbol, $S$.
Instead of giving a proof of this fact, we look at an example.
Consider the following context-free grammar:
\begin{align*}
  S&\PRODUCES AB\\
  A&\PRODUCES aAb\\
  A&\PRODUCES \varepsilon\\
  B&\PRODUCES bB\\
  B&\PRODUCES b
\end{align*}
This grammar generates the language $\{a^nb^m\st m > n\}$.  The pushdown
automaton constructed from this grammar by the procedure given above has
the following set of transition rules:
\begin{align*}
  &\big((q_0,\varepsilon,\varepsilon),(q_1,S)\big)\\
  &\big((q_1,a,a),(q_1,\varepsilon)\big)\\
  &\big((q_1,b,b),(q_1,\varepsilon)\big)\\
  &\big((q_1,\varepsilon,S),(q_1,AB)\big)\\
  &\big((q_1,\varepsilon,A),(q_1,aAb)\big)\\
  &\big((q_1,\varepsilon,A),(q_1,\varepsilon)\big)\\
  &\big((q_1,\varepsilon,B),(q_1,bB)\big)\\
  &\big((q_1,\varepsilon,B),(q_1,b)\big)\\
\end{align*}
Suppose that the automaton is run on the input $aabbbb$.  We can trace the
sequence of transitions that are applied in a computation that accepts this
input, and we can compare that computation to a left derivation of the
string:

{\def\stack#1{\vbox{\halign{##\hfil\cr#1}}}

\bigskip
\halign to \hsize{\qquad$#$\hfil&&$#$\hfil\cr
  \hbox{\bf\underbar{Transition}}                  &\qquad\hbox{\stack{\bf\underbar{Input}\cr\bf\underbar{Consumed}\cr}}  &
                          \qquad\hbox{\bf\underbar{Stack}} &\qquad\hbox to 0 pt{\bf\underbar{Derivation}\hss}\cr\noalign{\medskip}
  \big((q_0,\varepsilon,\varepsilon),(q_1,S)\big)  &                      &\qquad S         &\cr
  \big((q_1,\varepsilon,S),(q_1,AB)\big)           &                      &\qquad AB        &\qquad S\ &\YIELDS AB\cr
  \big((q_1,\varepsilon,A),(q_1,aAb)\big)          &                      &\qquad aAbB      &          &\YIELDS aAbB\cr
  \big((q_1,a,a),(q_1,\varepsilon)\big)            &\qquad\qquad a        &\qquad AbB       &\cr
  \big((q_1,\varepsilon,A),(q_1,aAb)\big)          &\qquad\qquad a        &\qquad aAbbB     &          &\YIELDS aaAbbB\quad\cr
  \big((q_1,a,a),(q_1,\varepsilon)\big)            &\qquad\qquad aa       &\qquad AbbB      &&\cr
  \big((q_1,\varepsilon,A),(q_1,\varepsilon)\big)  &\qquad\qquad aa       &\qquad bbB       &          &\YIELDS aabbB\cr
  \big((q_1,b,b),(q_1,\varepsilon)\big)            &\qquad\qquad aab      &\qquad bB        &&\cr
  \big((q_1,b,b),(q_1,\varepsilon)\big)            &\qquad\qquad aabb     &\qquad B         &&\cr
  \big((q_1,\varepsilon,B),(q_1,bB)\big)           &\qquad\qquad aabb     &\qquad bB        &          &\YIELDS aabbbB\cr
  \big((q_1,b),(q_1,b,\varepsilon)\big)            &\qquad\qquad aabbb    &\qquad B         &&\cr
  \big((q_1,\varepsilon,B),(q_1,b)\big)            &\qquad\qquad aabbb    &\qquad b         &          &\YIELDS aabbbb\cr
  \big((q_1,b,b),(q_1,\varepsilon)\big)            &\qquad\qquad aabbbb   &\qquad           &&\cr
}

}


\bigskip


Note that at all times during this computation, the concatenation of the input that has been consumed
so far with the contents of the stack is equal to one of the strings in the left derivation.
Application of a rule of the form $\big((q_1,\sigma,\sigma),(q_1,\varepsilon)\big)$ has the
effect of removing one terminal symbol from the ``Stack'' column to the ``Input Consumed'' column.
Application of a rule of the form $\big((q_1,\varepsilon,A),(q_1,x)\big)$ has the
effect of applying the next step in the left derivation to the non-terminal symbol on the top
of the stack.  (In the ``Stack'' column, the pushdown automaton's stack is shown with its
top on the left.)  In the end, the entire input string has been consumed and the stack is empty, which
means that the string has been accepted by the pushdown automaton.  It should be easy to see
that for any context free grammar $G$, the same correspondence will always hold between
left derivations and computations performed by the pushdown automaton constructed from $G$.


\bigbreak

The computation of a pushdown automaton can involve nondeterminism.
That is, at some point in the computation, there might be more than
one transition rule that apply.  When this is not the case---that is,
when there is no circumstance in which two different transition rules
apply---then we say that the pushdown automaton is \nw[pushdown automaton!deterministic]{deterministic}.
Note that a deterministic pushdown automaton can have transition
rules of the form $\big((q_i,\varepsilon,x),(q_j,y)\big)$ (or
even $\big((q_i,\varepsilon,\varepsilon),(q_j,y)\big)$ if that is
the \emph{only} transition from state $q_i$).  Note also that is is possible
for a deterministic pushdown automaton to get ``stuck''; that is, it
is possible that no rules apply in some circumstances even though the
input has not been completely consumed or the stack is not empty.
If a deterministic pushdown automaton gets stuck while reading a string
$x$, then $x$ is not accepted by the automaton.

The automaton given at the beginning of this section,
which accepts the language $\{a^nb^n\st n\in\N\}$, 
is not deterministic.  However, it is easy to construct a deterministic
pushdown automaton for this language:

\medskip
\centerline{\scaledeps{2truein}{fig-5-pa-2}}
\smallskip

\noindent However, consider the language $\{ww^R\st w\in\{a,b\}^*\}$.  Here is
a pushdown automaton that accepts this language:

\medskip
\centerline{\scaledeps{2truein}{fig-5-pa-3}}
\smallskip

\noindent In state $q_0$, this machine copies the first part of its input
string onto the stack.  In state $q_1$, it tries to match the remainder of the
input against the contents of the stack.  In order for this to work, it must ``guess''
where the middle of the string occurs by following the transition
from state $q_0$ to state $q_1$.  
In this case, it is by no means clear that it is possible to
construct a deterministic pushdown automaton that accepts the same
language.

\medbreak

At this point, it might be tempting to define a deterministic context-free
language as one for which there exists a deterministic pushdown automaton
which accepts that language.  However, there is a technical problem with
this definition:  we need to make it possible for the pushdown automaton
to detect the end of the input string.  Consider the language
$\{w\st w\in\{a,b\}^* \AND n_a(w)=n_b(w)\}$, which consists of strings over the
alphabet $\{a,b\}$ in which the number of $a$'s is equal to the number
of $b$'s.  This language is accepted by the following pushdown automaton:

\medskip
\centerline{\scaledeps{3.3truein}{fig-5-pa-4}}
\smallskip

\noindent In this automaton, a $c$ is first pushed onto the stack,
and it remains on the bottom of the stack until the computation ends.
During the process of reading an input string,
if the machine is in state $q_3$, then the
number of $a$'s that have been read is greater than or equal to
the number of $b$'s that have been read, and the stack contains
(copies of) the excess $a$'s that have been read.  Similarly,
if the machine is in state $q_4$, then the
number of $b$'s that have been read is greater than or equal to
the number of $a$'s that have been read, and the stack contains
(copies of) the excess $b$'s that have been read.
As the computation proceeds, if the stack contains nothing but a $c$,
then the number of $a$'s that have been consumed by the machine
is equal to the number of $b$'s that have been consumed; in such
cases, the machine can pop the $c$ from the stack---leaving the
stack empty---and jump to state $q_2$.  If the entire string has
been read at that time, then the string is accepted.  This involves
nondeterminism because the automaton has to ``guess'' when to
jump to state $q_2$; it has no way of knowing whether it has
actually reached the end of the string.

Although this pushdown automaton is not deterministic, we can
modify it easily to get a deterministic pushdown automaton that
accepts a closely related language.  We just have to add a
special end-of-string symbol to the language.  We use the
symbol {\tt\$} for this purpose.  The following deterministic
automaton accepts the language $\{w\hbox{\tt\$}\st w\in \{a,b\}^*\AND n_a(w)=n_b(w)\}\,$:

\medskip
\centerline{\scaledeps{3.3truein}{fig-5-pa-4b}}
\smallskip

\noindent In this modified automaton, it is only possible for the
machine to reach the accepting state $q_2$ by reading the end-of-string
symbol at a time when the number of $a$'s that have been consumed is equal
to the number of $b$'s.  Taking our cue from this example, we define 
what it means for a language to be deterministic context-free as follows:

\begin{definition}
Let $L$ be a language over an alphabet $\Sigma$, and let {\tt\$} be
a symbol that is not in $\Sigma$.  We say that $L$ is a \nw[context-free language!deterministic]{deterministic
context-free language}
if there is a deterministic pushdown automaton
that accepts the language $L\hbox{\tt\$}$ (which is equal to 
$\{w\hbox{\tt\$}\st w\in L\}$).
\end{definition}

There are context-free languages that are not deterministic context-free.
This means that for pushdown automata, nondeterminism adds real power.
This contrasts with the case of finite automata, where deterministic
finite automata and nondeterministic finite automata are equivalent in
power in the sense that they accept the same class of languages.

A deterministic context-free language can be parsed efficiently.
LL(1) parsing and LR(1) parsing can both be defined in terms of deterministic
pushdown automata, although we have not pursued that approach here.







\begin{exercises}

\problem Identify the context-free language that is accepted by each of the following
pushdown automata.  Explain your answers.

\medskip

\ppart \vtop{\hbox{}\vskip-10pt\hbox{\scaledeps{2.5truein}{fig-5-pa-ex1}}}

\medskip

\ppart \vtop{\hbox{}\vskip-10pt\hbox{\scaledeps{3.5truein}{fig-5-pa-ex2}}}

\medskip

\ppart \vtop{\hbox{}\vskip-10pt\hbox{\scaledeps{3.5truein}{fig-5-pa-ex3}}}

\medskip

\ppart \vtop{\hbox{}\vskip-10pt\hbox{\scaledeps{2.25truein}{fig-5-pa-ex4}}}


\problem Let $B$ be the language over the alphabet $\{\,\hbox{\tt(}\,,\hbox{\tt)}\,\}$ that consists of 
strings of parentheses that are balanced in the sense that every left parenthesis has
a matching right parenthesis.  Examples include {\tt ()}, {\tt (())()}, {\tt((())())()(())},
and the empty string.  Find a deterministic pushdown automaton with a single state that
accepts the language $B$.  Explain how your automaton works, and explain the circumstances
in which it will \emph{fail} to accept a given string of parentheses.

\problem Suppose that $L$ is language over an alphabet $\Sigma$.
Suppose that there is a deterministic pushdown automaton that accepts $L$.
Show that $L$ is deterministic context-free.  That is, show how to construct
a deterministic pushdown automaton that accepts the language $L${\tt\$}.
(Assume that the symbol {\tt\$} is not in $\Sigma$.)

\problem Find a deterministic pushdown automaton that accepts the language $\{wcw^R\st w\in\{a,b\}^*\}$.

\problem Show that the language $\{a^nb^m\st n\not=m\}$ is deterministic context-free.

\problem Show that the language $L=\{w\in\{a,b\}^*\st n_a(w) > n_b(w)\}$ is deterministic context-free.

\problem Let $M=(Q,\Sigma,\Lambda,q_0,\partial,F)$ be a pushdown automaton.  Define $L^\prime(M)$ to be 
the language $L^\prime(M)=\{w\in\Sigma^*\ |$ it is possible for $M$ to start in state $q_0$,
read all of $w$, and end in an accepting state$\}$.  $L^\prime(M)$ differs from $L(M)$ in that
for $w\in L^\prime(M)$, we do not require that the stack be empty at the end of the computation.
\ppart Show that there is a pushdown automaton $M^\prime$ such that $L(M^\prime)=L^\prime(M)$.
\ppart Show that a language $L$ is context-free if and only if there is a pushdown automaton
$M$ such that $L=L^\prime(M)$.
\ppart Identify the language $L^\prime(M)$ for each of the automata in Exercise 1.

\problem Let $L$ be a regular language over an alphabet $\Sigma$, and let $K$
be a context-free language over the same alphabet.  Let $M=(Q,\Sigma,q_0,\delta,F)$ be a DFA that
accepts $L$, and let $N=(P,\Sigma,\Lambda,p_0,\partial,E))$ be a pushdown automaton that accepts $K$.
Show that the language $L\cap K$ is context-free by constructing a pushdown automaton
that accepts $L\cap K$.  The pushdown automaton can be constructed as a ``cross product''
of $M$ and $N$ in which the set of states is $Q\times P$.  The construction is analogous
to the proof that the intersection of two regular languages is regular, as outlined
in Exercise~3.6.7.


\end{exercises}




\section{Non-context-free Languages}\label{S-grammars-4}

We have seen that there are context-free languages that are not
regular.  The natural question arises, are there languages that
are not context-free?  It's easy to answer this question in the
abstract:  For a given alphabet $\Sigma$, there are uncountably
many languages over $\Sigma$, but there are only
countably many context-free languages over $\Sigma$.  It follows
that most languages are not context-free.  However, this answer
is not very satisfying since it doesn't give us any example of
a specific language that is not context-free.

As in the case of regular languages, one way to show that
a given language $L$ is not context-free is to find some property
that is shared by all context-free languages and then to show that
$L$ does not have that property.  For regular languages, the
Pumping Lemma gave us such a property.  It turns out that
there is a similar Pumping Lemma for context-free languages.
The proof of this lemma uses parse trees.  In the proof, we
will need a way of representing abstract parse trees, without
showing all the details of the tree.  The picture

\medskip
\centerline{\scaledeps{0.5truein}{fig-5-4}}
\smallskip

\noindent represents a parse tree which has the non-terminal symbol
$A$ at its root and the string $x$ along the ``bottom'' of the tree.
(That is, $x$ is the string made up of all the symbols at the
endpoints of the tree's branches, read from left to right.)  Note that
this could be a partial parse tree---something that could be a part of a
larger tree.  That is, we do not require $A$ to be the start symbol
of the grammar and we allow $x$ to contain both terminal and
non-terminal symbols.  The string $x$, which is along the bottom
of the tree, is referred to as the \nw[yield of a parse tree]{yield}
of the parse tree.  Sometimes, we need to show more explicit detail in
the tree.  For example, the picture

\medskip
\centerline{\scaledeps{1truein}{fig-5-5}}

\noindent represents a parse tree in which the yield is the
string $xyz$.  The string $y$ is the yield of a smaller tree, with
root $B$, which is contained within the larger tree.
Note that any of the strings $x$, $y$, or $z$ could be the
empty string.  

We will also need the concept of the \nw[height of
a parse tree]{height} of a parse tree.  The height of a parse tree is
the length of the longest path from the root of the tree to the
tip of one of its branches.

Like the version for regular languages, the Pumping Lemma for
context-free languages shows that any sufficiently long string
in a context-free language contains a pattern that can be repeated
to produce new strings that are also in the language.  However,
the pattern in this case is more complicated.  For regular
languages, the pattern arises because any sufficiently long path
through a given DFA must contain a loop.   For context-free
languages, the pattern arises because in a sufficiently
large parse tree, along a path from the root of the tree to the
tip of one of its branches, there must be some non-terminal
symbol that occurs more than once.

\begin{theorem}[Pumping Lemma for Context-free Languages]
Suppose that $L$ is a context-free language.
Then there is an integer $K$ such that any string $w\in L(G)$
with $|w|\ge K$ has the property that $w$ can be written
in the form $w=uxyzv$ where
\Item{$\bullet\,$}$x$ and $z$ are not both equal to the empty string;
\Item{$\bullet\,$}$|xyz|< K$; and
\Item{$\bullet\,$}For any $n\in\N$, the string $ux^nyz^nv$ is in $L$.
\end{theorem}
\begin{proof}
Let $G=(V,\Sigma,P,S)$ be a context-free grammar for the language $L$.
Let $N$ be the number of non-terminal symbols in $G$, plus 1.  That is,
$N=|V|+1$.  Consider all possible parse trees for the grammar $G$
with height less than or equal to $N$.  (Include parse trees with any
non-terminal symbol as root, not just parse trees with root $S$.) 
There are only finitely many such parse trees, and therefore there
are only finitely many different strings that are the yields of
such parse trees.  Let $K$ 
be an integer which is greater than the length of any such string.

Now suppose that $w$ is any string in $L$ whose length is greater
than or equal to $K$.  Then \textit{any} parse tree for $w$ must have height
greater than $N$.  (This follows since $|w|\ge K$ and the yield of 
any parse tree of height $\le N$ has length less than $K$.)
Consider a parse tree for $w$ of minimal size, that is one that contains
the smallest possible number of nodes.  Since the height of this parse
tree is greater than $N$, there is at least one path from the
root of the tree to tip of a branch of the tree that has length
greater than $N$.  Consider the longest such path.  The symbol at
the tip of this path is a terminal symbol, but all the other symbols
on the path are non-terminal symbols.  There are at least $N$ such
non-terminal symbols on the path.  Since the number of different
non-terminal symbols is $|V|$ and since $N=|V|+1$, some non-terminal
symbol must occur twice on the path.  In fact, some non-terminal
symbol must occur twice among the bottommost $N$ non-terminal
symbols on the path.  Call this symbol $A$.  Then we see that
the parse tree for $w$ has the form shown here:

\medskip
\centerline{\scaledeps{2truein}{fig-5-6}}

\noindent The structure of this tree breaks the string $w$ into
five substrings, as shown in the above diagram.
We then have $w=uxyzv$.  It only remains to show that $x$,
$y$, and $z$ satisfy the three requirements stated in the 
theorem.

Let $T$ refer to the entire parse tree, let $T_1$ refer to the 
parse tree whose root is the upper $A$ in the diagram, and 
let $T_2$ be the parse tree whose root is the lower $A$ in
the diagram.  Note that the height of $T_1$ is less than 
or equal to $N$.  (This follows from two facts:  The path shown in
$T_1$ from its root to its base has length less than or equal to
$N$, because we chose the two occurrences of $A$ to be among the $N$
bottommost non-terminal symbols along the path in $T$ from its
root to its base.  We know that there is no longer path from
the root of $T_1$ to its base, since we chose the path in $T$
to be the longest possible path from the root of $T$ to its base.)
Since any parse tree with height less than or equal to $N$ has
yield of length less than $K$, we see that $|xyz|<K$.

If we remove $T_1$ from $T$ and replace it with
a copy of $T_2$, the result is a parse tree with
yield $uyv$, so we see that the string $uyv$ is in the language
$L$.  
Now, suppose that both $x$ and $z$ are equal to the
empty string.  In that case, $w=uyv$, so the tree we have
created would be another parse tree for $w$.  But this tree
is smaller than $T$, so this would contradict the fact that
$T$ is the smallest parse tree for $w$.  We see that
$x$ and $z$ cannot both be the empty string.

If we remove $T_2$ from $T$ and replace it with a copy
of $T_1$, the result is a parse tree with yield $ux^2yz^2v$,
so we see that $ux^2yz^2v\in L$.  The two parse trees that
we have created look like this:

\medskip
\centerline{\scaledeps{4truein}{fig-5-7}}
\smallskip

\noindent Furthermore, we can apply the process of replacing
$T_2$ with a copy of $T_1$ to the tree on the right above to
create a parse tree with yield $ux^3yz^3v$.  Continuing in this
way, we see that $ux^nyz^nv\in L$ for all $n\in\N$.
This completes the proof of the theorem.
\end{proof}


Since this theorem guarantees that all context-free languages
have a certain property, it can be used to show that specific
languages are not context-free.  The method is to show that
the language in question does not have the property that is
guaranteed by the theorem.  We give two examples.

\begin{corrolary}\label{T-CFG-anbncn}
Let $L$ be the language $\{a^nb^nc^n \st n\in\N\}$.  Then
$L$ is not a context-free language.
\end{corrolary}
\begin{proof}
We give a proof by contradiction.  Suppose that $L$ is
context-free.  Then, by the Pumping Lemma for Context-free Languages,
there is an integer $K$ such that every string $w\in L$ with 
$|w|\ge K$ can be written in the form $w=uxyzv$ where
$x$ and $z$ are not both empty, $|xyz|<K$, and $ux^nyz^nv\in L$
for every $n\in\N$.

Consider the string $w=a^Kb^Kc^K$, which is in $L$,
and write $w=uxyzv$, where $u$, $x$, $y$, $z$, and $v$ satisfy the
stated conditions.  Since $|xyz|<K$, we see that if $xyz$
contains an $a$, then it cannot contain a $c$.  And if it
contains a $c$, then it cannot contain an $a$.  It is also possible
that $xyz$ is made up entirely of $b\text{'}s$.  In any of these cases,
the string $ux^2yz^2v$ cannot be in $L$, since it does not contain
equal numbers of $a\text{'}s$, $b\text{'}s$, and $c\text{'}s$.  But this contradicts
the fact that $ux^nyz^nv\in L$ for all $n\in\N$.  This contradiction
shows that the assumption that $L$ is context-free is incorrect.
\end{proof}
\smallskip

\begin{corrolary}
Let $\Sigma$ be any alphabet that contains at least two symbols.
Let $L$ be the language over $\Sigma$ defined by $L=\{ss\st s\in\Sigma^*\}$.
Then $L$ is not context-free.
\end{corrolary}
\begin{proof}
Suppose, for the sake of contradiction, that $L$ is
context-free.  Then, by the Pumping Lemma for Context-free Languages,
there is an integer $K$ such that every string $w\in L$ with 
$|w|\ge K$ can be written in the form $w=uxyzv$ where
$x$ and $z$ are not both empty, $|xyz|<K$, and $ux^nyz^nv\in L$
for every $n\in\N$.

Let $a$ and $b$ represent distinct symbols in $\Sigma$.
Let $s=a^Kba^Kb$ and let $w=ss=a^Kba^Kba^Kba^Kb$, which is in $L$.
Write $w=uxyzv$, where $u$, $x$, $y$, $z$, and $v$ satisfy the
stated conditions.

Since $|xyz|<K$, $x$ and $z$ can, together, contain no more than one $b$.  If
either $x$ or $y$ contains a $b$, then $ux^2yz^2v$ contains exactly five
$b\text{'}s$.  But any string in $L$ is of the form $rr$ for some string $r$
and so contains an even number of $b\text{'}s$.  The fact that
$ux^2yz^2z$ contains five $b\text{'}s$ contradicts the fact that $ux^2yz^2v\in L$.
So, we get a contradiction in the case where $x$ or $y$ contains a $b$.

Now, consider the case where $x$ and $y$ consist entirely of $a\text{'}s$.
Again since $|xyz|<K$, we must have either that $x$ and $y$ are both
contained in the same group of $a\text{'}s$ in the string $a^Kba^Kba^Kba^Kb$,
or that $x$ is contained in one group of $a\text{'}s$ and $y$ is contained in
the next.  In either case, it is easy to check that the string
$ux^2yz^2v$ is no longer of the form $rr$ for any string $r$,
which contradicts the fact that $ux^2yz^2v\in L$.

Since we are led to a contradiction in every case, we see that the
assumption that $L$ is context-free must be incorrect.
\end{proof}


Now that we have some examples of languages that are not context-free,
we can settle some other questions about context-free languages.
In particular, we can show that the intersection of two context-free
languages is not necessarily context-free and that the complement of
a context-free language is not necessarily context-free.

\begin{theorem}
The intersection of two context-free languages is not necessarily a
context-free language.
\end{theorem}
\begin{proof}
To prove this, it is only necessary to produce an example of two
context-free languages $L$ and $M$ such that $L\cap M$ is not a
context-free languages.  Consider the following languages, defined
over the alphabet $\Sigma=\{a,b,c\}$:
\begin{align*}
   L&=\{a^nb^nc^m \st n\in\N\text{ and }m\in\N\}\\
   M&=\{a^nb^mc^m \st n\in\N\text{ and }m\in\N\}
\end{align*}
Note that strings in $L$ have equal numbers of $a\text{'}s$ and $b\text{'}s$ while
strings in $M$ have equal numbers of $b\text{'}s$ and $c\text{'}s$.  It follows that
strings in $L\cap M$ have equal numbers of $a\text{'}s$, $b\text{'}s$, and $c\text{'}s$.
That is,
\begin{align*}
   L\cap M=\{a^nb^nc^n\st n\in\N\}
\end{align*}
We know from the above theorem that $L\cap M$ is not context-free.
However, both $L$ and $M$ are context-free.  The language $L$
is generated by the context-free grammar
\begin{align*}
   S &\PRODUCES TC\\
   C &\PRODUCES cC\\
   C &\PRODUCES \EMPTYSTRING\\
   T &\PRODUCES aTb\\
   T &\PRODUCES \EMPTYSTRING
\end{align*}
and $M$ is generated by a similar context-free grammar.
\end{proof}
\smallskip

\begin{corrolary}
The complement of a context-free language is not necessarily context-free.
\end{corrolary}
\begin{proof}
  Suppose for the sake of contradiction that the complement of every
context-free language is context-free.  

Let $L$ and $M$ be two context-free languages over the alphabet $\Sigma$.
By our assumption, the complements $\overline{L}$ and $\overline{M}$
are context-free.  By Theorem~\ref{T-CFG-closures}, it follows
that $\overline{L}\cup\overline{M}$ is context-free.  Applying our
assumption once again, we have that $\overline{\overline{L}\cup\overline{M}}$
is context-free.
But $\overline{\overline{L}\cup\overline{M}}=L\cap M$, so we have
that $L\cap M$ is context-free.

We have shown, based on our assumption that the complement of any
context-free language is context-free, that the intersection of any
two context-free languages is context-free.  But this contradicts
the previous theorem, so we see that the assumption cannot be true.
This proves the theorem.
\end{proof}

Note that the preceding theorem and corollary say only that
$L\cap M$ is not context-free for \textit{some} context-free languages
$L$ and $M$ and that $\overline{L}$ is not context-free for 
\textit{some} context-free language $L$.  There are, of course, many 
examples of context-free languages $L$ and $M$ for which
$L\cap M$ and $\overline{L}$ \textit{are} in fact context-free.

Even though the intersection of two context-free languages is not necessarily
context-free, it happens that the intersection of a context-free language
with a \textit{regular} language is always context-free.  This is not difficult
to show, and a proof is outlined in Exercise~4.4.8.
I state it here without proof:

\begin{theorem}
Suppose that $L$ is a context-free language and that $M$ is a regular
language.  Then $L\cap M$ is a context-free language.
\end{theorem}

For example, let $L$ and $M$ be the languages
defined by $L=\{w\in \{a,b\}^*\st w=w^R\}$ and
$M=\{w\in\{a,b\}^*\st$ the length of $w$ is a multiple of 5$\}$.  Since 
$L$ is context-free and $M$ is regular, we know that
$L\cap M$ is context-free.  The language $L\cap M$ consists of
every palindrome over the alphabet $\{a,b\}$ whose length is a
multiple of five.

This theorem can also be used to show that certain languages are
\textit{not} context-free.  For example, consider the language
$L=\{w\in\{a,b,c\}^*\st n_a(w)$ $=n_b(w)=n_c(w)\}$.  (Recall that
$n_x(w)$ is the number of times that the symbol $x$ occurs in the
string $w$.)  We can use a proof by contradiction to show that
$L$ is not context-free. 
Let $M$ be the regular language defined by the regular
expression $a^*b^*c^*$.  It is clear that
$L\cap M=\{a^nb^nc^n\st n\in\N\}$.  If $L$ were context-free,
then, by the previous theorem, $L\cap M$ would be context-free.
However, we know from Theorem~\ref{T-CFG-anbncn} that $L\cap M$
is not context-free.  So we can conclude that $L$ is not
context-free.


\begin{exercises}

\problem Show that the following languages are not context-free:
\ppart $\{a^nb^mc^k\st n>m>k\}$
\ppart $\{w\in\{a,b,c\}^*\st n_a(w)>n_b(w)>n_c(w)\}$
\ppart $\{www\st w\in\{a,b\}^*\}$
\ppart $\{a^nb^mc^k\st n,m\in\N\text{ and } k=m*n\}$
\ppart $\{a^nb^m\st m=n^2\}$

\problem Show that the languages $\{a^n\st \,\text{n is a prime number}\}$
and $\{a^{n^2}\st n\in \N\}$ are not context-free.  (In fact, it 
can be shown that a language over the alphabet $\{a\}$ is
context-free if and only if it is regular.)

\problem Show that the language $\{w\in\{a,b\}^*\st n_a(w)=n_b(w)$
and $w$ contains the string $baaab$ as a substring$\}$ is context-free.

\problem Suppose that $M$ is any finite language and that
$L$ is any context-free language.  Show that the language
$L\SETDIFF M$ is context-free.  (Hint: Any finite language is a
regular language.)

\end{exercises}



\section{General Grammars}\label{S-grammars-5}

At the beginning of this chapter the general idea of a grammar as a set of
rewriting or production rules was introduced.  For most of the chapter, however,
we have restricted our attention to context-free grammars, in which production
rules must be of the form $A\PRODUCES x$ where $A$ is a non-terminal symbol.
In this section, we will consider general grammars, that is, grammars in which
there is no such restriction on the form of production rules.  For a general
grammar, a production rule has the form $u\PRODUCES x$, where $u$ is string
that can contain both terminal and non-terminal symbols.  For convenience, we
will assume that $u$ contains at least one non-terminal symbol, although
even this restriction could be lifted without changing the class of languages
that can be generated by grammars.  Note that a context-free grammar is, in fact,
an example of a general grammar, since production rules in a general grammar
are allowed to be of the form $A\PRODUCES x$.  They just don't have to be of
this form.  I will use the unmodified term \nw{grammar}\index{general grammar} to
refer to general grammars.\footnote{There is another special type of grammar that
is intermediate between context-free grammars and general grammars.  In a
so-called \nw{context-sensitive grammar}, every production rule is of the form
$u\PRODUCES x$ where $|x|\ge|u|$.  We will not cover context-sensitive grammars in
this text.}  The definition of grammar is identical to the
definition of context-free grammar, except for the form of the production rules:

\begin{definition}
A \nw{grammar} is a 4-tuple $(V,\Sigma,P,S)$,
where:

\Item{1.\ }$V$ is a finite set of symbols.  The elements of $V$
are the non-terminal symbols of the grammar.

\Item{2.\ }$\Sigma$ is a finite set of symbols such that $V\cap\Sigma=\emptyset$.
The elements of $\Sigma$ are the terminal symbols of the grammar.

\Item{3.\ }$P$ is a set of production rules.  Each rule is of the
form $u\PRODUCES x$ where $u$ and $x$ are strings in $(V\cup \Sigma)^*$
and $u$ contains at least one symbol from $V$.

\Item{4.\ }$S\in V$.  $S$ is the start symbol of the grammar.
\end{definition}

Suppose $G$ is a grammar.  Just as in the context-free case,
the language generated by $G$ is denoted by $L(G)$ and is defined
as $L(G)=\{x\in\Sigma^*\st S \YIELDS_G^* x\}$.  That is, a string
$x$ is in $L(G)$ if and only if $x$ is a string of terminal symbols
and there is a derivation that produces $x$ from the start symbol,
$S$, in one or more steps.

The natural question is whether there are languages that can be generated
by general grammars but that cannot be generated by context-free languages.
We can answer this question immediately by giving an example of such
a language.  Let $L$ be the language $L=\{w\in\{a,b,c\}^*\st n_a(w)=n_b(w)=n_c(w)\}$.
We saw at the end of the last section that $L$ is not context-free.
However, $L$ is generated by the following grammar:
\begin{align*}
  S&\PRODUCES SABC\\
  S&\PRODUCES \EMPTYSTRING\\
  AB&\PRODUCES BA\\
  BA&\PRODUCES AB\\
  AC&\PRODUCES CA\\
  CA&\PRODUCES AC\\
  BC&\PRODUCES CB\\
  CB&\PRODUCES BC\\
  A&\PRODUCES a\\
  B&\PRODUCES b\\
  C&\PRODUCES c
\end{align*}
For this grammar, the set of non-terminals is $\{S,A,B,C\}$ and the set of
terminal symbols is $\{a,b,c\}$.  Since both terminals and non-terminal
symbols can occur on the left-hand side of a production rule in a general
grammar, it is not possible, in general, to determine which symbols are non-terminal and
which are terminal just by looking at the list of production rules.
However, I will follow the convention that uppercase letters are always
non-terminal symbols.  With this convention, I can continue to specify a
grammar simply by listing production rules.

The first two rules in the above grammar make it possible to produce the strings
$\EMPTYSTRING$, $ABC$, $ABCABC$, $ABCABCABC$, and so on.  Each of these strings
contains equal numbers of $A\text{'}s$, $B\text{'}s$, and $C\text{'}s$.
The next six rules allow the order of the non-terminal symbols in the string
to be changed.  They make it possible to arrange the $A\text{'}s$, $B\text{'}s$,
and $C\text{'}s$ into any arbitrary order.  Note that these rules could not occur
in a context-free grammar.  The last three rules convert the non-terminal symbols
$A$, $B$, and $C$ into the corresponding terminal symbols $a$, $b$, and $c$.
Remember that all the non-terminals must be eliminated in order to produce
a string in $L(G)$.  Here, for example, is a derivation of the string
$baabcc$ using this grammar.  In each line, the string that will be replaced
on the next line is underlined.
\begin{align*}
   S&\YIELDS \underline{S}ABC\\
    &\YIELDS \underline{S}ABCABC\\
    &\YIELDS \underline{AB}CABC\\
    &\YIELDS BA\underline{CA}BC\\
    &\YIELDS BAA\underline{CB}C\\
    &\YIELDS \underline{B}AABCC\\
    &\YIELDS b\underline{A}ABCC\\
    &\YIELDS ba\underline{A}BCC\\
    &\YIELDS baa\underline{B}CC\\
    &\YIELDS baab\underline{C}C\\
    &\YIELDS baabc\underline{C}\\
    &\YIELDS baabcc
\end{align*}
We could produce any string in $L$ in a similar way.  Of course,
this only shows that $L\SUB L(G)$.  To show that $L(G)\SUB L$,
we can observe that for any string $w$ such that $S\;\YIELDSTAR w$,
$n_A(w)+n_a(w) = n_B(w)+n_b(w) = n_C(w)+n_c(w)$.  This follows since the
rule $S\YIELDS SABC$ produces strings in which $n_A(w)=n_B(w)=n_C(w)$,
and no other rule changes any of the quantities
$n_A(w)+n_a(w)$, $n_B(w)+n_b(w)$, or $n_C(w)+n_c(w)$.
After applying these rules to produce a string $x\in L(G)$, we must
have that $n_A(x)$, $n_B(x)$, and $n_C(x)$ are zero.  The fact that
$n_a(x)=n_b(x)=n_c(x)$ then follows from the fact that
$n_A(x)+n_a(x) = n_B(x)+n_b(x) = n_C(x)+n_c(x)$.  That is, $x\in L$.
\medskip

Our first example of a non-context-free language was $\{a^nb^nc^n\st n\in \N\}$.
This language can be generated by a general grammar similar to the previous
example.  However, it requires some cleverness to force the $a\text{'}s$,
$b\text{'}s$, and $c\text{'}s$ into the correct order.  To do this, instead of
allowing $A\text{'}s$, $B\text{'}s$, and $C\text{'}s$ to transform themselves
spontaneously into $a\text{'}s$, $b\text{'}s$, and $c\text{'}s$, we use additional
non-terminal symbols to transform them only after they are in the correct position.
Here is a grammar that does this:
\begin{align*}
  S&\PRODUCES SABC\\
  S&\PRODUCES X\\
  BA&\PRODUCES AB\\
  CA&\PRODUCES AC\\
  CB&\PRODUCES BC\\
  XA&\PRODUCES aX\\
  X&\PRODUCES Y\\
  YB&\PRODUCES bY\\
  Y&\PRODUCES Z\\
  ZC&\PRODUCES cZ\\
  Z&\PRODUCES \EMPTYSTRING
\end{align*}
Here, the first two rules produce one of the strings $X$, $XABC$, $XABCABC$,
$XABCABCABC$, and so on.  The next three rules allow $A\text{'s}$ to move to the
left and $C\text{'}s$ to move to the right, producing a string of the form $XA^nB^nC^n$,
for some $n\in\N$.  The rule $XA\PRODUCES aX$ allows the
$X$ to move through the $A\text{'}s$ from left to right, converting $A\text{'}s$
to $a\text{'}s$ as it goes.  After converting the $A\text{'}s$, the $X$ can be
transformed into a $Y$.  The $Y$ will then move through the $B\text{'}s$, converting
them to $b\text{'}s$.  Then, the $Y$ is transformed into a $Z$, which is responsible
for converting $C\text{'}s$ to $c\text{'}s$.  Finally, an application of the
rule $Z\PRODUCES\EMPTYSTRING$ removes the $Z$, leaving the string $a^nb^nc^n$.

Note that if the rule $X\PRODUCES Y$ is applied before all the $A\text{'}s$ have
been converted to $a\text{'}s$, then there is no way for the remaining $A\text{'}s$
to be converted to $a\text{'}s$ or otherwise removed from the string.  This means
that the derivation has entered a dead end, which can never produce a string
that consists of terminal symbols only.  The only derivations that can produce
strings in the language generated by the grammar are derivations in which the
$X$ moves past all the $A\text{'}s$, converting them all to $a\text{'}s$.  At this
point in the derivation, the string is of the form $a^nXu$ where $u$ is a string
consisting entirely of $B\text{'}s$ and $C\text{'}s$.  At this point, the
rule $X\PRODUCES Y$ can be applied, producing the string $a^nYu$.  Then, if a string
of terminal symbols is ever to be produced, the $Y$ must move past all the $B\text{'}s$,
producing the string $a^nb^nYC^n$.  You can see that the use of three separate
non-terminals, $X$, $Y$, and $Z$, is essential for forcing the symbols in
$a^nb^nc^n$ into the correct order.

\medbreak

For one more example, consider the language $\{a^{n^2}\st n\in\N\}$.  Like the other
languages we have considered in this section, this language is not context-free.
However, it can be generated by a grammar.  Consider the grammar
\begin{align*}
  S&\PRODUCES DTE\\
  T&\PRODUCES BTA\\
  T&\PRODUCES \EMPTYSTRING\\
  BA&\PRODUCES AaB\\
  Ba&\PRODUCES aB\\
  BE&\PRODUCES E\\
  DA&\PRODUCES D\\
  Da&\PRODUCES aD\\
  DE&\PRODUCES \EMPTYSTRING
\end{align*}
The first three rules produce all strings of the form $DB^nA^nE$, for $n\in\N$.
Let's consider what happens to the string $DB^nA^nE$ as the remaining rules are applied.
The next two rules allow a $B$ to move to the right until it reaches the $E$.
Each time the $B$ passes an $A$, a new $a$ is generated, but a $B$ will simply
move past an $a$ without generating any other characters.  Once the $B$ reaches
the $E$, the rule $BE\PRODUCES E$ makes the $B$ disappear.  Each $B$ from the
string $DB^nA^nE$ moves past $n$ $A\text{'}s$ and generates $n$ $a\text{'}s$.
Since there are $n$ $B\text{'}s$, a total of $n^2$ $a\text{'}s$ are generated.
Now, the only way to get rid of the $D$ at the beginning of the string is for
it to move right through all the $A\text{'}s$ and $a\text{'}s$ until it reaches
the $E$ at the end of the string.  As it does this, the rule $DA\PRODUCES D$
eliminates all the $A\text{'}s$ from the string, leaving the string $a^{n^2}DE$.
Applying the rule $DE\PRODUCES\EMPTYSTRING$ to this gives $a^{n^2}$.  This
string contains no non-terminal symbols and so is in the language generated
by the grammar.  We see that every string of the form $a^{n^2}$ is generated
by the above grammar.  Furthermore, only strings of this form can be generated
by the grammar.  

\medbreak

Given a fixed alphabet $\Sigma$, there are only countably many different
languages over $\Sigma$ that can be generated by grammars.  Since there
are uncountably many different languages over $\Sigma$, we know that
there are many languages that cannot be generated by grammars.
However, it is surprisingly difficult to find an actual example of
such a language.

As a first guess, you might suspect that just as $\{a^nb^n \st n\in\N\}$
is an example of a language that is not regular and
$\{a^nb^nc^n\st n\in\N\}$ is an example of a language that is not
context-free, so $\{a^nb^nc^nd^n\st n\in\N\}$ might be an example
of a language that cannot be generated by any grammar.  However,
this is not the case.  The same technique that was used
to produce a grammar that generates $\{a^nb^nc^n\st n\in\N\}$ can
also be used to produce a grammar for $\{a^nb^nc^nd^n\st n\in\N\}$.
In fact, the technique extends to similar languages based on any
number of symbols.

Or you might guess that there is no grammar for the
language $\{a^n\st\,$ $n$ is a prime number$\,\}$.  Certainly, producing
prime numbers doesn't seem like the kind of thing that we would
ordinarily do with a grammar.  Nevertheless, there is a grammar that
generates this language.  We will not actually write down the grammar,
but we will eventually have a way to prove that it exists.

The language $\{a^{n^2}\st n\in\N\}$ really doesn't seem all that
``grammatical'' either, but we produced a grammar for it above.
If you think about how this grammar works, you might get the feeling
that its operation is more like ``computation'' than ``grammar.''
This is our clue.  A grammar can be thought of as a kind of program,
albeit one that is executed in a nondeterministic fashion.  It turns
out that general grammars are precisely as powerful as any other
general-purpose programming language, such as Java or C++.  More
exactly, a language can be generated by a grammar if and only if
there is a computer program whose output consists of a list 
containing all the strings and only the
strings in that language.  Languages that have this property
are said to be \nw[recursively enumerable language]{recursively 
enumerable languages}.  (This term
as used here is {\it not\/} closely related to the idea of a recursive
subroutine.)  The languages that can be generated by general
grammars are precisely the recursively enumerable languages.
We will return to this topic in the next chapter.

It turns out that there are many forms of computation that are
precisely equivalent in power to grammars and to computer programs,
and no one has ever found any form of computation that is more
powerful.  This is one of the great discoveries of the twentieth
century, and we will investigate it further in the next chapter.


\begin{exercises}

\problem Find a derivation for the string $caabcb$, according to the first example
grammar in this section.
Find a derivation for the string $aabbcc$, according to the second example
grammar in this section.
Find a derivation for the string $aaaa$, according to the third example
grammar in this section.

\problem Consider the third sample grammar from this section, which generates
the language $\{a^{n^2}\st n\in\N\}$.  Is the non-terminal symbol $D$ necessary
in this grammar?  What if the first rule of the grammar were replaced by
$S\PRODUCES TE$ and the last three rules were replaced by $A\PRODUCES\EMPTYSTRING$
and $E\PRODUCES\EMPTYSTRING\,$?  Would the resulting grammar still generate
the same language?  Why or why not?

\problem Find a grammar that generates the language $L=\{w\in\{a,b,c,d\}^*\st
n_a(w)=n_b(w)=n_c(w)=n_d(w)\}$.  Let $\Sigma$ be any alphabet.
Argue that the language $\{w\in\Sigma^*\st\,$ all symbols in $\Sigma$ occur equally
often in $w\,\}$ can be generated by a grammar.

\problem For each of the following languages, find a grammar that generates
the language.  In each case, explain how your grammar works.
\pparts{
   \{a^nb^nc^nd^n\st n\in\N\}&
   \{a^nb^mc^{nm}\st n\in\N\text{ and }m\in\N\}\cr
   \{ww\st w\in\{a,b\}^*\}&
   \{www\st w\in\{a,b\}^*\}\cr
   \{a^{2^n}\st n\in\N\}&
   \{w\in\{a,b,c\}^*\st n_a(w)>n_b(w)>n_c(w)\}\cr
}


\end{exercises}




\endinput


% !TEX root = main.tex

\chapter{Turing Machines and Computability}\label{C-turing}

\renewcommand{\b}{{\tt\#}}
\newcommand{\at}{{\tt\char`\@}}

\startchapter{We saw hints} at the end of the previous chapter that
``computation'' is a more general concept than we might have thought.
General grammars, which at first encounter don't seem to have much
to do with algorithms or computing, turn out to be able to do things
that are similar to the tasks carried out by computer programs.
In this chapter, we will see that general grammars are precisely
equivalent to computer programs in terms of their computational
power, and that both are equivalent to a particularly simple model
of computation known as a \nw{Turing machine}.  We shall also see
that there are limits to what can be done by computing.


\section{Turing Machines}\label{S-turing-1}

Historically, the theoretical study of computing began before computers
existed.  One of the early models of computation was developed in the
1930s by the British mathematician, Alan Turing, who was interested in
studying the theoretical abilities and limitations of computation.
His model for computation is a very simple abstract computing machine
which has come to be known as a \nw{Turing machine}.  While Turing
machines are not applicable in the same way that regular expressions,
finite-state automata, and grammars are applicable, their use as a
fundamental model for computation means that every computer scientist
should be familiar with them, at least in a general way.

A Turing machine is really not much more complicated than a finite-state 
automaton or a pushdown automaton.\footnote{In fact, Turing machines can
be shown to be equivalent in their computational power
to pushdown automata with two independent stacks.}
Like a FSA, a Turing machine has a finite number of 
possible states, and it changes from state to state as it computes.
However, a Turing machine also has an infinitely long \nw[tape of a Turing machine]{tape}
that it can use for input and output.  The tape extends to infinity in
both directions.  The tape is divided into \nw[cell]{cells}, which
are in one-to-one correspondence with the
integers,~$\Z$. Each cell can either be blank or it can hold a symbol from
a specified alphabet.  The Turing machine can move back and forth
along this tape, reading and writing symbols and changing state.
It can read only one cell at a time, and possibly write a new
value in that cell.  After doing this, it can change state and
it can move by one cell either to the left or to the right.
This is how the Turing machine computes.  To use a Turing machine,
you would write some input on its tape, start the machine, and let
it compute until it halts.  Whatever is written on the tape at that
time is the output of the computation.

Although the tape is infinite, only a finite number
of cells can be non-blank at any given time.  
If you don't like the idea
of an infinite tape, you can think of a finite tape that can be
extended to an arbitrarily large size as the Turing machine computes:
If the Turing machine gets to either end of the tape, it will pause and
wait politely until you add a new section of tape.  In other words,
it's not important that the Turing machine have an infinite amount of
memory, only that it can use as much memory as it needs for a given
computation, up to any arbitrarily large size.   In this way, a Turing
machine is like a computer that can ask you to buy it a new disk drive
whenever it needs more storage space to continue a computation.\footnote{The
tape of a Turing machine can be used to store arbitrarily large amounts of
information in a straightforward way.  Although we can imagine using
an arbitrary amount of memory with a computer, it's not so easy.  Computers
aren't set up to keep track of unlimited amounts of data.  If you think 
about how it might be done, you probably won't come with anything better
than an infinite tape. (The problem is that computers use integer-valued
addresses to keep track of data locations.  If a limit is put on
the number of bits in an address, then only a fixed, finite amount
of data can be addressed.  If no limit is put on the number of bits
in an address, then we are right back to the problem of storing an
arbitrarily large piece of data---just to represent an address!)}

A given Turing machine has a fixed, finite set of states.  One of
these states is designated as the \nw[start state of a Turing machine]{start
state}.  This is the state in which the Turing machine begins a computation.
Another special state is the \nw[halt state of a Turing machine]{halt
state}.  The Turing machine's computation ends when it enters its
halt state.  It is possible that a computation might never end because
the machine never enters the halt state.  This is analogous to an 
infinite loop in a computer program.

At each step in its computation,
the Turing machine reads the contents of the tape cell where it is located.
Depending on its state and the symbol that it reads, the machine
writes a symbol (possibly the same symbol) to the cell, moves one cell
either to the left or to the right, and (possibly) changes its state.
The output symbol, direction of motion, and new state are determined
by the current state and the input symbol.  Note that either the input
symbol, the output symbol, or both, can be blank. 
A Turing machine has a fixed set of \nw[rule in a Turing machine]{rules}
that tell it how to compute.  Each rule
specifies the output symbol, direction of motion, and new state for
some combination of current state and input symbol.  The machine has
a rule for every possible combination of current state and input symbol,
except that there are no rules for what happens if the current state
is the halt state.  Of course, once the machine enters the halt state,
its computation is complete and the machine simply stops.

I will use the character \b\ to represent a blank in a way
that makes it visible.  I will always use $h$ to represent the halt
state.  I will indicate the directions, left and right, with
$L$ and $R$, so that $\{L,R\}$ is the set of possible directions of
motion.  With these conventions, we can give the formal definition of
a Turing machine as follows:

\begin{definition}
A \nw{Turing machine} is a 4-tuple $(Q,\Lambda,q_0,\delta)$,
where:

\IItem{1.\ }$Q$ is a finite set of states, including the halt state, $h$.

\IItem{2.\ }$\Lambda$ is an alphabet which includes the blank symbol, \b.

\IItem{3.\ }$q_0\in Q$ is the start state.

\IItem{4.\ }$\delta\colon (Q\SETDIFF\{h\})\times\Lambda \to \Lambda\times 
\{L,R\}\times Q$ is the transition function.  The fact that
$\delta(q,\sigma)=(\tau,d,r)$ means that when the Turing machine is
in state $q$ and reads the symbol $\sigma$, it writes the symbol
$\tau$, moves one cell in the direction $d$, and enters state $r$.

\end{definition}

\medskip

Even though this is the formal definition, it's easier to work with
a transition diagram representation of Turing machines.  The transition
diagram for a Turing machine is similar to the transition diagram for
a DFA.  However, there are no ``accepting'' states (only a halt state).
Furthermore, there must be a way to specify the output symbol and
the direction of motion for each step of the computation.
We do this by labeling arrows with notations of the
form $(\sigma,\tau,L)$ and $(\sigma,\tau,R)$, where
$\sigma$ and $\tau$ are symbols in the Turing machine's alphabet.
For example,

\medbreak
\centerline{\eps{turing1}}
\medbreak

\noindent indicates that when the machine is in state $q_0$ and
reads an $a$, it writes a $b$, moves left, and enters state $h$.


Here, for example, is a transition diagram for a simple Turing machine
that moves to the right, changing $a$'s to $b$'s and \textit{vice
versa}, until it finds a $c$.  It leaves blanks (\b's) unchanged.
When and if the machine encounters a $c$, it moves to the left
and halts:

\medbreak
\centerline{\eps{turing2}}
\medbreak


To simplify the diagrams, I will leave out any transitions that are
not relevant to the computation that I want the machine to perform.
You can assume that the action for any omitted transition is
to write the same symbol that was read, move right, and halt.

For example, shown below is a transition diagram for a Turing machine
that makes a copy of a string of $a$'s and $b$'s.  To use this machine,
you would write a string of $a$'s and $b$'s on its tape, place
the machine on the first character of the string, and start the
machine in its start state,~$q_0$.  When the machine halts, there will be
two copies of the string on the tape, separated by a blank.
The machine will be positioned on the first character of the leftmost
copy of the string.  Note that this machine uses $c$'s and
$d$'s in addition to $a$'s and $b$'s.  While it is copying the
input string, it temporarily changes the $a$'s and $b$'s that it
has copied to $c$'s and $d$'s, respectively.  In this way it can 
keep track of which characters it has already copied.  After the
string has been copied, the machine changes the $c$'s and $d$'s
back to $a$'s and $b$'s before halting.

\breakSixByNine

\medbreak
\centerline{\scaledeps{4 true in}{turing3}}
\medbreak

In this machine, state $q_0$ checks whether the next character
is an $a$, a $b$, or a \b\ (indicating the end of the string).
States $q_1$ and $q_2$ add an $a$ to the end of the new string,
and states $q_3$ and $q_4$ do the same thing with a $b$.
States $q_5$ and $q_6$ return the machine to the next character
in the input string.  When the end of the input string is reached,
state $q_7$ will move the machine back to the start of the input
string, changing $c$'s and $d$'s back to $a$'s and $b$'s as it goes.
Finally, when the machine hits the \b\ that precedes the input string,
it moves to the right and halts.  This leave it back at the first
character of the input string.  It would be a good idea to work through
the execution of this machine for a few sample input strings.
You should also check that it works even for an input string of
length zero.

\medbreak

Our primary interest in Turing machines is as language processors.
Suppose that $w$ is a string over an alphabet $\Sigma$.  We will assume
that $\Sigma$ does not contain the blank symbol.  We can use $w$ as
input to a Turing machine $M=(Q,\Lambda,q_0,\delta)$ provided that
$\Sigma\SUB\Lambda$.  To use $w$ as input for $M$, we will write
$w$ on $M$'s tape and assume that the remainder of the tape is blank.
We place the machine on the cell containing the first character
of the string, except that if $w=\EMPTYSTRING$ then we simply place the
machine on a completely blank tape.   Then we start the machine in its 
initial state, $q_0$, and see what computation it performs.
We refer to this setup as ``running $M$ with input $w$.''

When $M$ is run with input $w$, it is possible that it will just keep
running forever without halting.  In that case, it doesn't make
sense to ask about the output of the computation.  Suppose however
that $M$ does halt on input $w$.  Suppose, furthermore, that when
$M$ halts, its tape is blank except for a string $x$ of non-blank
symbols, and that the machine is located on the first character
of $x$.  In this case, we will say that ``$M$ halts with output $x$.''
In addition, if $M$ halts with an entirely blank tape, we say that
``$M$ halts with output $\varepsilon$.''
Note that when we run $M$ with input $w$, one of three things can happen:
(1)~$M$~might halt with some string as output; (1)~$M$~might fail to halt; 
or (3)~$M$~might halt in some configuration that doesn't
count as outputting any string.

The fact that a Turing machine can produce an output value allows us
for the first time to deal with computation of \textit{functions}.
A function $f\colon A\to B$ takes an input value in the set $A$
and produces an output value in the set $B$.  If the sets are sets
of strings, we can now ask whether the values of the function can
be computed by a Turing machine.  That is, is there a Turing machine $M$
such that, given any string $w$ in the domain of $f$ as input,
$M$ will compute as its output the string $f(w)$.  If this is
that case, then we say that $f$ is a Turing-computable function.
\begin{definition} Suppose
that $\Sigma$ and $\Gamma$ are alphabets that do not contain \b\ and that
$f$ is a function from $\Sigma^*$ to $\Gamma^*$.  We say that
$f$ is \nw{Turing-computable} if there is a Turing machine
$M=(Q,\Lambda,q_0,\delta)$ such that $\Sigma\SUB\Lambda$ and $\Gamma\SUB\Lambda$
and for each string $w\in\Sigma^*$, when $M$ is run with input $w$,
it halts with output $f(w)$.  In this case, we say that $M$
\nw[none]{computes} the function $f$.
\end{definition}
\noindent For example, let $\Sigma=\{a\}$ and define $f\colon\Sigma^*\to\Sigma^*$
by $f(a^n)=a^{2n}$, for $n\in\N$.  Then $f$ is Turing-computable
since it is computed by this Turing machine:

\medbreak

\breakSixByNine

\centerline{\scaledeps{4 true in}{turing4}}
\medbreak

We can also use Turing machines to define ``computable languages.''
There are actually two different notions of Turing-computability
for languages.  One is based on the idea of Turing-computability
for functions.  Suppose that $\Sigma$ is an alphabet and that
$L\SUB\Sigma^*$.  The \nw{characteristic function} of $L$
is the function $\chi_L\colon\Sigma^*\to\{0,1\}$ defined
by the fact that $\chi_L(w)=1$ if $w\in L$ and $\chi_L(w)=0$
if $w\not\in L$.  Note that given the function $\chi_L$,
$L$ can be obtained as the set $L=\{w\in\Sigma^*\st \chi_L(w)=1\}$.
Given a language $L$, we can ask whether the corresponding function
$\chi_L$ is Turing-computable.  If so, then we can use a Turing
machine to decide whether or not a given string $w$ is in $L$.
Just run the machine with input $w$.  It will halt with output $\chi_L(w)$.
(That is, it will halt and when it does so, the tape will be blank except for
a 0 or a 1, and the machine will be positioned on the 0 or~1.)
If the machine halts with output 1, then $w\in L$.  If the machine halts with
output 0, then $w\not\in L$.
\begin{definition}
Let $\Sigma$ be an alphabet that does not contain \b\ and let $L$ be a language over $\Sigma$.
We say that $L$ is \nw{Turing-decidable} if there is a Turing machine
$M=(Q,\Lambda,q_0,\delta)$ such that $\Sigma\SUB\Lambda$, $\{0,1\}\SUB\Lambda$,
and for each $w\in\Sigma^*$, when $M$ is run with input $w$, it halts
with output $\chi_L(w)$.  (That is, it halts with output 0 or 1, and
the output is 0 if $w\not\in L$ and is 1 if $w\in L$.)  In this case,
we say that $M$ \nw[none]{decides} the language $L$.
\end{definition}

The second notion of computability for languages is based on the
interesting fact that it is possible for a Turing machine to run
forever, without ever halting.
Whenever we run a Turing machine $M$ with input $w$,
we can ask the question, will $M$ ever halt or will it run forever?  If $M$
halts on input $w$, we will say that $M$ ``accepts'' $w$.  We can then
look at all the strings over a given alphabet that are accepted by
a given Turing machine.  This leads to the notion of Turing-acceptable
languages.
\begin{definition}
Let $\Sigma$ be an alphabet that does not contain \b, and let $L$ be a language over $\Sigma$.
We say that $L$ is \nw{Turing-acceptable} if there is a Turing machine
$M=(Q,\Lambda,q_0,\delta)$ such that $\Sigma\SUB\Lambda$, and
for each $w\in\Sigma^*$, $M$ halts on input $w$ if and only if $w\in L$.
In this case, we say that $M$ \nw[none]{accepts} the language $L$.
\end{definition}

It should be clear that any Turing-decidable language is Turing-acceptable.
In fact, if $L$ is a language over an alphabet $\Sigma$,
and if $M$ is a Turing machine that
decides $L$, then it is easy to modify $M$ to produce a Turing machine
that accepts $L$.  At the point where $M$ enters the halt state with
output 0, the new machine should enter a new state in which it simply
moves to the right forever, without ever halting.  Given an input
$w\in\Sigma^*$, the modified machine will halt if and only if $M$
halts with output 1, that is, if and only if $w\in L$.

\begin{exercises}

\problem Let $\Sigma=\{a\}$.  Draw a transition diagram for a Turing
machine that computes the function $f\colon\Sigma^*\to\Sigma^*$ where
$f(a^n)=a^{3n}$, for $n\in\N$. Draw a transition diagram for a Turing
machine that computes the function $f\colon\Sigma^*\to\Sigma^*$ where
$f(a^n)=a^{3n+1}$, for $n\in\N$.

\problem Let $\Sigma=\{a,b\}$.
Draw a transition diagram for a Turing machine that
computes the function $f\colon\Sigma^*\to\Sigma^*$ where
$f(w)=w^R$.

\problem Suppose that $\Sigma$, $\Gamma$, and $\Xi$ are alphabets and that
$f\colon\Sigma^*\to\Gamma^*$ and $g\colon\Gamma^*\to\Xi^*$ are 
Turing-computable functions.  Show that $g\circ f$ is Turing-computable.

\problem We have defined computability for functions $f\colon\Sigma^*\to\Gamma^*$,
where $\Sigma$ and $\Gamma$ are alphabets.  How could Turing machines
be used to define computable functions from $\N$ to $\N\,$?
(Hint: Consider the alphabet $\Sigma=\{a\}$.)

\problem Let $\Sigma$ be an alphabet and let $L$ be a language over $\Sigma$.
Show that $L$ is Turing-decidable if and only if its complement,
$\overline{L}$, is Turing-decidable.

\problem Draw a transition diagram for a Turing machine which
decides the language $\{a^nb^n\st n\in\N\}$.  (Hint: Change the
$a$'s and $b$'s to \$'s in pairs.)  Explain in general terms how to
make a Turing machine that decides the language $\{a^nb^nc^n\st n\in\N\}$.

\problem Draw a transition diagram for a Turing machine which
decides the language $\{a^nb^m\st n>0$ and $m$ is a multiple of $n\}$.
(Hint: Erase $n$ $b$'s at a time.)

\problem Based on your answer to the previous problem and the copying
machine presented in this section, describe in
general terms how you would build a Turing machine to decide the
language $\{a^p\st p$ is a prime number$\}$.

\problem Let $g\colon \{a\}^*\to\{0,1\}^*$ be the function such that
for each $n\in\N$, $g(a^n)$ is the representation of $n$ as a binary
number.  Draw a transition diagram for a Turing machine that computes $g$.



\end{exercises}


\section{Computability}\label{S-turing-2}


At this point, it would be useful to look at increasingly complex
Turing machines, which compute increasingly complex functions and languages.
Although Turing machines are very simple devices, it turns out that
they can perform very sophisticated computations.  In fact, any
computation that can be carried out by a modern digital computer---even
one with an unlimited amount of memory---can be carried out by
a Turing machine.  Although it is not something that can be 
proved, it is widely believed that anything that can reasonably
be called ``computation'' can be done by a Turing machine.  This
claim is known as the \nw{Church-Turing Thesis}.

We do not have time to look at enough examples to convince you that
Turing machines are as powerful as computers, but the proof reduces
to the fact that computers are actually fairly simple in their basic
operation.  Everything that a computer does comes down to copying
data from one place to another, making simple comparisons between
two pieces of data, and performing some basic arithmetic operations.
It's possible for Turing machines to do all these things.  In fact,
it's possible to build a Turing machine to simulate the step-by-step
operation of a given computer.  Doing so proves that the Turing machine
can do any computation that the computer could do, although it will,
of course, work much, much more slowly.

\medbreak

We can, however, look briefly at some other models of computation
and see how they compare with Turing machines.  For example, there
are various ways in which we might try to increase the power of
a Turing machine.  For example, consider a \nw{two-tape Turing machine}
that has two tapes, with a read/write head on each tape.  In each step
of its computation, a two-tape Turing machine reads the symbols under
its read/write heads on both tapes.
Based on these symbols and on its current state, it
can write a new symbol onto each tape, independently
move the read/write head on each tape one cell to the left or
right, and change state.

It might seem that with two tapes available, two-tape Turing machines
might be able to do computations that are impossible for ordinary
one-tape machines.  In fact, though, this is not the case.  The reason,
again, is simulation:  Given any two-tape Turing machine, it is possible
to build a one-tape Turing machine that simulates the step-by-step
computation of the two-tape machine.  Let $M$ be a two-tape Turing
machine.  To simulate $M$ with a one-tape machine, $K$, we must store
the contents of both of $M$'s tapes on one tape, and we must keep
track of the positions of both of $M$'s read/write heads.
Let {\at} and \$ be symbols that are not in the alphabet of $M$.  
The {\at} will be used to mark the position of a read/write head, and
the \$ will be used to delimit the parts of $K$'s tape that
represent the two tapes of $M$.  For example, suppose that one
of $M$'s tapes contains the symbols ``{\tt abb\#\#cca}'' with the
read/write head on the first {\tt b}, and that the other tape contains
``{\tt 01\#111\#001}'' with the read/write head on the final {\tt 1}.  This
configuration would be represented on $K$'s tape as
``{\tt\$a{\at}bb\#\#cca\$01\#111\#00{\at}1\$}''.  To simulate one
step of $M$'s computation, $K$ must scan its entire tape, looking for
the {\at}'s and noting the symbol to the right of each {\at}.  Based on
this information, $K$ can update its tape and its own state to
reflect $M$'s new configuration after one step of computation.
Obviously, $K$ will take more steps than $M$ and it will operate
much more slowly, but this argument makes it clear that one-tape
Turing machines can do anything that can be done by two-tape
machines.

We needn't stop there.  We can imagine $n$-tape Turing machines, for
$n>2$.  We might allow a Turing machine to have multiple read/write
heads that move independently on each tape.  We could even allow
two or three-dimensional tapes.  None of this makes any difference 
as far as computational power goes, since each type of Turing machine
can simulate any of the other types.\footnote{We can also define 
\nw{non-deterministic Turing machines} that can have several possible
actions at each step.  Non-deterministic Turing machines cannot be
used to compute functions, since a function can have only one possible
output for any given input.  However, they can be used to accept
languages.  We say that a non-deterministic Turing machine accepts
a language $L$ is it is \emph{possible} for the machine to halt
on input $w$ if and only if $w\in L$.  The class of languages 
accepted by non-deterministic Turing machines is the same as the
class of languages accepted by deterministic Turing machines.
So, non-determinism does not add any computational power.}

\medbreak

We have used Turing machines to define Turing-acceptable languages
and Turing-decidable languages.  The definitions seem to depend
very much on the peculiarities of Turing machines.  But the same
classes of languages can be defined in other ways.  For example,
we could use programs running on an idealized computer, with an
unlimited amount of memory, to accept or decide languages.  Or we
could use $n$-tape Turing machines.  The
resulting classes of languages would be exactly the same as the
Turing-acceptable and Turing-decidable languages.

We could look at other ways of specifying languages ``computationally.''
One of the most natural is to imagine a Turing machine or computer
program that runs forever and outputs an infinite list of strings
over some alphabet $\Sigma$.  In the case of Turing machines, it's
convenient to think of a two-tape Turing machine that lists the strings
on its second tape.  The strings in the list form a language
over $\Sigma$.  A language that can be listed in this way is
said to be \nw[recursively enumerable language]{recursively enumerable}. 
Note that we make no
assumption that the strings must be listed in any particular order,
and we allow the same string to appear in the output any number of
times.  Clearly, a recursively enumerable language is ``computable''
in some sense.  Perhaps we have found a new type of computable language.
But no---it turns out that we have just found another way of
describing the Turing-acceptable languages.  The following theorem
makes this fact official and adds one more way of describing
the same class of languages:

\begin{theorem}
Let $\Sigma$ be an alphabet and let $L$ be a language over $\Sigma$.
Then the following are equivalent:

\smallskip
\IItem{1.\ }There is a Turing machine that accepts $L$.

\smallskip
\IItem{2.\ }There is a two-tape Turing machine that runs forever, making 
a list of strings on its second tape, such that a string $w$ is in 
the list if and only if $w\in L$.

\smallskip
\IItem{3.\ }There is a Turing-computable function $f\colon\{a\}^*\to\Sigma^*$
such that $L$ is the range of the function $f$.
\end{theorem}

While I will not give a complete, formal proof of this theorem, it's not
too hard to see why it is true.  Consider a language that satisfies
property 3.  We can use the fact that $L$ is the range of a Turing-computable function, $f$,
to build a two-tape Turing machine that lists $L$.  The machine will
consider each of the strings $a^n$, for $n\in \N$, in turn, and it will compute
$f(a^n)$ for each $n$.  Once the value of $f(a^n)$ has been computed, it can be copied
onto the machine's second tape, and the machine can move on to do the same
with $a^{n+1}$.  This machine writes all the elements of $L$ 
(the range of~$f$) onto its second tape,
so $L$ satisfies property 2.  Conversely, suppose that
there is a two-tape Turing machine, $M$, that lists $L$.  Define a function
$g\colon\{a\}^*\to\Sigma^*$ such that for $n\in\N$, $g(a^n)$ is the $(n+1)^{th}$ item in the
list produced by $M$.  Then the range of $g$ is $L$, and $g$ is Turing-computable
since it can be computed as follows:  On input $a^n$, simulate the computation
of $M$ until it has produced $n+1$ strings, then halt, giving the $(n+1)^{th}$
string as output.  This shows that property 2 implies property 3, so these
properties are in fact equivalent.

We can also check that property 2 is equivalent to property 1. 
Suppose that $L$ satisfies property~2. Consider
a two-tape Turing machine, $T$, that lists the elements of $L$.  We must build
a Turing machine, $M$, which accepts $L$. We do this
as follows:  Given an input $w\in\Sigma^*$,
$M$ will simulate the computation of $T$.  Every time the simulated $T$ produces a string
in the list, $M$ compares that string to $w$.  If they are the same, $M$ halts.
If $w\in L$, eventually it will be produced by $T$, so $M$ will eventually halt.
If $w\not\in L$, then it will never turn up in the list produced by $T$, so
$M$ will never halt.  Thus, $M$ accepts the language $L$.  This shows that
property 2 implies property 1.  

The fact that property 1 implies property
2 is somewhat harder to see.  First, we note that it is possible for a Turing
machine to generate every possible string in $\Sigma^*$, one-by-one,
in some definite order (such
as order of increasing length, with something like alphabetical order
for strings of the same length).  Now, suppose that $L$ is Turing-acceptable
and that $M$ is a Turing machine that accepts $L$.  We need a two-tape
Turing machine, $T$ that makes a list of all the elements of $L$.
Unfortunately, the following idea does \textit{not} work:  Generate each
of the elements in $\Sigma^*$ in turn, and see whether $M$ accepts it.
If so, then add it to the list on the second tape.  It looks like we have a machine that
lists all the elements of $L$.  The problem is that the only way for $T$ to
``see whether $M$ accepts'' a string is to simulate the computation of $M$.
Unfortunately, as soon as we try this for any string $w$ that is not in $L$,
the computation never ends!  $T$ will get stuck in the simulation and will
never even move on to the next string.  To avoid this problem, $T$ must simulate
multiple computations of $M$ at the same time.  $T$ can keep track of
these computations in different regions of its first tape (separated by \$'s).
Let the list of all strings in $\Sigma^*$ be $x_1$, $x_2$, $x_3$,~\dots. Then $T$ should
operate as follows:

\smallskip
\IItem{1.\ }Set up the simulation of $M$ on input $x_1$, and simulate one 
          step of the computation for $x_1$

\smallskip
\IItem{2.\ }Set up the simulation of $M$ on input $x_2$, and simulate one 
          step of the computation for $x_1$ and one step of the computation for $x_2$.

\smallskip
\IItem{3.\ }Set up the simulation of $M$ on input $x_3$, and simulate one 
          step of each of the computations, for $x_1$, $x_2$, and $x_3$.

\smallskip
\IItem{}\dots
          
\smallskip
\IItem{n.\ }Set up the simulation of $M$ on input $x_n$, and simulate one 
          step of each of the computations, for $x_1$, $x_2$, \dots, $x_n$.

\noindent and so on.  Each time one of the computations halts, $T$ should
write the corresponding $x_i$ onto its second tape.  Over the course of
time, $T$ simulates the computation of $M$ for each input $w\in\Sigma^*$
for an arbitrary number of steps.  If $w\in L$, the simulated computation will
eventually end and $w$ will appear on $T$'s second tape.  On the other hand,
if $w\not\in L$, then the simulated computation will never end, so $w$ will
not appear in the list.  So we see that $T$ does in fact make a list of all
the elements, and only the elements of $L$.  This completes an outline of
the proof of the theorem.

\medbreak

Next, we compare Turing machines to a completely different method
of specifying languages: general grammars.  Suppose $G=(V,\Sigma,P,S)$ is a general
grammar and that $L$ is the language
generated by $G$.  Then there is a Turing machine, $M$, that accepts
the same language, $L$.  The alphabet for $M$ will be $V\cup\Sigma\cup\{\text{\$,\b}\}$,
where \$ is a symbol that is not in $V\cup\Sigma$. (We also assume that \b\ is not in $V\cup\Sigma$.)
Suppose that $M$ is started with input $w$, where $w\in\Sigma^*$.
We have to design $M$ so that it will halt if and only if $w\in L$.
The idea is to have $M$ find each string that can be derived
from the start symbol $S$.  The strings will be written to $M$'s tape
and separated by \$'s.  $M$ can begin by writing the start symbol,
$S$, on its tape, separated from $w$ by a~\$.  Then it repeats
the following process indefinitely:  For each string on the tape
and for each production rule, $x\PRODUCES y$, of $G$, search the
string for occurrences of $x$.  When one is found, add a \$ to the
end of the tape and copy the string to the end of the tape, replacing
the occurrence of $x$ by $y$.  The new string represents the results
of applying the production rule $x\PRODUCES y$ to the string.
Each time $M$ produces a new string, it compares
that string to $w$.  If they are equal, then $M$ halts.  If $w$ is
in fact in $L$, then eventually $M$ will produce the string $w$ and
will halt.  Conversely, if $w$ is not in $L$, then $M$ will go on producing
strings forever without ever finding $w$, so $M$ will never halt.
This shows that, in fact, the language $L$ is accepted by $M$.

Conversely, suppose that $L$ is a language over an alphabet $\Sigma$,
and that $L$ is Turing-acceptable.  Then it is possible to find a grammar
$G$ that generates $L$.  To do this, it's convenient to use the
fact that, as discussed above, there is a Turing-computable function
$f\colon \{a\}^*\to\Sigma$ such that $L$ is the range of~$f$.
Let $M=(Q,\Lambda,q_0,\delta)$ be a Turing machine that computes
the function $f$.  We can build a grammar, $G$, that imitates the computations
performed by $M$.  The idea is that most of the production rules of $G$ will
imitate steps in the computation of $M$.  Some additional rules are added
to get things started, to clean up, and to otherwise bridge the
conceptual gap between grammars and Turing machines.

The terminal symbols of $G$ will be the symbols from the alphabet,~$\Sigma$. 
For the non-terminal symbols,
we use: the states of $M$, every member of $\Lambda$ that is not
in $\Sigma$, two special symbols $<$ and $>$, and two additional
symbols $S$ and $A$.  (We can assume that all
these symbols are distinct.)  $S$~will be the start symbol of $G$.
As for production rules, we begin with the following three rules:
\begin{align*}
   S&\PRODUCES \hbox{$<$}q_0A\hbox{$>$}\\
   A&\PRODUCES aA\\
   A&\PRODUCES\EMPTYSTRING
\end{align*}
These rules make it possible to produce any string of the form
$<${}$q_0a^n${}$>$.  This is the only role that $S$ and $A$ play
in the grammar.  Once we've gotten rid of $S$ and $A$, strings
of the remaining terminal and non-terminal symbols represent
configurations of the Turing machine $M$.  The string will contain
exactly one of the states of $M$ (which is, remember, one of the
non-terminal symbols of $G$). This tells us which state $M$ is
in.  The position of the state-symbol tells us where
$M$ is positioned on the tape: the state-symbol is located
in the string to the left of the symbol on which $M$ is positioned.
And the special symbols $<$ and $>$ just represent
the beginning and the end of a portion of the tape of $M$.
So, the initial string $<${}$q_0a^n${}$>$ represents
a configuration in which $M$ is in its start state, and
is positioned on the first $a$ in a string of $n$ $a$'s.
This is the starting configuration of $M$ when it is run
with input $a^n$.

Now, we need some production
rules that will allow the grammar to simulate the computations
performed by $M$.  For each state $q_i$ and each symbol $\sigma\in\Lambda$,
we need a production rule that imitates the transition rule $\delta(q_i,\sigma)
=(\tau,d,q_j)$.  If $d=R$, that is if the machine moves to the right,
then all we need is the rule
\begin{align*}
   q_i\sigma&\PRODUCES \tau q_j
\end{align*}
This represents that fact that $M$ converts the $\sigma$ to a $\tau$,
moves to the right, and changes to state $q_j$.  If $d=L$, that is
if the machine moves to the left, then we will need several rules---one rule for
each $\lambda\in\Lambda$, namely
\begin{align*}
   \lambda q_i\sigma&\PRODUCES q_j\lambda\tau
\end{align*}
This rule says that $M$ changes the $\sigma$ to a $\tau$, moves left,
and changes to state $q_j$.  The $\lambda$ doesn't affect the
application of the rule, but is necessary to represent the fact
that $M$ moves left.

Each application of one of these rules represents one step in
the computation of $M$.   There is one remaining requirement for correctly
simulating $M$.  Since $M$'s tape contains an infinite number of cells
and we are only representing a finite portion of that tape, we need a way
to add and remove \b's at the ends of the string.  We can use the
following four rules to do this:
\begin{align*}
   \hbox{$<$}&\PRODUCES \hbox{$<$\b}\\
   \hbox{$<$\b}&\PRODUCES \hbox{$<$}\\
   \hbox{$>$}&\PRODUCES \hbox{\b$>$}\\
   \hbox{\b$>$}&\PRODUCES \hbox{$>$}\\
\end{align*}
These rules allow blank symbols to appear at the ends of the string
when they are needed to continue the computation, and to disappear
from the ends of the string whenever we like.

Now, suppose that $w$ is some element of $L$.  Then $w=f(a^n)$ for some $n\in\N$.
We know that on input $a^n$, $M$ halts with output $w$.  If we
translate the computation of $M$ into the corresponding sequence
of production rules in $G$,
we see that for the grammar $G$, $<${}$q_0a^n${}$>$
$\YIELDSTAR$ $<${}$hw${}$>$, where $h$ is the halt state of $M$.
Since we already know that $S$ $\YIELDSTAR$ $<${}$q_0a^n${}$>$,
for every $n\in\N$, we see that in fact $S$ $\YIELDSTAR$ 
$<${}$hw${}$>$ for each $w\in L$.  We almost have it! We
want to show that $S$~$\YIELDSTAR$~$w$.
If we can just
get rid of the $<$, the $h$, and the $>$, we will have that
$<${}$hw${}$>$ $\YIELDSTAR$ $w$ and we can then deduce that
$S$~$\YIELDSTAR$~$w$ for each $w\in L$, as desired.  We can do this by adding 
just a few more rules to $G$.  We want to let the $h$ eliminate the $<$,
move through the $w$, and then eliminate the $>$ along with itself.
We need the rules
\begin{align*}
   \hbox{$<$}h&\PRODUCES h\\
   h\hbox{$>$}&\PRODUCES \EMPTYSTRING
\end{align*}
and, for each $\sigma\in\Sigma$,
\begin{align*}
   h\sigma&\PRODUCES \sigma h
\end{align*}
We have constructed $G$ so that it generates every string in $L$.
It is not difficult to see that the strings in $L$ are in fact the
only strings that are generated by $G$.  That is, $L$ is precisely
$L(G)$.

We have now shown, somewhat informally, that a language
$L$ is Turing-acceptable if and only if there is a grammar $G$
that generates $L$.  Even though Turing machines and grammars
are very different things, they are equivalent in terms of
their ability to describe languages.  We state this as a theorem:
\begin{theorem}
A language $L$ is Turing acceptable (equivalently, recursively enumerable)
if and only if there is a general grammar that generates $L$.
\end{theorem}

\bigskip

In this section, we have been talking mostly about recursively enumerable
languages (also known as the Turing-acceptable languages).  What
about the Turing-decidable languages?  
We already know that if a language $L$ is Turing-decidable,
then it is Turing-acceptable.  The converse is not true (although
we won't be able to prove this until the next section).  However, suppose
that $L$ is a language over the alphabet $\Sigma$ and that both
$L$ and its complement,  $\overline{L}=\Sigma^*\SETDIFF L$, are Turing-acceptable.
Then $L$ is Turing-decidable.

For suppose that $M$ is a Turing machine that accepts the language
$L$ and that $M'$ is a Turing machine that accepts $\overline{L}$.
We must show that $L$ is Turing-decidable.  That is,
we have to build a Turing machine $T$ that decides $L$. For each
$w\in\Sigma^*$, when $T$ is run with input $w$, it should halt with
output 1 if $w\in L$ and with output $0$ if $w\not\in L$.  To do this,
$T$ will simulate the computation of both $M$ and $M'$ on input $w$.
(It will simulate one step in the computation of $M$, then one step
in the computation of $M'$, then one step of $M$, then one step of $M'$,
and so on.)  If and when the simulated computation of $M$ halts, then
$T$ will halt with output~1; since $M$ accepts $L$, this will happen if and
only if $w\in L$.  If and when the simulated computation of $M'$ halts, then
$T$ will halt with output~0; since $M$ accepts $L$, this will happen if and
only if $w\not\in L$.  So, for any $w\in\Sigma^*$, $T$ halts with the
desired output.  This means that $T$ does in fact decide the language $L$.

It is easy to prove the converse, and the proof is left as an exercise. So
we see that a language is Turing-decidable if and only if both it and
its complement are Turing-acceptable.  Since Turing-acceptability can
be defined using other forms of computation besides Turing machines,
so can Turing-decidability.  For example, a language is Turing-decidable
if and only if both it and its complement can be generated by general grammars.
We introduced the term ``recursively enumerable''
as a synonym for Turing-acceptable, to get away from the association with a
particular form of computation.  Similarly, we define the term ``recursive''
as a synonym for Turing-decidable.  That is, a language $L$
is said to be \nw[recursive language]{recursive} if and only if it
is Turing-decidable.  We then have the theorem:

\begin{theorem}\label{T-re}
Let $\Sigma$ be an alphabet and let $L$ be a language over $\Sigma$.
Then $L$ is recursive if and only if both $L$ and its
complement, $\Sigma^*\SETDIFF L$, are recursively enumerable.
\end{theorem}

\begin{exercises}

\problem The language $L=\{a^m\st m>0\}$ is the range of the function
$f(a^n)=a^{n+1}$.  Design a Turing machine that computes this function,
and find the grammar that generates the language $L$ by
imitating the computation of that machine.

\problem Complete the proof of Theorem \ref{T-re} by proving
the following:  If $L$ is a recursive language over an
alphabet $\Sigma$, then both
$L$ and $\Sigma^*\SETDIFF L$ are recursively enumerable.

\problem Show that a language $L$ over an alphabet $\Sigma$
is recursive if and only if there are grammars $G$
and $H$ such that the language generated by $G$ is $L$ and the
language generated by $H$ is $\Sigma^*\SETDIFF L$.

\problem This section discusses recursive languages and recursively
enumerable languages.  How could one define recursive subsets of
$\N$ and recursively enumerable subsets of $\N$?

\problem Give an informal argument to show that a subset $X\SUB\N$ is
recursive if and only if there is a computer program
that prints out the elements of $X$ {\it in increasing order}.

\end{exercises}



\section{The Limits of Computation}\label{S-turing-3}

Recursively enumerable languages are languages that can be defined by computation.
We have seen that there are many different models of compu\-tation---Turing machines,
two-tape Turing machines, grammars, computer programs---but they all lead
to the same class of languages.  In fact, every computational method for
specifying languages that has ever been developed produces only recursively
enumerable languages.  There is something about these languages---some pattern
or property---that makes them ``computable,''  and it is some intrinsic
property of the languages themselves, not some peculiarity of any given
model of computation.

This is especially interesting since most languages are not recursively enumerable.
Given an alphabet $\Sigma$, there are uncountably many languages over $\Sigma$, but
only countably many of them are recursively enumerable.  The rest---the vast
majority---are not recursively enumerable.  What can we say about
all these non-recursively-enumerable languages?  If the language $L$ is not
recursively enumerable, then there is no algorithm for listing the members of
$L$.  It might be possible to define $L$ by specifying some property that
all its members satisfy, but that property can't be computable.  That is, there
can be no computer program or Turing machine that tests whether a given
string $w$ has the property, since if there were, then we could write a
program that lists the members of $L$.

So, even though almost every language is non-recursively-enumerable, it's
difficult to find a particular language that is not recursively enumerable.
Nevertheless, in this section we will find one such language.  At that same
time, we will find an example of a language that is recursively enumerable
but not recursive.  And we will discover some interesting limitations to
the power of computation.

\medskip

The examples that we will look at in this section involve Turing
machines that work with other Turing machines as data.  For this to
work, we need a symbolic representation of Turing machines---a
representation that can be written on the tape of another Turing
machine.  This will let us create two machines:  First,
a Turing machine that can generate Turing machines
on demand by writing their symbolic representations on its tape.
We will design a Turing machine $G$ to do this.  And second,
a Turing machine that can simulate the computation of other
Turing machines whose descriptions are written on its tape.


In order to do all this, we must put some limitations on
the states and alphabetic symbols that can be used in the Turing machines
that we consider.
Clearly, given any Turing machine, we can change the names of the
states without changing the behavior of the machine.  So, without any
loss of generality, we can assume that all states have names chosen
from the list: $h$, $q$, $q'$, $q''$, $q'''$, $q''''$,~\dots.
We assume that $h$ is the halt state and $q$ is the start state.
Note that there is an infinite number of possible states, but any
given Turing machine will only use finitely many states from this
list.  

As for the alphabets of the Turing machines, I want to look at
Turing machines whose alphabets include the symbols 0, 1, $a$, and of
course~\b.
These are the symbols that the machines will use for input and output.
The alphabets can also include other symbols.  We will assume that
these auxiliary symbols are chosen from the list: $a'$, $a''$, $a'''$,
$a''''$,~\dots.  Given a Turing machine whose alphabet includes
the symbols 0, 1, $a$, and \b, we can rename any other symbols in its
alphabet using names from this list.  This renaming will not
affect any of the behavior that we are interested in.

Now suppose we have one of these standard Turing machines---one
whose states are chosen from the list $h$, $q$, $q'$, $q''$, $q'''$,~\dots,
whose start state is $q$, and whose symbols are chosen from the list
\b, 0, 1, $a$, $a'$, $a''$, $a'''$,~\dots.  Such a machine can be
completely encoded as a string of symbols over the alphabet
$\{h,q,L,R,\b,0,1,a,{}',\text{\tt\$}\}$.  A transition rule
such as $\delta(q'',0)=(a''',L,q)$ can be encoded as a
string $q''0a'''Lq$.  To encode a complete machine, simply encode
each of its transition rules in this way and join them together in a string,
separated by {\tt\$}'s.  We now have the symbolic representation for
Turing machines that we need.

Note that a string over the alphabet $\{h,q,L,R,\b,0,1,a,{}',\text{\$}\}$
might or might not encode a Turing machine.  However, it is a simple
matter to check whether such a string is the code for a Turing machine.
We can imagine the following process:  Generate all the strings over the
alphabet $\{h,q,L,R,\b,0,1,a,{}',\text{\$}\}$.  Check each string
to see whether it encodes a Turing machine.  If so, add the string
to an output list. In this way, we can generate a list of
all strings that encode standard Turing machines.  In effect,
the standard Turing machines, or at least their symbolic representations,
form a recursively enumerable set.  Let $T_0$ be the
machine encoded by the first string in this list of
standard Turing machines; let $T_1$ be
the machine encoded by the second string; let $T_2$ be the
machine encoded by the third string; and so on.  The list
$T_0$, $T_1$, $T_2$,~\dots, includes every standard Turing machine.
Furthermore, given $n\in\N$, we can find the symbolic representation
for $T_n$ by generating strings in the list until we have $n+1$ strings.
Furthermore---and this is the essential point---we can use a Turing
machine to do all these calculations.  In fact, there is
a Turing machine that, when run with input $a^n$, will halt with
the string representation of $T_n$ written on its tape as output.
The Turing machine that does this is $G$, the first of the
two machines that we need.

The second machine that we need will be called $U$.  It is a
so-called \nw{Universal Turing Machine}.  The single Turing machine
$U$ can simulate the computation of any standard Turing machine, $T$,
on any input.  Both the symbolic representation of $T$ and that of
the input string are written to $U$'s tape, separated by a
space.  As $U$ simulates the computation of $T$, it will need
some way to keep track of what state $T$ is in and of
the position of $T$ on its (simulated) tape.  It does this
by writing the current state of $T$ on its tape, following
$T$'s input string, and by adding a special symbol, such as \at,
to the input string to mark $T$'s position.  When $U$ is first
started, it begins by adding the \at\ to the beginning of the
input string and writing a $q$ after the string to represent
the start state of $T$.  It is then relatively straightforward
for $U$ to simulate the computation of $T$.  For each step
in the computation of $T$, it can determine the current state
of $T$ (which is recorded on $U$'s tape) and the symbol which
$T$ is currently reading (which is on $U$'s tape, after the \at).
$U$ searches the symbolic representation of $T$ for the
rule that tells $T$ what to do in this situation.  Using
this rule, $U$ can update its representation of $T$'s state,
position, and tape to reflect the result of applying the rule.
If the new state of $T$ is the halt state, then $U$ also halts.
Otherwise, it goes on to simulate the next step in $T$'s computation.
Note that when $U$ is given $T$ and an input string $w$ as
input, $U$ will halt if and only if $T$ halts on input $w$.
(Obviously, this is a very inefficient simulation, but we
are not concerned with efficiency here.)

So, we have our two machines, $G$ and $U$.
After all this setup, we are finally in a position to look at
the major theorem that we have been working towards.

\begin{theorem}
Let $T_0$, $T_1$, $T_2$, \dots, be the standard Turing machines,
as described above.  Let $K$ be the language over the alphabet $\{a\}$
defined by $$K=\{a^n\st\,T_n\text{ halts when run with input }a^n\}.$$
Then $K$ is a recursively enumerable language, but $K$ is not
recursive.  The complement $$\overline{K}=\{a^n\st\,T_n\text{ does
not halt when run with input }a^n\}.$$
is a language that is not recursively enumerable.
\end{theorem}

First note that if both $K$ and $\overline{K}$ were recursively
enumerable, then $K$ would be recursive, by Theorem~\ref{T-re}.
So, once we show that $K$ is recursively enumerable but not
recursive, it follows immediately that $\overline{K}$ cannot
be recursively enumerable.  That is, the second part of the
theorem follows from the first.

To show that $K$ is recursively enumerable, it suffices to find
a Turing machine, $M$, that accepts $K$.  That is, when run
with input $a^n$, for $n\in\N$, $M$ should halt if and only if
$a^n\in K$.  We can build $M$ from the Turing machines $G$ and $U$
which were introduced above.  When started with input $a^n$, 
$M$ should proceed as follows:  
First copy the input.  Run $G$ on the
first copy of $a^n$.  This will produce a symbolic description
of the Turing machine $T_n$.  Now run $U$ to simulate the 
computation of $T_n$ on input $a^n$.  This simulation will end
if and only if $T_n$ halts when run with input $a^n$, that is, if and only
if $a^n\in K$.  The Turing machine $M$ that performs the computation
we have described accepts the language $K$.
This proves that $K$ is recursively enumerable.

To show that $K$ is not recursive, we need to show that
there is \textit{no} Turing machine that decides $K$.  
Let $H$ be any Turing machine.  We must show that
no matter what $H$ does, it
does not decide the language $K$.  We must do this without
knowing anything more about $H$ that the fact that is it
a Turing machine.  To say that $H$ decides
$K$ would mean that for any $n\in\N$, when $H$ is
run with input $a^n$, $H$ will halt with output 1 if $a^n\in K$
and will halt with output 0 if $a^n\not\in K$.  To show that
$H$ does not decide $K$ we need to show that there is some
$n\in\N$ such that when $H$ is run with input $a^n$, 
$H$ either fails to halt or else halts but gives the wrong 
output.  Note in particular that we only need to find \textit{one}
$n$ for which $H$ does not give the correct result.
As we try to find $n$, we have nothing much to work with
but $H$ itself.\looseness=-1

To find $n$, we construct a Turing machine $M$ that is a simple
variation on $H$.  When $M$ is run on any input, it duplicates the behavior
of $H$ on that input until $H$ halts (if it ever does).  At that point, $M$ should
check $H$'s output.  If $H$ has halted with output $1$, then 
$M$ should go into an infinite loop, so that $M$ never halts in this case.
Otherwise, if the output of $H$ is not $1$, then $M$ should halt.
Now, we can assume that $M$ is one
of the standard Turing machines, say $M=T_n$.  (If $M$ is not
already one of these machines, it is because it uses different names
for its states and symbols.  Renaming the states and symbols will
produce an equivalent machine with the same behavior as $M$,
and we can replace $M$ with this standard machine.)  

We now have a Turing machine $T_n = M$ which has the following behavior
when it is run with input $a^n$ (note that the $n$ here is the same $n$ as
in $T_n$):
If $H$ halts with output 1 on input $a^n$, then $T_n$
will fail to halt on input $a^n$.  If $H$ halts with output 0
on input $a^n$, then $T_n$ fails to halt on input $a^n$.  (What $T_n$
might do in other cases is not relevant here.)

Remember that we are trying to show that $H$ does not decide the language
$K$.  I claim that, in fact, $H$ does not give the correct answer for $a^n$.  When $H$
is run with input $a^n$, it is supposed to halt with output 1 if $a^n\in K$,
and it is supposed to halt with output 0 if $a^n\not\in K$.  Recall that
$a^n\in K$ if and only if $T_n$ halts when run with input $a^n$.

Suppose that we run $H$ with input $a^n$.
If $H$ does not halt with output 0 or 1, then it has certainly not given the 
correct answer for $a^n$.  Now, suppose that $H$ halts with output 1 on input $a^n$.
In this case, by the properties of $T_n$ given above, we know that $T_n$ does not
halt on input $a^n$.  But that means, by definition of $K$, 
that $a^n\not\in K$.  By halting with output
1 in this case, $H$ has given the wrong answer for $a^n$.  Finally, suppose that
$H$ halts with output 0 on input $a^n$.  We then know that $T_n$ halts
on input $a^n$.  But that means that $a^n\in K$.  Again, by halting with output
0 in this case, $H$ has given the wrong answer for $a^n$.  So, in no case will
$H$ give the correct answer for $a^n$.  
This means that
$H$ does not decide the language $K$, because $H$ gives
an incorrect answer when it is run with the particular input $a^n$.
$H$ does not decide $K$, and since
$H$ was an arbitrary Turing machine, we see that there is
no Turing machine at all that decides the language $K$.  Thus,
$K$ is not a recursive language, as the theorem claims. 

\medbreak

To decide the language $K$ would be to solve the following
problem:  Given a Turing machine $T_n$, decide whether or
not $T_n$ will halt when it is run with input $a^n$.  This
problem is called the \nw{Halting Problem}.  We have shown
that there is no Turing machine that solves this problem.
Given the equivalence of Turing machines and computer programs,
we can also say that there is no computer program that
solves the halting problem.  We say that the halting problem
is \nw[computational unsolvability]{computationally unsolvable}.

The halting problem is just one
of many problems that cannot be solved by Turing machines or
computer programs.  In fact, almost any interesting yes/no
question that can be asked about Turing machines or programs
is in this class:  Does this Turing machine halt for all possible
inputs in $\Sigma^*$?  Given this input, will this program
ever halt?  Do these two programs (or Turing machines) have
the same output for each possible input?  Will this Turing
machine ever halt if it is started on a blank tape?
All these problems are computationally unsolvable in the
sense that there is no Turing machine or computer program
that will answer them correctly in all cases. The
existence of such problems is a real limitation on the
power of computation.






\endinput

% !TEX root = main.tex

\chapter{Sequential Logic}

\input{bth-logic-seq}

% !TEX root = main.tex

\chapter{Functional Programming}
\input{fp-scala.tex}

%Hold off on the following, because Creative Scala probably does it better
%\input{fp-drawing.tex}

%\input{fp-list.tex}

%\input{fp-stream.tex}


\printindex

\end{document}

